\documentclass[parskip=full,11pt]{scrartcl}
\usepackage[utf8]{inputenc}
\usepackage[T1]{fontenc}
\usepackage[german]{babel}
\usepackage[useregional]{datetime2}
\usepackage[pdfborderstyle={/S/U/W 0}]{hyperref}
\usepackage[nameinlink]{cleveref}
\usepackage[section]{placeins}
\usepackage{xcolor}
\usepackage{graphicx}
\usepackage{csquotes}
\usepackage{amsmath} % for $\text{}$
\usepackage{pflichtenheft}
\usepackage{enumitem}
\setlist{nosep}

\newcommand\urlpart[2]{$\underbrace{\text{\texttt{#1}}}{\text{#2}}$}
\raggedbottom
\crefname{figure}{Abb}{Abb}

\newcommand\producttitle{treff.}
\hypersetup{
	pdftitle={Pflichtenheft: \producttitle},
	bookmarks=true,
}

% section numbers in margins:
\renewcommand\sectionlinesformat[4]{\makebox[0pt][r]{#3}#4}

% header & footer
\usepackage{scrlayer-scrpage}
\lofoot{\today}
\refoot{\today}
\pagestyle{scrheadings}

\title{Pflichtenheft: \producttitle}
\author{Lukas Dippon
        \and Jens Kienle
        \and Matthias Noll
        \and Fabian Röpke
        \and Tim Schmidt
        \and Simon Vögele}

\begin{document}
\maketitle
\thispagestyle{empty} % removed page number from title

\section*{} % No section number on intro, also removes it from TOC
Studenten und Mitarbeiter des KIT treffen sich gerne zum gemeinsamen Essen oder Lernen.
Dazu ist an einem bestimmten Tag immer notwendig zu wissen, ob bereits ein Treffen vereinbart wurde.
Am Treffpunkt angekommen, möchte man wissen, an welchem Ort sich die Gruppe befindet, um nicht lange suchen zu müssen.
Unsere App soll es angemeldeten Benutzern ermöglichen, sich in Gruppen zu organisieren.
In den Gruppen kann der Zeit- und Treffpunkt bestimmt werden.
Nach der Festlegung des Treffens soll die App die GPS-Standorte der Mitglieder anzeigen, um das Treffen zu vereinfachen.

\pagebreak
\tableofcontents

%%%%%%%%%%%%%%
\pagebreak
\section{Kriterien}
% Diese Section sollte kurz und knapp "für Manager" sein
% und auf eine Seite passen.

\subsection{Pflichtkriterien}
\criterion{Gruppen}{groups}{10}
Mehrere Nutzer können sich in Gruppen zusammenschließen.
Ein Nutzer kann in beliebig vielen Gruppen Mitglied sein.

\criterion{Eigene Position auf Umgebungskarten}{location}{20}
Der Nutzer ist in der Lage, seine aktuelle Position auf einer Karte der nahen
Umgebung einzusehen.

\criterion{Festlegbare Treffpunkte}{meetingpoints}{30}
Innerhalb von Gruppen können Treffpunkte festgelegt werden,
die den Mitgliedern auf der Karte angezeigt werden.

\criterion{Gegenseitige Ortung}{positions}{40}
Nutzer sind in der Lage, ihre aktuelle Position mit anderen Gruppenmitgliedern
über einen festlegbaren Zeitraum zu teilen.
Diese wird dann allen Gruppenmitgliedern auf der Karte angezeigt und laufend
aktualisiert.

\criterion{Datenschutz}{privacy}{50}
Die Positionsdaten des Nutzers werden nur während des festgelegten Zeitraumes
(\critLink{positions}) gesendet und serverseitig nicht
längerfristig gespeichert.

\criterion{Nutzerkonten}{account}{60}
Ein persönliches Nutzerkonto ermöglicht es Nutzern,
Gruppenzugehörigkeiten und Datenschutzoptionen zu speichern.

\subsection{Kürkriterien}
\criterionOpt{Abstimmungen}{vote}{10}
Innerhalb von Gruppen können die Nutzer über nächste Treffpunkte abstimmen.

\criterionOpt{1 zu 1 Verbindung}{1to1}{20}
Der Positionsaustausch zwischen zwei einzelnen Nutzern außerhalb von Gruppen
soll möglich und besonders einfach sein.

\criterionOpt{Detailliertere Innenraumkarten}{indoor}{30}
Innenräume mit hoher Nutzerdichte bekommen detaillierte Innenraumkarten.

\criterionOpt{Gruppenchat}{chat}{40}
Innerhalb der Gruppen können simple Textnachrichten versendet werden.

\criterionOpt{Eigene Server}{ownServer}{50}
Es ist dem Nutzer möglich, einen eigenen Server zu erstellen und zu verwalten.

\criterionOpt{Server-Wechsel}{switchServer}{60}
Um \critLink{ownServer} ausnutzen zu können, ist es dem Nutzer
möglich, den verwendeten Server zu wechseln.

\criterionOpt{Zeitplan für Positionsübertragung}{timetable}{70}
Der Nutzer kann zu bestimmten Zeiten seine Position automatisch an bestimmte
Gruppen übertragen lassen.

\criterionOpt{Passwortzurücksetzung}{passwordReset}{80}
Der Nutzer kann sein Passwort zurücksetzen lassen, wenn er sich über eine zuvor
festgelegte E-Mail-Adresse authentifizieren kann.

\subsection{Abgrenzungskriterien}
\criterionDemarc{Kein Social Network}{socialNetwork}{10}
Die App bietet keine Plattform um Bilder oder andere persönliche Eindrücke mit
anderen zu teilen.

\criterionDemarc{Kontakt nur zu Freunden}{friendsOnly}{20}
Die App möchte die Kommunikation in bereits bestehenden Gruppen vereinfachen,
nicht den Kontakt zu Fremden ermöglichen.

\criterionDemarc{Beidseitiges Einverständnis}{consensual}{30}
Die Anwendung ermöglicht nicht das Orten von Personen,
die ihre Position nicht ausdrücklich zugänglich machen.
Beispielsweise können Eltern mit dieser App nicht ständig ihre Kinder orten.

%%%%%%%%%%%%%%
\pagebreak
\section{Produkteinsatz}
Das Produkt dient zur Erleichterung des Zusammenfindens und der
Treffpunktvereinbarung in geschlossenen Gruppen von Freunden, Kollegen oder
Familienmitgliedern.

\subsection{Anwendungsbereiche}
\begin{itemize}
    \item Private Treffen in der Freizeit oder während der Arbeit
    \item Zusammenfindung auf größeren Veranstaltungen
\end{itemize}

\subsection{Zielgruppen}
\begin{itemize}
    \item Freundesgruppen
    \item Arbeitskollegen
    \item Familien
\end{itemize}

\subsection{Betriebsbedingungen}
\begin{itemize}
    \item Mobiler Einsatz in Umgebungen mit GPS-Empfang und Zugang zum Internet
\end{itemize}

\subsection{Anwendungsfallszenarien}
\begin{description}
\item[Familienfeier]
An Weihnachten kann es oft schwer sein, ein Essen mit der ganzen Familie zu organisieren.
Auch, wenn die Planung schon abgeschlossen ist, kann die App eine sinnvolle Hilfe darstellen:

Die Verabredung "Weihnachtsessen" ist bereits erstellt und alle Mitglieder der Gruppe möchten daran teilnehmen.
Nach und nach treffen alle beim Gastgeber ein. Allerdings schafft es Bob nicht, zur vereinbarten Uhrzeit zu erscheinen.
Aufgrund von Stau auf der Autobahn verspätet er sich.
Als pflichtbewusster Autofahrer bedient er sein Handy nicht während der Fahrt.
Stattdessen hat er bereits vor Antritt der Fahrt seinen Standort freigegeben.
Dies erkennt auch der Gastgeber auf der Karte von treff. So kann er den Kochvorgang dann entsprechend anpassen.

\item[Abendprogramm (nutzt auch Kürkriterien)]
Studenten treffen sich auch Abseits des Studiums gerne, um den Abend gemeinsam zu verbringen.
Die Nutzung der Anwendung kann dabei folgendermaßen aussehen:

Nachdem sich die Gruppe entschieden hat, am Abend etwas zu unternehmen,
können über den Chat Ideen gesammelt werden, wann und wo man sich treffen solle.
Anschließend wird eine Abstimmung gestartet, an der sich jeder beteiligen kann.
Währenddessen nutzt Alice die Chatfunktion, um die Gruppe darüber zu informieren, dass sie ihren neuen Freund Bob mitbringt.
Durch die Abstimmung wurde inzwischen eine Verabredung bestimmt.
Zudem hat sich ein Teil der Gruppe dazu entschieden, sich vor der eigentlichen Verabredung noch bei Alice zu Hause zu treffen.
Am Abend schalten alle Mitglieder, vor Beginn des Treffens, die Standortübermittlung frei.
Der Teil der Gruppe, der nicht bei Alice ist, erkennt nun auf der Karte,
dass kurz vor Beginn der Verabredung noch niemand bei Alice losgegangen ist.
Um nicht lange unwissend warten zu müssen,
gehen nun alle zu Alice, um von dort aus dann gemeinsam zum verabredeten Ort weiterzuziehen.
\end{description}

%%%%%%%%%%%%%%
\pagebreak
\section{Produktumgebung}
Das Produkt beinhaltet ein Client-Server-Modell.
Der Server dient lediglich als Vermittler zwischen Klienten.
Die Nutzer verwenden den Klienten auf ihrem mobilen Endgerät.
\subsection{Software}
\begin{itemize}
    \item Client: Android 6 (\enquote{Marshmallow}, API-Level 23) oder
        höher.
    \item Server: Apache Tomcat 8
\end{itemize}

\subsection{Hardware}
% TODO: Mindestanforderungen an Hardware-Performance
\begin{itemize}
    \item Client: Android-fähiges mobiles Endgerät mit
        \begin{itemize}
            \item GPS-Empfänger
            \item Netzwerkkarte mit WLAN-Modul oder Mobilfunkeinheit
        \end{itemize}
    \item Server: Handelsüblicher Servercomputer, der Apache Tomcat 8
        unterstützt und eine Internetanbindung besitzt.
\end{itemize}

%%%%%%%%%%%
\pagebreak
\section{Funktionale Anforderungen}

\subsection{Pflichtanforderungen}

\functionality{Benutzerkonten}{account}{10}
\fulfills{account}%
Der Benutzer wird beim Starten der Anwendung aufgefordert, sich ein
Benutzerkonto zu erstellen oder sich mit einem bereits vorhandenen anzumelden.

\functionality{Gründen von Gruppen}{groupfounding}{20}
\fulfills{groups}%
Nutzer können Gruppen erstellen und diese verwalten, indem sie andere Nutzer
hinzufügen oder entfernen.

\functionality{Umgebungskarte}{map}{30}
\fulfills{location}%
Der Nutzer kann eine Karte aufrufen, auf dem seine eigene Position angezeigt
wird.

\functionality{Erstellung von Verabredungen innerhalb von Gruppen}
    {meetingpoints}{40}
\fulfills{meetingpoints}%
Der Nutzer kann innerhalb einer Gruppe eine Verabredung, bestehend aus einer
Zeit und einem Treffpunkt, erstellen, der allen Gruppenmitgliedern auf der
Umgebungskarte (\fncLink{map}) angezeigt wird.
Außerdem legt er einen Zeitpunkt vor dem Treffen fest,
zu dem die Teilnehmer gebeten werden sollen,
ihren Standort freizugeben (\fncLink{positions}).

\functionality{Übertragung der eigenen Position an die Gruppe}
    {positions}{50}
\fulfills{positions}%
Der Nutzer kann seine aktuelle Position an die Gruppenmitglieder übertragen.
Übertragene Positionen werden allen Gruppenmitgliedern auf der Umgebungskarte
(\fncLink{map}) angezeigt.
Steht eine Verabredung an, wird der Nutzer automatisch gebeten, seine Position
für die Dauer der Verabredung freizugeben.

\functionality{Standortanfrage}{request}{60}
\fulfills{positions}%
Der Nutzer kann in einer Gruppe eine Standortanfrage mit spezifiziertem
Zeitraum senden.
Die Mitglieder der Gruppe werden dann von der App gebeten,
ihren Standort für den gewünschten Zeitraum freizugeben.

\functionality{Verwaltung von Kontodaten}{options}{70}
\fulfills{account}%
Der Nutzer soll folgende Änderungen an seinem Konto vornehmen können:
\begin{itemize}
		\item Nutzername ändern
    \item Mit dem Konto verbundenes Passwort ändern
    \item Mit dem Konto verbundene E-Mail-Adresse ändern
    \item Konto löschen
\end{itemize}

\functionality{Verabredungsliste}{eventlist}{80}
\fulfills{groups}%
\fulfills{meetingpoints}%
Der Nutzer kann eine Liste aufrufen, in der alle Verabredungen aller Gruppen
des Nutzers nach Zeitpunkt sortiert angezeigt werden.

\subsection{Küranforderungen}

\functionalityOpt{Abstimmungen}{vote}{10}
\fulfills{vote}%
Der Nutzer kann innerhalb von Gruppen Abstimmungen erstellen.
Diese werden zeitlich begrenzt.
Der Abstimmung lassen sich beliebig viele Treff- und Zeitpunkte hinzufügen,
sowie Stimmen abgeben und verändern.
Alle zur Wahl stehenden Treffpunkte werden dabei als solche auf der
Umgebungskarte (\fncLink{map}) markiert.
Nach Ablauf des Zeitlimits wird eine Verabredung, bestehend aus jeweils dem
Treff- und Zeitpunkt mit den meisten Stimmen, erstellt.
Alle anderen Treff- und Zeitpunkte werden verworfen.

\functionalityOpt{Gruppenchat}{chat}{20}
\fulfills{chat}%
Jede Gruppe verfügt über einen Chat, in dem simple Nachrichten an alle
Gruppenmitglieder versandt werden können.

\functionalityOpt{Gruppenrollen und -rechte}{admin}{30}
Innerhalb einer Gruppe können unterschiedliche Rollen
mit spezifischen Rechten definiert und an Mitglieder verteilt werden.

\functionalityOpt{Detaillierte Innenraumkarten}{indoor}{40}
\fulfills{indoor}%
Für Innenräume, in denen eine hohe Nutzerdichte erwartet wird, werden
detaillierte Innenraumkarten in die Umgebungskarte integriert. Die Menge an
Innenraumkarten kann mit weiteren, späteren Versionen noch erweitert werden.

\functionalityOpt{Eigene Server}{ownServer}{50}
\fulfills{ownServer}%
Es soll möglich sein, einen eigenen Server einzurichten und diesen statt des
standardmäßig zur Verfügung stehenden Servers zu verwenden.
Zwischen verschiedenen Servern findet keine Kommunikation statt.
Insbesondere werden Nutzerkonten nur serverweit gespeichert.
Für die Einrichtung eines eigenen Servers wird keine Hardware zur Verfügung
gestellt und die benötigten Systemadministrationskenntnisse vorrausgesetzt.

\functionalityOpt{Serverwechel}{switchServer}{60}
\fulfills{switchServer}%
Das Wechseln des Servers soll innerhalb der App und noch vor dem Login in ein
Benutzerkonto möglich sein.

\functionalityOpt{Stummschalten}{mute}{70}
\fulfills{groups}%
Sowohl der Gruppenchat (\fncLink{chat}), als auch die
Abstimmungen (\fncLink{vote}) und Übertragungsanfragen
(\fncLink{positions}) lassen sich unabhängig voneinander
\enquote{stumm schalten}, d.h. es lassen sich jegliche Benachrichtungen für den
Nutzer unterdrücken.

\functionalityOpt{Zeitplan für Positionsübertragung}{timetable}{80}
\fulfills{timetable}%
Der Nutzer kann einzelnen Gruppen einen persönlichen wöchentlichen Zeitplan
zuteilen, in dem er festlegt, zu welchen Zeiten er automatisch seine Position
in der Gruppe freigibt.

\functionalityOpt{Passwortzurücksetzung}{passwordReset}{90}
\fulfills{passwordReset}%
Hat ein Nutzer sein Passwort vergessen, kann er eine E-Mail an seine zuvor
mit dem Konto verbundene E-Mail-Adresse senden lassen.
In dieser E-Mail sind Zugangsdaten, mit denen er sich temporär auch ohne das
Passwort authentifizieren und das Passwort ändern kann.
Dabei wird das vorherige Passwort nicht preisgegeben.

\functionalityOpt{Sprachen}{languages}{100}
Es werden als Sprachen Deutsch und Englisch angeboten.
Die Menge an Sprachen kann mit weiteren, späteren Versionen noch erweitert
werden.
Die Sprache der App wird automatisch der Systemsprache angepasst, sofern diese
unterstützt wird.

%%%%%%%%%%
\pagebreak
\section{Nicht-Funktionale Anforderungen}

\nonFunctionality{Standortteilung für zwei einzelne Personen}{1to1}{10}
\fulfills{1to1}%
Es soll für zwei Einzelpersonen noch einfacher sein, ihren Standort miteinander
zu teilen. Dies soll in weniger Schritten möglich sein, als manuell eine
Gruppe aus zwei Leuten zu erstellen.

\nonFunctionality{Einfache Bedienung}{handling}{20}
Damit die App sich gegen die gängigen Instant Messenger durchsetzen kann, muss
sie intuitiv und einfach bedienbar sein. Um dies zu erreichen, erfüllt die
graphische Benutzeroberfläche die \enquote{Material Design}-Richtlinien.

\nonFunctionality{Ressourcennutzung}{resourceUsage}{30}
Um Akkulaufzeit und mobiles Datenvolumen zu sparen, findet die Aktualisierung
von Positionsdaten maximal einmal pro Sekunde statt.

\nonFunctionality{Mobile Nutzung}{connectionLoss}{40}
Verbindungsabbrüche, wie sie im mobilen Netz öfter auftreten, führen nicht zu
Fehlfunktionen.

\nonFunctionality{Nutzeraufkommen}{userLoad}{50}
Es müssen mindestens 1000 Nutzer die Anwendung in Verbindung mit dem gleichen
Server gleichzeitig verwenden können.

\nonFunctionality{Datenschutz}{privacy}{60}
\fulfills{privacy}%
Der Umgang mit Nutzerdaten entpricht den national geltenden
Datenschutzgesetzen.

\nonFunctionality{Datenspeicherung}{data storage}{70}
Daten über Standorte werden serverseitig nur so lange gespeichert, wie diese
für andere Nutzer einsehbar sein sollen.

%%%%%%%%%%%
\pagebreak
\section{Tests}
\subsection{Funktionale Anforderungen}
\subsection{Muss}

\test{Account erstellen}{firstlogin}{10}
\tests{account}
\testStep{Alice hat sich die Go-App soeben herunter geladen.}
{Sie öffnet die App zum ersten Mal.}
{Ihr wird ein Formular angezeigt, in dem sie einen gewünschten Nutzernamen
und ein Passwort eingeben kann.}

\testStep{Alice hat \enquote{alice123} als Nutzernamen und ein geheimes Passwort eingegeben.}
{Sie bestätigt ihre Angaben.}
{Alice befindet sich nun im Hauptmenü der App, von wo aus sie Zugriff auf alle Funktionen hat.}

\test{Gruppe Erstellen}{grpcreate}{20}
\tests{groupfounding}

\testStep{Nutzer Bob ist eingeloggt.}
{Bob tippt auf den Tab  \enquote{Gruppen}.}
{Er bekommt die Liste aller Gruppen angezeigt in denen er Mitglied ist.}

\testStep{Bob möchte eine Gruppe mit Leuten aus dem Fußball-Training erstellen.}
{Er tippt auf den \enquote{Neue Gruppe erstellen}-Button.}
{Er bekommt eine Liste all seiner Kontakte, die er in der App hinzugefügt hat, angezeigt.}

\testStep{Seine Freunde Charlie und Dave spielen ebenfalls Fußball.}
{Er tippt die beiden Namen an.}
{Die App schlägt ihm vor die Gruppe zu erstellen.}

\testStep{Bob hat keine weiteren Freunde.}
{Er tippt auf Gruppe erstellen}
{Das Übersichtsmenü der Gruppe wird angezeigt.}

\test{Karte anzeigen}{map}{30}
\tests{map}
\testStep{Alice ist eingeloggt und befindet sich auf der Startseite.}
{Sie tippt auf die Schaltfläche für die Umgebungskarte.}
{Auf einer Landkarte wird ihr sowohl die eigene Position, als auch die von ihren
eventuellen Verabredungen in ihrer Nähe angezeigt.
Falls es andere Mitglieder in ihren Gruppen gibt, die momentan
ihre Position teilen ist diese ebenfalls auf der Karte sichtbar.}


\test{Verabredung erstellen}{event}{40}
\tests{meetingpoints}
\testStep{Bob möchte sich mit seinen Freunden zum Mittagessen treffen
und hat die App geöffnet.}
{Er wählt die Gruppe aus, in der viele seiner Freunde und Kommilitonen Mitglied sind.}
{Er bekommt alle anstehenden Verabredungen dieser Gruppe angezeigt.}

\testStep{Bob sieht, dass noch keine Verabredung zum Mittagessen vorhanden ist.}
{Er tippt auf die Schaltfläche \enquote{Neue Verabredung}.}
{Die App zeigt ihm ein Formular, in dem er Name, Zeit und Ort der Verabredung eingeben kann.}

\testStep{Bob hat als Namen \enquote{Mittagessen}, 12:50 Uhr als Uhrzeit angegeben und auf
einer Karte die Mensa als Treffpunkt markiert.}
{Er bestätigt seine Angaben mit \enquote{Verabredung erstellen}.}
{Ein Dialog um anzugeben, wann die anderen Mitglieder benachrichtigt werden sollen, erscheint.}

\testStep{Bob entscheidet sich für \enquote{20 min vor Verabredung}.}
{Wieder bestätigt er seine Angaben.}
{Auf einer Umgebungskarte wird nun die eben erstellte Verabredung markiert.
Alle anderen Gruppenmitglieder erhalten eine Benachrichtigung,
dass eine neue Verabredung hinzugefügt wurde.}

\testStep{Es ist nun 12:30 Uhr, eine weitere Benachrichtigung wurde an alle Gruppenmitglieder
gesendet mit einer Aufforderung jetzt ihren Standort zu teilen.}
{Bob und seine Freunde Ted, Victor und Grace stimmen zu, dass ihr Standort geteilt wird.}
{Auf der Karte ist nun neben der Markierung für die Verabredung auch ein Marker für
jeden der Freunde zu sehen.}


\test{Terminplanung}{eventlist}{50}
\tests{eventlist}
\testStep{Oscar ist in der App eingeloggt und möchte sehen, wie sein Terminplan für die nächsten Tage aussieht.}
{Er navigiert von der Startseite auf die Verabredungsliste.}
{Die App zeigt ihm eine Liste aller zukünftigen Verabredungen.
Unter Anderem zum Beispiel ein Treffen mit seiner Lerngruppe heute um 14 Uhr
und ein gemeinsames Abendessen mit seinen Freunden aus dem Training morgen um 19 Uhr.}


\test{Standortanfrage}{request}{60}
\tests{request}
\testStep{Bob ist in der App eingeloggt und möchte eine Standortanfrage an die Mitglieder seiner Lerngruppe senden.}
{Er navigiert von der Startseite auf die Gruppenliste, wählt seine Lerngruppe aus und wählt aus, eine Standortanfrage zu senden.}
{Die App zeigt nun ein Formular, auf dem Bob den Zeitraum der Anfrage spezifizieren kann.}

\testStep{Bob möchte die Standorte innerhalb der nächsten Stunde einsehen}
{Er gibt den Zeitraum von einer Stunde ein und startet die Anfrage.}
{Die App sendet eine Standortanfrage an alle Mitglieder der Lerngruppe.}

\testStep{Alice hat die Anfrage erhalten und möchte ihren Standort freigeben.}
{Sie wählt innerhalb der App aus, die Anfrage zu akzeptieren.}
{Der Standort von Alice wird nun auf der Karte angezeigt.}

\testStep{Ned hat die Anfrage erhalten und möchte seinen Standort nicht freigeben.}
{Er wählt innerhalb der App aus, die Anfrage abzulehnen.}
{Der Standort von Ned bleibt auf der Karte verborgen.}

\testStep{Eine Stunde ist vergangen.}
{Keine Eingaben werden getätigt.}
{Der Standort von Alice ist auf der Karte nicht mehr einsehbar}


\test{Kontoverwaltung}{account}{70}
\tests{account}
\testStep{Craig ist um die Sicherheit seiner Daten im Internet besorgt.
	Er möchte deshalb sein Passwort regelmäßig ändern.}
{Craig navigiert zu den Kontoeinstellungen und wählt \enquote{Passwort ändern}.}
{Es wird ein Formular in das Craig sein altes und sein gewünschtes neues Passwort eintragen kann.}

\testStep{Craig hat alle nötigen Angaben gemacht.}
{Er bestätigt seine Angaben.}
{Craig erhält eine E-Mail als Bestätigung, dass sein Passwort geändert wurde.}
Anmerkung: Mit dem gleichen Ablauf kann der Nutzer auch seinen Nutzernamen oder seine
E-Mail-Adresse ändern oder seinen Account löschen.


\subsection{Kür-Anforderungen}
\test{Abstimmung}{poll}{80}
\tests{vote}
\testStep{Bob möchte mit seinen Freunden Abends in eine Bar gehen.
Alle diese Freunde sind bereits Mitglied in der Gruppe \enquote{Lerngruppe}.}
{Bob navigiert auf die Gruppe und wählt \enquote{Neue Abstimmung}.}
{In einem Formular kann Bob einen Namen eingeben und Wahlmöglichkeiten hinzufügen,
	sowie eine Uhrzeit wählen.}

\testStep{Bob nennt die Abstimmung \enquote{Barabend}, wählt als Uhrzeit 19 Uhr und 
fügt \enquote{Enchiladas} und \enquote{Ox} als Optionen hinzu. Er bestimmt eine Abstimmzeit von einer Stunde.}
{Bob bestätigt seine Angaben.}
{Eine neue Abstimmung wird erstellt und alle Gruppenmitglieder werden darüber benachrichtigt.
Die möglichen Treffpunkte werden bei allen Mitgliedern auf der Umgebungskarte angezeigt.}

\testStep{Bob hat bereits für Ox abgestimmt, weil er dann nicht so weit laufen muss.}
{Alice wurde über die Abstimmung benachrichtigt, möchte aber heute nicht viel Geld ausgeben.
Sie fügt deshalb \enquote{Z10} als Option zu der Abstimmung hinzu.}
{Die Abstimmung wird um eine Option erweitert und 
wieder alle Mitglieder über die Veränderung benachrichtigt.}

\testStep{Alle Mitglieder werden über die Veränderung benachrichtigt und Einige stimmen ab.}
{Obwohl Bob zuvor schon eine Stimme für Ox abgegeben hat, wählt er jetzt zusätzlich Z10.}
{Nach einer Stunde ist die Abstimmung zu Ende und das Ergebnis steht fest.
Die Abstimmung wird automatisch gelöscht und stattdessen eine Verabredung erstellt mit der 
eingestellten Uhrzeit und dem Treffpunkt, welcher die meisten Stimmen bekommen hatte.}


\test{Gruppenchat}{chat}{90}
\tests{chat}
\testStep{Alice hat eine Verabredung für heute Abend erstellt weil sie sich mit ihren Freunden
treffen will. Ted hat allerdings keine Zeit und möchte Missverständnisse vermeiden, sodass
die Anderen nicht aus Versehen auf ihn warten.}
{Er öffnet die Gruppe und wählt die Chat Funktion aus. Er schreibt \enquote{Ich kann heute Abend leider nicht, braucht nicht Zu warten.} und schickt die Nachricht ab.}
{Die anderen Gruppenmitglieder werden benachrichtigt, dass sie eine neue Nachricht haben.}

\testStep{Alice hat Ted schon eine Weile nicht mehr gesehen und findet Teds Fehlen schade.}
{Sie navigiert ebenfalls auf den Chat und antwortet \enquote{Schade, hoffentlich dann beim
nächsten Mal}.}
{Wieder bekommen die Anderen eine Benachrichtigung.}


%%%%%%%%%%%%%
\pagebreak
\appendix

\section{Seitenentwürfe}

\guiFigure{mockups/login.png}
{login}
{10}
{Login-Screen beim ersten Öffnen der App.}

\guiFigure{mockups/home_groups.png}
{home-groups}
{20}
{Hier können Gruppen verwaltet und angesehen, sowie neue hinzugefügt werden.}

\guiFigure{mockups/group_map.png}
{group-map}
{30}
{Auf der Karte können die Positionen der anderen Gruppenmitglieder und
bevorstehende Treffpunkte eingesehen werden. Nutzer erscheinen ggf. gruppiert.}

\guiFigure{mockups/group_events.png}
{group-events}
{40}
{Innerhalb der Gruppe können Treffpunkte eingesehen und hinzugefügt werden.}

\guiFigure{mockups/group_members.png}
{group-members}
{50}
{Gruppenmitglieder können eingesehen, entfernt und hinzugefügt werden.  Es gibt
gruppenspezifische Einstellungsmöglichkeiten, sowie die Option, die Gruppe zu
verlassen.}

\guiFigure{mockups/home_groups_sidenav.png}
{home-groups-sidenav}
{60}
{Vom Startbildschirm aus können diverse Einstellungen erreicht werden.}

\section{Glossar}

\textbf{Android}:
Ein von der Open Handset Alliance entwickeltes Betriebssystem für Mobilgeräte.

\textbf{Apache Tomcat}:
Ein Open-Source-Webserver, der es erlaubt, in Java geschriebene Web-Anwendungen
auszuführen.

\textbf{GPS}:
Global Positioning System, ein globales Satellitensystem zur
Positionsbestimmung.

\textbf{Material Design}:
Von Google entwickelte Richtlinien für die Entwicklung von graphischen
Benutzeroberflächen.

\textbf{Verabredung}:
Ein festgelegter Ort und Zeitpunkt.

\end{document}
