\documentclass[parskip=full,11pt]{scrartcl}
\usepackage[utf8]{inputenc}
\usepackage[T1]{fontenc}
\usepackage[german]{babel}
\usepackage[useregional]{datetime2}
\usepackage[pdfborderstyle={/S/U/W 0}]{hyperref}
\usepackage[nameinlink]{cleveref}
\usepackage[section]{placeins}
\usepackage{xcolor}
\usepackage{graphicx}
\usepackage{csquotes}
\usepackage{amsmath} % for $\text{}$
\usepackage{pflichtenheft}

\newcommand\urlpart[2]{$\underbrace{\text{\texttt{#1}}}{\text{#2}}$}

\crefname{figure}{Abb}{Abb}

\hypersetup{
	pdftitle={Pflichtenheft},
	bookmarks=true,
}

% section numbers in margins:
\renewcommand\sectionlinesformat[4]{\makebox[0pt][r]{#3}#4}

% header & footer
\usepackage{scrlayer-scrpage}
\lofoot{\today}
\refoot{\today}
\pagestyle{scrheadings}

\newcommand\producttitle{GO-App}

\title{Pflichtenheft: \producttitle}
\author{Lukas Dippon
        \and Jens Kienle
        \and Matthias Noll
        \and Fabian Röpke
        \and Tim Schmidt
        \and Simon Vögele}

\begin{document}
\maketitle

\section{Einleitung}
Studenten und Mitarbeiter des KIT treffen sich gerne zum gemeinsamen Essen oder Lernen.
Dazu ist an einem bestimmten Tag immer notwendig zu wissen, ob bereits ein Treffen vereinbart wurde.
Am Treffpunkt angekommen, möchte man wissen, an welchem Ort sich die Gruppe befindet, um nicht lange suchen zu müssen.
Unsere App soll es angemeldeten Benutzern ermöglichen, sich in Gruppen zu organisieren.
In den Gruppen kann der Zeit- und Treffpunkt bestimmt werden.
Nach der Festlegung des Treffens soll die App die GPS-Standorte der Mitglieder temporär und anonym anzeigen, um das Treffen zu vereinfachen.

\pagebreak
\section{Kriterien}
% Diese Section sollte kurz und knapp "für Manager" sein
% und auf eine Seite passen.

\subsection{Muss}
\criterium{Vereinfachte Treffpunkte}{crt:easy}{10}
Durch das Freigeben von Positionsdaten können andere Nutzer den
Standort bestimmen.

\criterium{Datenschutz}{crt:safe}{20}
Die Positionsdaten werden nur mit Bestätigung
des Nutzers gesendet und serverseitig nicht längerfristig gespeichert.

\criterium{Gruppen}{crt:groups}{30}
Mehrere Nutzer können sich in Gruppen zusammenschließen. In diesen
können sie Positionsdaten austauschen und Treffpunkte festlegen.
Ein Nutzer kann in beliebig vielen Gruppen Mitglied sein.

\criterium{1 zu 1 Verbindung}{crt:1to1}{40}
Die Position kann auf Wunsch auch mit nur mit einem anderen Nutzer geteilt werden

\criterium{Account}{crt:acc}{50}
Ein persönlicher Account ermöglicht es allen Nutzern,
Gruppenzugehörigkeiten und Datenschutzoptionen zu speichern.

\subsection{Kann}
\criteriumOptional{Abstimmungen}{crt:vote}{10}
Innerhalb von Gruppen können die Nutzer z.B. über nächste Treffpunkte abstimmen

\criteriumOptional{Detailliertere Innenraumkarten}{crt:indoor}{20}
Innenräume mit hoher Nutzerdichte bekommen detaillierte Innenraumkarten.

\criteriumOptional{Seite mit Betreiberinfo}{crt:about}{30}
Der Dienst bietet eine Seite \enquote{Über Uns},
mit Informationen zum Betreiber.

\subsection{Abgrenzung}
\criteriumNot{Keine Wahl Kurz-URL}{crt:no-choice}{10}
Ein Nutzer hat keine Möglichkeit die Auswahl einer Kurz-URL zu beeinflussen.

\pagebreak
%%%%%%%%%%%%%%
\section{Produkteinsatz}
Das Produkt dient zur Erleichterung des Zusammenfindens und der
Treffpunktvereinbarung in geschlossenen Gruppen von Freunden, Kollegen oder
Familienmitgliedern.

\subsection{Anwendungsbereiche}
\begin{itemize}
    \item Private Treffen in der Freizeit oder während der Arbeit
    \item Zusammenfindung auf größeren Veranstaltungen
\end{itemize}

\subsection{Zielgruppen}
\begin{itemize}
    \item Freundesgruppen
    \item Arbeitskollegen
\end{itemize}

\subsection{Betriebsbedingungen}
\begin{itemize}
    % TODO: Unschön formuliert
    \item Mobiler Einsatz in Umgebungen mit GPS-Empfang
\end{itemize}

%%%%%%%%%%%%%%
\section{Produktumgebung}
Das Produkt beinhaltet ein Client-Server-Modell.
Der Server dient lediglich als Vermittler zwischen Klienten.
Die Nutzer verwenden den Klienten auf ihrem mobilen Endgerät.
\subsection{Software}
\begin{itemize}
    \item Client: Android 4.1 (\enquote{Jelly Bean}, API-Level 16) oder
        höher.
    \item Server: Apache Tomcat 8
\end{itemize}

\subsection{Hardware}
% TODO: Mindestanforderungen an Hardware-Performance
\begin{itemize}
    \item Client: Android-fähiges mobiles Endgerät mit
        \begin{itemize}
            \item GPS-Empfänger
            \item Netzwerkkarte mit WLAN-Modul oder Mobilfunkeinheit
        \end{itemize}
    \item Server: Beliebiger Servercomputer, der Apache Tomcat 8 unterstützt
        und eine Internetanbindung besitzt
\end{itemize}

%%%%%%%%%%%
\section{Funktionale Anforderungen}

\functionality{Homepage der App unterteilt in Tabs für die unterschiedlichen Funktionen}{fnc:ui}{10}
\begin{itemize}
    \item Liste aller Gruppen, in denen man Mitglied ist (\functionalitylink{fnc:groupfounding})
    \item Liste aller Events, an welchen man teilnimmt (\functionalitylink{fnc:events})
    \item Einstellungen(\functionalitylink{fnc:options})
\end{itemize}

\functionality{Login in einen appspezifischen Account}{fnc:login}{20}
\fulfills{crt:acc}
Der Benutzer muss sich vor der Nutzung der App einen Account erstellen und sich mit diesem einloggen.

\functionality{Verwaltung von Einstellungen und Account- oder appbezogene Daten}{fnc:options}{30}
\fulfills{crt:acc}
Platzhalter % TODO: Beschreibung

\functionality{Gründen von Gruppen}{fnc:groupfounding}{40}
\fulfills{crt:groups}
Über einen Reiter im Hauptmenü können Gruppen erstellt werden.
Ihre Funktionen werden in \functionalitylink{fnc:groupmenu} bis \functionalitylink{fnc:locations} beschrieben.

\functionality{Gruppenmenü}{fnc:groupmenu}{50}
\fulfills{crt:groups}
Jede Gruppe besitzt ein eigenes Menü über welches ihre Funktionen angesteuert werden können:
\begin{itemize}
    \item Umgebungskarte (\functionalitylink{fnc:map})
    \item Anfrage von GPS-Standorten (\functionalitylink{fnc:locationrequest})
    \item Erstellen eines Events (\functionalitylink{fnc:events} und \functionalitylink{fnc:vote})
    \item Einladen von Mitgliedern (\functionalitylink{fnc:invitations})
    \item Liste aller Mitglieder (\functionalitylink{fnc:memberlist})
\end{itemize}

\functionality{Einladen von Mitgliedern}{fnc:invitations}{60}
\fulfills{crt:groups}
Es können Einladungen an Nutzer versandt.
Diese können von den eingeladenen Nutzern akzeptiert oder abgelehnt werden.

\functionality{Liste der Gruppenmitglieder}{fnc:memberlist}{70}
\fulfills{crt:groups}
Es werden alle Mitglieder aufgelistet.
Durch das Anklicken eines Mitgliedes kann wahlweise das zugehörige Profil besucht oder das Mitglied aus der Gruppe entfernt werden.

\functionality{Umgebungskarte}{fnc:map}{80}
\fulfills{crt:groups}
\fulfills{crt:easy}
Jede Gruppe verfügt über eine Umgebungskarte mit eigenem Standort.

\functionality{Anfrage von GPS-Standorten}{fnc:locationrequest}{90}
\fulfills{crt:groups}
\fulfills{crt:easy}
Erstellen einer GPS-Anfrage, bei der alle Gruppenmitglieder über ein Pop-Up gebeten werden ihren Standort für einen fest definierten Zeitraum freizugeben

\functionality{Anzeigen von anonymen GPS-Standorten}{fnc:locations}{100}
\fulfills{crt:easy}
\fulfills{crt:safe}
Alle GPS-Standorte der Grupenmitglieder, die aktuell freigestellt sind, werden erfasst und als Gruppierungen auf der in \functionalitylink{fnc:map} beschriebenen Gruppenkarte markiert.
Diese Gruppierungen bestehen aus geographisch nah beieinanderliegenden Gruppenmitgliedern.
Einzelpersonen werden dabei unterscheidbar von Gruppierungen symbolisiert.

\functionality{Events}{fnc:events}{110}
Gruppenmitglieder können ein Event, bestehend aus Ort und einen Zeit, definieren.
Das Event wird mit Uhrzeit in der Eventliste(\functionalitylink{fnc:eventlist}) aufgelistet und mit Uhrzeit auf der Karte (\functionalitylink{fnc:map}) markiert.
Die Mitglieder werden eine kurze Zeit vor dem Event automatisch von der App über ein Pop-Up gebeten, ihren Standort temporär freizugeben.

\functionality{Abstimmungen}{fnc:vote}{120}
\fulfills{crt:vote}
Das Event(\functionalitylink{fnc:events}) lässt sich durch eine Abstimmung vereinbaren.
Diese ist zeitlich begrenzt durch eine vom Ersteller festgelegte Zeit.
Hierbei können inerhalb der Zeitgrenze beliebig oft von den Gruppenmitgliedern Orte und Zeitpunkte hinzugefügt werden, die eigene Stimme verändert, oder die Zeitgrenze verlängert werden.

\functionality{Anzeigen von Treffpunkten}{fnc:meetingpoints}{130}
\fulfills{crt:vote}
\fulfills{crt:easy}
Während einer Abstimmung werden alle möglichen Orte auf der Karte (\functionalitylink{fnc:map}) als solche angezeigt.
Durch das Anklicken des Ortes in der Abstimmung lassen sich die Markierungen auf der Karte erreichen.
Nach Vollendung der Abstimmung wird der Ort mit den meisten Stimmen als finaler Treffpunkt festgelegt, während alle anderen Orte von der Karte gelöscht werden.

\functionality{Standortteilung für einzelne Personen}{fnc:1to1}{140}
\fulfills{crt:1to1}
\fulfills{crt:easy}
Es soll auch für Einzelpersonen möglich sein ihren Standort zu teilen.
Dies soll intern als Zweiergruppe implementiert sein, aber in der App als extra Reiter schnell und ohne den Umweg über die Gruppengründung möglich sein.

\functionality{Eventliste}{fnc:eventlist}{150}
Alle vereinbarten Treffen werden nach Zeitpunkt sortiert angezeigt.
Das Auswählen eines Treffens zeigt die aktuelle Umgebungskarte(\functionalitylink{fnc:map}) der entsprechenden Gruppe an.

%%%%%%%%%%%
\section{Nicht-Funktionale Anforderungen}

\nonFunctionality{Einfache Bedienung}{nfc:handling}{10}
Damit die App sich gegen die gängigen Instant Messenger durchsetzen kann, muss sie intuitiv und einfach bedienbar sein.

\nonFunctionality{Ressourcennutzung}{nfc:persistence}{20}
Um Akkulaufzeit und mobiles Datenvolumen zu Sparen, findet die Kommunikation mit dem Server nur nach festen Zeitintervallen statt.

\nonFunctionality{Mobile Nutzung}{nfc:mobile usage}{30}
Die Anwendung muss auch bei geringen Verbindungsgeschwindigkeiten, wie sie bei mobilem Netz auftreten können, einwandfrei funktionieren.

\nonFunctionality{Nutzeraufkommen}{nfc:extensibility}{40}
Es müssen mindestens x Nutzer die Anwendung gleichzeitig verwenden können.  %TODO: Zahl der Nutzer festlegen

\nonFunctionality{Schneller Start}{nfc:launch}{50}
Das Starten der Anwendung darf nicht länger als x dauern.   %TODO: Startzeit festlegen

\nonFunctionality{Verlässlichkeit der Anzeige}{nfc:reliability}{60}
Es sollen nur Standorte angezeigt werden, die eine Genauigkeit von 20m oder besser aufweisen.

\nonFunctionality{Datenschutz}{nfc:privacy}{70}
Der Umgang mit Nutzerdaten entpricht den national geltenden Datenschutzgesetzen.

\nonFunctionality{Datenspeicherung}{nfc:data storage}{80}
Daten über Standorte werden serverseitig nur so lange gespeichert, wie diese für andere Nutzer einsehbar sein sollen.

\nonFunctionality{Löschen von Konten}{nfc:deletion}{90}
Nutzerkonten und alle dazugehörigen Daten sollen vollständig gelöscht werden können.

\nonFunctionality{Unabhängigkeit}{nfc:independency}{100}
Die Registrierung soll ohne Zwang von zusätlichen Konten, wie z.B bei Google oder Facebook möglich sein.

\nonFunctionality{Kompatibilität}{nfc:compatibility}{110}
Die Anwendung muss alle Andriod Versionen, ab einschließlich Android Version 5, unterstützen.



%%%%%%%%%%%
\section{Tests}

\test{Gruppe Erstellen}{tst:grpcreate}{10}
\tests{fnc:login}

\teststep{Nutzer \enquote{Ned Stark} ist eingeloggt.}
{Ned tippt auf den Tab  \enquote{Gruppen}.}
{Er bekommt die Liste aller Gruppen angezeigt in denen er Mitglied ist.}

\teststep{Ned möchte eine Gruppe mit Leuten aus dem Fußball-Training erstellen.}
{Er tippt auf den \enquote{+}-Button (Neue Gruppe erstellen).}
{Er bekommt eine Liste all seiner Kontakte, die er im Produkt hinzugefügt hat, angezeigt.}

\teststep{Seine Freunde Robert B. und Edmund T. spielen ebenfalls Fußball.}
{Er tippt die beiden Namen an.}
{Das Produkt schlägt ihm vor die Gruppe zu erstellen.}

\teststep{Ned hat keine weiteren Freunde.}
{Er tippt auf Gruppe erstellen}
{Der Homescreen der Gruppe wird angezeigt.} % TODO: Homescreen?

%%%%%%%%%%%%%
\pagebreak
\appendix

\section{Seitenentwürfe}

% made via https://gomockingbird.com/projects/mnf0cwf/4gXVnC

\begin{figure}[hb]
	\fbox{\includegraphics[height=80mm]{screenshots/menu.png}}
	\caption{\label{fig:menu}
		Menü mit der Möglichkeit auf Gruppen, Kontakte und Einstellungen zuzugreifen.
		 \testlink{tst:grpcreate}.
	}
\end{figure}

\begin{figure}[hb]
		\fbox{\includegraphics[height=80mm]{screenshots/karte.png}}
		\caption{\label{fig:map}
			Auf der Karte kann die Position der anderen Gruppenmitglieder eingesehen werden.
			Orte mit mehreren Mitgliedern erscheinen größer.
			\testlink{tst:grpcreate}.
		}
\end{figure}

\begin{figure}[hb]
		\fbox{\includegraphics[height=80mm]{screenshots/gruppen.png}}
		\caption{\label{fig:groups}
			Hier können Gruppen verwaltet und angesehen, sowie neue hinzugefügt werden.
			\testlink{tst:grpcreate}.
		}
\end{figure}

\section{Glossar}

\textbf{Besucher}:
Eine Person, welche den Dienst nutzt.
Kann eingeloggt sein oder nicht.

\textbf{Dienst}:
Die Software im laufenden Betrieb. Software as a Service.

\textbf{Homepage}:
Seite, die beim Besuchen der Betreiberdomain \emph{ohne Pfad} angezeigt wird. Auch \enquote{Startseite}.

\textbf{Nutzer}:
Ein eingeloggter Besucher.

\end{document}
