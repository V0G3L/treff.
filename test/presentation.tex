\documentclass[aspectratio=1610]{beamer}
\usepackage[utf8]{inputenc}
\usepackage[T1]{fontenc}
\usepackage[german]{babel}
\usepackage[useregional]{datetime2}
\usepackage[nameinlink]{cleveref}
\usepackage[section]{placeins}
\usepackage{xcolor}
\usepackage{graphicx}
\usepackage{csquotes}
\usepackage{amsmath} % for $\text{}$
\usepackage{enumitem}
\usepackage{testbericht}
\usepackage{algorithm}
\usepackage{algorithmicx}
\usepackage{algpseudocode}
\usepackage{listings}
\usepackage{bera}
\usepackage{pdfpages}
\usepackage{colortbl}
\usepackage{chronosys}

\colorlet{punct}{red!60!black}
\definecolor{background}{HTML}{EEEEEE}
\definecolor{delim}{RGB}{20,105,176}
\colorlet{numb}{magenta!60!black}
\lstdefinelanguage{json}{
    basicstyle=\normalfont\ttfamily,
    numbers=left,
    numberstyle=\scriptsize,
    stepnumber=1,
    numbersep=8pt,
    showstringspaces=false,
    breaklines=true,
    frame=lines,
    backgroundcolor=\color{background},
    literate=
     *{0}{{{\color{numb}0}}}{1}
      {1}{{{\color{numb}1}}}{1}
      {2}{{{\color{numb}2}}}{1}
      {3}{{{\color{numb}3}}}{1}
      {4}{{{\color{numb}4}}}{1}
      {5}{{{\color{numb}5}}}{1}
      {6}{{{\color{numb}6}}}{1}
      {7}{{{\color{numb}7}}}{1}
      {8}{{{\color{numb}8}}}{1}
      {9}{{{\color{numb}9}}}{1}
      {:}{{{\color{punct}{:}}}}{1}
      {,}{{{\color{punct}{,}}}}{1}
      {\{}{{{\color{delim}{\{}}}}{1}
      {\}}{{{\color{delim}{\}}}}}{1}
      {[}{{{\color{delim}{[}}}}{1}
      {]}{{{\color{delim}{]}}}}{1},
}

\setlist{nosep}

\newcommand\urlpart[2]{$\underbrace{\text{\texttt{#1}}}{\text{#2}}$}
\raggedbottom
\crefname{figure}{Abb}{Abb}

\newcommand\producttitle{treff.}
\hypersetup{
	pdftitle={Qualitätssicherungsphase: \producttitle},
	bookmarks=true,
}

% header & footer
\usepackage{scrlayer-scrpage}
%\lofoot{\today}
%\refoot{\today}
\pagestyle{scrheadings}

\title{\includegraphics[width = 50mm]{images/logo_crop.png}}
\subtitle{\huge Qualitätssicherungsphase}
\author{Lukas Dippon
	\and Jens Kienle
	\and Matthias Noll
    \\Fabian Röpke
	\and Tim Schmidt
	\and Simon Vögele}

\begin{document}

	\begin{frame}[plain]
	\maketitle
	\end{frame}

%%%%%%%% Vergleich mit Pflichtenheft %%%%%%%%%%

	\begin{frame}[plain]
        \frametitle{\textbf{Vergleich mit Pflichtenheft} -- Anforderungen}

        \begin{itemize}
          \item[-] einige Kür-Kriterien nicht oder teilweise erfüllt
          \item[-] keine Push-Benachrichtigungen \\
                   $\Rightarrow$ keine Chatbenachrichtigungen,
                   Standortanfragen etc
          \item[-] Nutzerkonten können nicht gelöscht werden \\
                   (schwerwiegende Lockingprobleme)
          \item[-] Material Design mehr als Richtlinien, weniger als Gesetz
          \item[-] Leistung bei hohem Nutzeraufkommen nicht garantiert
        \end{itemize}

  \end{frame}

  \begin{frame}[plain]
        \frametitle{\textbf{Vergleich mit Pflichtenheft} -- Testszenarien}

        \begin{itemize}
          \item[-] kleine Abweichungen in den Reaktionen \\
                   (Entwurfsänderung oder obige Probleme)
          \item[-] teilweise ungetestete Szenarien \\
                   (fehlende Implementierung)
          \item[-] Email wird nicht wie spezifiziert verwendet
        \end{itemize}

  \end{frame}

%%%%%%%% Server %%%%%%%%%%

	\begin{frame}[plain]
        \frametitle{\textbf{Server} -- Verlauf der Zeilenabdeckung durch JUnit}
        \includegraphics[width = \columnwidth]{images/test-coverage-graph.png}
  \end{frame}

  \begin{frame}[plain]
        \frametitle{\textbf{Server} -- Probleme}

        \begin{itemize}
          \item[-] Nebenläufigkeit schwer testbar
          \item[-] Inkompetenz bei Verwendung des zur Verfügung
                   gestellten Servers
        \end{itemize}
  \end{frame}

  \begin{frame}[plain]
        \frametitle{\textbf{Server} -- Deadlocks}
        \only<-2>{
            \begin{figure}
                \begin{tabular}{ c | c | c }
                    \textbf{Test \#} & \textbf{Run \#} & \textbf{Result} \\
                    \hline
                    1 & 1 & \cellcolor{green!25}Works \\
                    1 & 2 & \cellcolor{green!25}Works \\
                    1 & 3 & \cellcolor{gray!25}Doesn't terminate \\
                    1 & 4 & \cellcolor{green!25}Works \\
                    1 & 5 & \cellcolor{gray!25}Doesn't terminate \\
                \end{tabular}
                \caption{Effektivität von Concurrency-Tests hängt vom Scheduler
                ab}
            \end{figure}
        }
        \only<2>{
            \centering{Kein Compiler-Fehler/-Warnung und nicht reproduzierbar
            im Debugger}\\
            \centering{$\implies$ Statische Analyse nötig}
        }
        \only<3>{
            \begin{figure}
                \footnotesize
                \startchronology[startyear=1,stopyear=50,dates=false,
                arrowcolor=red,width=0.9\textwidth]
                \chronoevent{1}{\code{call execute()}}
                \chronoevent[markdepth=30pt]{13}{\code{readLock(x)}}
                \chronoevent{25}{\code{Set s = \{u,v,w,x,y,z\}}}
                \chronoevent[markdepth=30pt]{40}{\code{writeLockAllInSet(s)}}
                \chronoperiode[color=cyan]{13}{22}{Crit. Section}
                \chronoperiode[color=red,dates=false]{40}{50}{Deadlock}
                \stopchronology
                \caption{Deadlocks auch bei $\vert Threads\vert = 1$}
            \end{figure}
        }
        \only<4>{
            \begin{figure}
                \footnotesize
                \startchronology[startyear=1,stopyear=50,dates=false,
                arrowcolor=orange,width=0.9\textwidth]
                \chronoevent{1}{\code{call execute()}}
                \chronoevent[markdepth=45pt]{13}{\code{readLock(x)}}
                \chronoevent[markdepth=15pt]{22}{\code{releaseReadLock(x)}}
                \chronoevent[markdepth=45pt]{30}{\code{Set s = new HashSet()}}
                \chronoevent[markdepth=15pt]{40}{%
                    \code{writeLockAllInSet(s)}}
                \chronoperiode[color=cyan]{13}{22}{Crit. Section}
                \chronoperiode[color=orange,dates=false]{40}{50}{Pot. Deadlock}
                \stopchronology
                \caption{Deadlocks durch falsche Locking-Order}
            \end{figure}
        }

    \end{frame}

    \begin{frame}[plain]
        \frametitle{\textbf{Server} -- Scheinbar zufällige Update-Listen}
        \only<1-3>{
            \begin{itemize}
                \item 1. 23 Updates produzieren
                \item 2. 23 mal alle Updates abrufen
            \end{itemize}
        }
        \vspace{20px}
        \begin{minipage}{0.45\textwidth}
            \only<2-3>{
                \begin{figure}
                    \begin{tabular}{ c | c }
                        \textbf{Query \#} & \textbf{Response} \\
                        \hline
                        1 & (1,2,3,\dots,23) \\
                        2 & () \\
                        3 & () \\
                        4 & () \\
                        5 & () \\
                        \dots & () \\
                    \end{tabular}
                    \caption{Erwartete Ausgabe}
                \end{figure}
            }
        \end{minipage}
        \begin{minipage}{0.45\textwidth}
            \only<3-3>{
                \begin{figure}
                    \begin{tabular}{ c | c }
                        \textbf{Query \#} & \textbf{Response} \\
                        \hline
                        1 & (1,4,9,13,15,18,21,23) \\
                        2 & (2,5,10,14,16,19,22) \\
                        3 & (3,6,11,17,20) \\
                        4 & (7) \\
                        5 & (8) \\
                        \dots & () \\
                    \end{tabular}
                    \caption{Tatsächliche Ausgabe}
                \end{figure}
            }
        \end{minipage}
        \only<4->{
            \begin{figure}
                \includegraphics[width=300px]{images/server-updates-compareto.png}
                \caption{Ungewollte Äquivalenz durch Zeitvergleich}
            \end{figure}
        }
        \only<5->{
            \begin{minipage}{.5\textwidth}
                \centering
                Problem:\\
                Millisekunden-Präzision nicht ausreichend
            \end{minipage}%
        }%
        \only<6->{%
                \begin{minipage}{.5\textwidth}
                \centering
                Lösung:\\
                Sortierung über IDs statt Timestamps
            \end{minipage}
        }
    \end{frame}

%%%%%%%% Client %%%%%%%%%%

	\begin{frame}[plain]
        \frametitle{\textbf{Klient} -- Herausforderungen}

        \begin{itemize}
          \item[-] Nutzeroberfläche schwer automatisch zu testen
          \item[-] Android Komponenten teilweise schwer zu mocken
          \item[-] Server ist für die meisten Funktionen notwendig
        \end{itemize}
  \end{frame}

\begin{frame}[plain]
        \frametitle{\textbf{Klient} -- Testmethoden: Unit Tests}

        \begin{itemize}
          \item[-] ViewModels
          \item[-] Entitäten
          \item[-] Repositories
          \item[-] Commands und Updates
          \item[-] RequestEncoder
        \end{itemize}
  \end{frame}

\begin{frame}[plain]
        \frametitle{\textbf{Klient} -- Testmethoden: Instrumentation Tests}

        \begin{itemize}
          \item[-] Komponenten, für die Mock Klassen nicht zweckmäßig sind
          \item[-] Werden direkt auf Smartphone ausgeführt
          \item[-] DAOs
        \end{itemize}
  \end{frame}

\begin{frame}[plain]
        \frametitle{\textbf{Klient} -- Testmethoden: Manuell auf Smartphone}

        \begin{itemize}
          \item[-] Simulierter Server
          \item[-] Gut zum Testen der Nutzeroberfläche
          \item[-] Größtenteils schon Integrationstests des gesamten Klienten
          \item[-] Sehr Zeitaufwändig bei mehreren Iterationen
        \end{itemize}
  \end{frame}

\begin{frame}[plain]
        \frametitle{\textbf{Klient} -- Gefundene Fehler (Auswahl)}

        \begin{itemize}
          \item[-] Flüchtigkeitsfehler
          \item[-] Stabilität bei Verbindungsabbrüchen
          \item[-] Kleinere Anzeigefehler (noch offen)
          \item[-] Updatestruktur des Klienten (noch offen)
        \end{itemize}
  \end{frame}

%%%%%%%% Workflow %%%%%%%%%%
    \begin{frame}[plain]
        \frametitle{\textbf{Arbeitsweise}}
            \textbf{Aufteilung der Gruppe}
            \begin{itemize}
                \setlength\itemsep{0.3em}
                \item[--] 3 Personen für den Klient
                    \begin{itemize}
                        \item[-] Fabian
                        \item[-] Lukas
                        \item[-] Matthias
                    \end{itemize}
                \item[--] 3 Personen für den Server
                    \begin{itemize}
                        \item[-] Jens
                        \item[-] Simon
                        \item[-] Tim
                    \end{itemize}
            \end{itemize}
    \end{frame}
    \begin{frame}[plain]
            \frametitle{\textbf{Arbeitsweise}}
            \begin{minipage}{0.5\textwidth}
        	    \includegraphics[width = \columnwidth - 30pt]
                {images/meet-im-voip.png}
            \end{minipage}%
            \begin{minipage}{0.5\textwidth}
                \textbf{Kommunikation}
                \begin{itemize}
                    \setlength\itemsep{0.3em}
                    \item[--] Regelmäßige Treffen innerhalb der Teilgruppen
                        über VoIP-Dienste
                    \item[--] Klärung von Fragen zur API zwischen den
                        Teilgruppen über VoIP und den Instant Messenger
                        Telegram
                \end{itemize}
            \end{minipage}
    \end{frame}


%%%%%%%% Reflection %%%%%%%%%%

	\begin{frame}[plain]
        \frametitle{\textbf{Selbstreflexion}}
        \begin{minipage}{0.5\textwidth}
            \textbf{Positive Erfahrungen}
            \begin{itemize}
                \item[--] TODO % TODO
            \end{itemize}
        \end{minipage}%
        \begin{minipage}{0.5\textwidth}
            \textbf{Mögl. Verbesserungen}
            \begin{itemize}
                \item[--] Mehr Leute im Client-Team
                \item[--] Projekt zu groß?
                    \begin{itemize}
                        \item[-] 30k LoC statt 10k LoC
                    \end{itemize}
                \item[--] Nebenläufigkeit bereits im Entwurf komplett
                    durchplanen
                \item[--] Fertige Frameworks für Interaktionen mit
                    SQL-Datenbank nutzen
                \item[--] Google Services für Login/Push/Chat etc
            \end{itemize}
        \end{minipage}
    \end{frame}

\end{document}
