\documentclass[parskip=full,11pt]{scrartcl}
\usepackage[utf8]{inputenc}
\usepackage[T1]{fontenc}
\usepackage[german]{babel}
\usepackage[useregional]{datetime2}
\usepackage[pdfborderstyle={/S/U/W 0}]{hyperref}
\usepackage[nameinlink]{cleveref}
\usepackage[section]{placeins}
\usepackage[top=2.5cm, bottom=2.5cm, left=4cm, right=3cm]{geometry}
\usepackage{xcolor}
\usepackage{graphicx}
\usepackage{csquotes}
\usepackage{amsmath} % for $\text{}$
\usepackage{testbericht} % local .sty file
\usepackage{enumitem}
\usepackage{algorithm}
\usepackage{algorithmicx}
\usepackage{algpseudocode}

\setlist{nosep}

\newcommand\urlpart[2]{$\underbrace{\text{\texttt{#1}}}{\text{#2}}$}
\raggedbottom
\crefname{figure}{Abb}{Abb}

\newcommand\producttitle{treff.}
\newcommand\protocolversion{0.3}
\hypersetup{
	pdftitle={\producttitle~- Testbericht},
	bookmarks=true,
}

% section numbers in margins:
\renewcommand\sectionlinesformat[4]{%
    \ifstr{#1}{subsubsection}{%
        \makebox[0pt][r]{#4}%
    }{%
        \makebox[0pt][r]{#3}#4%
    }%
}%

% header & footer
\usepackage{scrlayer-scrpage}
%\lofoot{\today}
%\refoot{\today}
\pagestyle{scrheadings}
%\let\raggedsection\centering

\title{\includegraphics[width = 80mm]{images/logo_crop.png}}
\subtitle{\huge Testbericht}
\author{Lukas Dippon
        \and Jens Kienle
        \and Matthias Noll
        \and Fabian Röpke
        \and Tim Schmidt
        \and Simon Vögele}

\begin{document}

\maketitle
\thispagestyle{empty} % removed page number from title

\pagebreak
\tableofcontents

%%%%%%%%%%%%%%%%%%%
\pagebreak
\section{Erfüllung der Kriterien}

\section{Testszenarien}

\subsection{Pflichtanforderungen}
	\subsubsection{funktionierende Testszenarien}
	\begin{itemize}
		\item T10 : Account erstellen
		\item T20 : Kontakt hinzufügen
		\item T40 : Karte anzeigen
		\item T60 : Übertragung der eigenen Position an die Gruppe
	\end{itemize}

	\subsubsection{problematische Testszenarien}
	\begin{itemize}
		\item T80 : Kontoverwaltung
			\begin{itemize}
			\item Beim Löschen eines Accounts werden nicht die nötigen Updates
			produziert, um diesen aus all seinen Gruppen, Verabredungen, Abstimmungen
			etc zu entfernen
			\item Beim Ändern des Passworts wird keine Bestätigungsmail versand, da
			dies keinen Mehrwert mit sich bringen würde
			\end{itemize}
	\end{itemize}

\subsection{Küranforderung}
	\subsubsection{funktionierende Testszenarien}
	\begin{itemize}
	\item placeholder
	\end{itemize}

	\subsubsection{ungetestete Testszenarien}
	\begin{itemize}
		\item T100 : Abstimmung
		\item T120 : Innenraumkarten
		\item T130 : Stummschalten
		\item T140 : Gruppenrollen und -rechte
		\item T160 : Eigene Server
		\item T170 : Serverwechsel
		\item T180 : Passwort zurücksetzen
		\item T190 : Zeitplan für Positionsübertragung
	\end{itemize}

\end{document}
