\documentclass[parskip=full,11pt]{scrartcl}
\usepackage[utf8]{inputenc}
\usepackage[T1]{fontenc}
\usepackage[german]{babel}
\usepackage[useregional]{datetime2}
\usepackage[pdfborderstyle={/S/U/W 0}]{hyperref}
\usepackage{amsmath} % for $\text{}$
\usepackage[nameinlink]{cleveref}
\usepackage[section]{placeins}
\usepackage[top=2.5cm, bottom=2.5cm, left=4cm, right=3cm]{geometry}
\usepackage{xcolor}
\usepackage{graphicx}
\usepackage{csquotes}
\usepackage{testbericht} % local .sty file
\usepackage{enumitem}
\usepackage{algorithm}
\usepackage{algorithmicx}
\usepackage{algpseudocode}

\setlist{nosep}

\newcommand\urlpart[2]{$\underbrace{\text{\texttt{#1}}}{\text{#2}}$}
\raggedbottom
\crefname{figure}{Abb}{Abb}

\newcommand\producttitle{treff.}
\newcommand\protocolversion{0.3}
\hypersetup{
	pdftitle={\producttitle~- Testbericht},
	bookmarks=true,
}

% section numbers in margins:
\renewcommand\sectionlinesformat[4]{%
    \ifstr{#1}{subsubsection}{%
        #4%
    }{%
        \makebox[0pt][r]{#3}#4%
    }%
}%

% header & footer
\usepackage{scrlayer-scrpage}
%\lofoot{\today}
%\refoot{\today}
\pagestyle{scrheadings}
%\let\raggedsection\centering

\title{\includegraphics[width = 80mm]{images/logo_crop.png}}
\subtitle{\huge Testbericht}
\author{Lukas Dippon
        \and Jens Kienle
        \and Matthias Noll
        \and Fabian Röpke
        \and Tim Schmidt
        \and Simon Vögele}

\begin{document}

\maketitle
\thispagestyle{empty} % removed page number from title

\pagebreak
\tableofcontents

%%%%%%%%%%%%%%%%%%%
\pagebreak
\section{Nicht Erfüllte Anforderungen}
	\subsection{Funktionale Anforderungen}
		\subsubsection{Plicht}
			\begin{itemize}
                \item \textbf{Erstellung von Verabredungen innerhalb einer Gruppe:}\\
											\label{notification}Gruppenmitglieder werden nicht
                      automatisch gebeten, ihren Standort freizugeben.
                      Die Verwendung von Push-Benachrichtigungen stellte
                      sich ohne Verwendung von Google-Services als
                      äußerst schwierig und zeitaufwendig heraus. Da jede
                      andere Form von Benachrichtigung nicht zweckmäßig
                      ist, wurde auf diese verzichtet.
                \item \textbf{Standortanfrage:}\\
											\label{requestPosition}Da diese Funktion ohne
											Push-Benachrichtigungen nicht zweckmäßig ist,
											wurde sie nicht implementiert
                \item \textbf{Verwaltung von Kontodaten:}\\
                      \label{deleteaccount}Das Löschen von Accounts wurde
                      implementiert, aber während der Qualitätssicherung
                      ist aufgefallen, dass grobe Fehler gemacht wurden.
                      Da sich das Beheben dieser Fehler als sehr komplex
                      herausgestellt hat, wurde aus Zeitgründen darauf
                      verzichtet.
			\end{itemize}

		\subsubsection{Kür}
			\begin{itemize}
                \item \textbf{Abstimmungen:}\\
							Nur auf Serverseite implementiert und nicht vollständig getestet.
                \item \textbf{Gruppenrollen und -rechte:}\\
							Nur auf Serverseite implementiert und getestet.
                \item \textbf{Detaillierte Innenraumkarten:}\\
							nicht implementiert.
                \item \textbf{Stummschalten:}\\
							nicht implementiert
                \item \textbf{Zeitplan für Positionsübertragung:}\\
							nicht implementiert
                \item \textbf{Passwortzurücksetzung:}\\
							nicht implementiert
			\end{itemize}

	\subsection{Nicht-Funktionale Anforderungen}
		\subsubsection{Plicht}
			\begin{itemize}
                \item \textbf{Einfache Bedienung:}\\
											Die Material Design Richtlinien wurden größtenteils
											beachtet. Jedoch enstanden kleine Abweichungen
											durch z.B. persönliche Präferenzen bei der Verwendung
											der Farben des KIT-Logos
                \item \textbf{Nutzeraufkommen:}\\
											Aus zeitlichen Gründen konnte das Verhalten und die
											Leistung bei hohem Nutzeraufkommen noch nicht getestet
											und daher nicht garantiert werden.
			\end{itemize}

\section{Testszenarien}

	\subsection{Pflichtanforderungen}
		\subsubsection{funktionierende Testszenarien}
			\begin{itemize}
				\item \textbf{T10 : Account erstellen}
				\item \textbf{T30 : Gruppe erstellen}
                \item \textbf{T40 : Karte anzeigen}
				\item \textbf{T60 : Übertragung der eigenen Position an die Gruppe}
				\item \textbf{T90 : Terminplanung}
			\end{itemize}

		\subsubsection{Nicht vollständig funktionierende Testszenarien}
			\begin{itemize}
				\item \textbf{T20 : Kontakt hinzufügen}
					\begin{itemize}
						\item Angepasste Reaktion zur Entwurfsänderung "Kontaktanfragen"
						\item Eine gegenseitige Kontaktanfrage zweier Nutzer führt
									vorübergehend zu einem Anzeigefehler im Klienten
					\end{itemize}
				\item \textbf{T50 : Verabredung erstellen}
					\begin{itemize}
						\item Der Dialog zur Benachrichtigung der anderen
                        Mitglieder erscheint nicht (vgl. \ref{notification})
					\end{itemize}
				\item \textbf{T70 : Standortanfrage}
					\begin{itemize}
						\item nicht implementiert (vgl. \ref{requestPosition})
					\end{itemize}
				\item \textbf{T80 : Kontoverwaltung}
					\begin{itemize}
            \item Das Löschen des Accounts ist nicht möglich
									(vgl. \ref{deleteaccount})
          	\item Das Implementieren des Emailversands stellte sich
                  zum Ende der Implementierungsphase als
                  komplizierter heraus, als gedacht. Daher wurde aus
                  Zeitgründen darauf verzichtet.
					\end{itemize}
				\end{itemize}

	\subsection{Küranforderung}
		\subsubsection{Funktionierende Testszenarien}
			\begin{itemize}
				\item \textbf{T110 : Chat}\\
				 			Jedoch wird der Nutzer nicht über neue Nachrichten benachrichtigt.
							Dies ist auf die fehlenden Push-Benachrichtigungen zurückzuführen.
							(vgl. \ref{notification})
				\item \textbf{T150 : Sprache}
				\item \textbf{T160 : Eigene Server}
				\item \textbf{T170 : Serverwechsel}
	    \end{itemize}

	\subsubsection{Ungetestete Testszenarien}
		\begin{itemize}
    	    \item \textbf{T100: Abstimmungen}
                \begin{itemize}
                	\item serverseitig implementiert, aber aus Zeitgründen nicht
                    		vollständig getestet
                	\item klientseitig nicht implementiert
                \end{itemize}
			\item \textbf{T120: Innenraumkarten}
                \begin{itemize}
                	\item weder server- noch klientseitig implementiert
                \end{itemize}
			\item \textbf{T130: Stummschalten}
                \begin{itemize}
                	\item weder server- noch klientseitig implementiert
                \end{itemize}
			\item \textbf{T140: Gruppenrollen und -rechte}
                \begin{itemize}
                \item serverseitig implementiert und getestet
                	\item klientseitig nicht implementiert
                \end{itemize}
			\item \textbf{T180: Passwort zurücksetzen}
                \begin{itemize}
                	\item weder server- noch klientseitig implementiert
                \end{itemize}
			\item \textbf{T190: Zeitplan für Positionsübertragung}
                \begin{itemize}
                	\item weder server- noch klientseitig implementiert
                \end{itemize}
		\end{itemize}

\section{Art der Tests}
	\subsection{Server}
		\begin{itemize}
			\item Durch Unit Tests sind 82\% der Zeilen abgedeckt.
						Jedoch ist ein großer Teil der ungetestet Zeilen Klassen
						diverse Küranforderung zuzuordnen, die nur serverseitig
						implementiert wurden und daher aktuell keine Verwendung in
						der App finden.
			\item Zusätzlich wurden die Anwendungsfälle, die im Pflichtenheft
						definiert wurden, durchgespielt und somit die Nebenläufigkeit
						bei einer kleinen Anzahl von Nutzern getestet
		\end{itemize}

	\subsection{Klient}
		\begin{itemize}
			\item Durch Unit Tests sind ??\% der Zeilen abgedeckt. %TODO
			\item Da nicht alles sinnvoll durch Unit Tests getestet werden konnte,
						wurde dieser überwiegend durch das Durchspielen der
						Anwendungsfälle getestet
		\end{itemize}

\section{Probleme beim Testen}
	\begin{itemize}
		\item Die Nebenläufigkeit der Befehle ist schwer zu testen
		\item Der Klient ist schwer über Unit Tests zu testen
		\item Der Klient benötigt einen funktionierenden Server,
					um getestet zu werden
		\item Teilweise zeigt der Klient anderes Verhalten bei Verwendung einer
					anderen Androidversion
		\item Der zur Verfügung gestellte Server konnte nicht verwendet werden
	\end{itemize}
\end{document}
