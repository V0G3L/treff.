\documentclass[parskip=full,11pt]{scrartcl}
\usepackage[utf8]{inputenc}
\usepackage[T1]{fontenc}
\usepackage[german]{babel}
\usepackage[useregional]{datetime2}
\usepackage[pdfborderstyle={/S/U/W 0}]{hyperref}
\usepackage{amsmath} % for $\text{}$
\usepackage[nameinlink]{cleveref}
\usepackage[section]{placeins}
\usepackage[top=2.5cm, bottom=2.5cm, left=4cm, right=3cm]{geometry}
\usepackage{xcolor}
\usepackage{graphicx}
\usepackage{csquotes}
\usepackage{testbericht} % local .sty file
\usepackage{enumitem}
\usepackage{algorithm}
\usepackage{algorithmicx}
\usepackage{algpseudocode}

\setlist{nosep}

\newcommand\urlpart[2]{$\underbrace{\text{\texttt{#1}}}{\text{#2}}$}
\raggedbottom
\crefname{figure}{Abb}{Abb}

\newcommand\producttitle{treff.}
\newcommand\protocolversion{0.3}
\hypersetup{
	pdftitle={\producttitle~- Testbericht},
	bookmarks=true,
}

% section numbers in margins:
\renewcommand\sectionlinesformat[4]{%
    \ifstr{#1}{subsubsection}{%
        #4%
    }{%
        \makebox[0pt][r]{#3}#4%
    }%
}%

% header & footer
\usepackage{scrlayer-scrpage}
%\lofoot{\today}
%\refoot{\today}
\pagestyle{scrheadings}
%\let\raggedsection\centering

\title{\includegraphics[width = 80mm]{images/logo_crop.png}}
\subtitle{\huge Testbericht}
\author{Lukas Dippon
        \and Jens Kienle
        \and Matthias Noll
        \and Fabian Röpke
        \and Tim Schmidt
        \and Simon Vögele}

\begin{document}

\maketitle
\thispagestyle{empty} % removed page number from title

\pagebreak
\tableofcontents

%%%%%%%%%%%%%%%%%%%
\pagebreak
\section{Nicht Erfüllte Anforderungen}
\subsection{Funktionalle Anforderungen}
\subsubsection{Plicht}
TODO
\subsubsection{Kür}
\paragraph{\small Abstimmungen:} \hspace{0pt} \\
Nur auf Serverseite implementiert und nicht vollständig getestet.

\paragraph{\small Gruppenrollen und -rechte:} \hspace{0pt} \\
Nur auf Serverseite implementiert und getestet.

\paragraph{\small Innenraumkarten:} \hspace{0pt} \\
Wurde nicht implementiert.

\paragraph{\small Eigener server:} \hspace{0pt} \\
Die Möglichkeit, einen eigenen Server auf der definierten Schnittstelle zwischen
Server und Klient aufbauend zu entwickeln, ist gegeben. Allerdings ist der
Server im Klient fest verankert und man müsste sich den Klient mit minimalen
Änderungen selbst erstellen.

\paragraph{\small Serverwechsel:} \hspace{0pt} \\
Das Wechseln des Servers innerhalb der App ist nicht möglich.

\paragraph{\small Stummschalten:} \hspace{0pt} \\
Stummschalten ist nicht implementiert.

\paragraph{\small Zeitplan für Positionsübertragung:} \hspace{0pt} \\
Das Übermittlen der Position nach einem Zeitplan ist nicht möhlich.

\paragraph{\small Passwortzurücksetzung:} \hspace{0pt} \\
Das Zurücksetzen des Passworts ist nicht möglich

\paragraph{\small Sprachen:} \hspace{0pt} \\
Die App bietet eine deutsche und eine englische Übersetzung, diese passt sich
automatisch der Systemsprache an.

\subsection{Nicht Funktionalle Anforderungen}
\subsubsection{Plicht}
\paragraph{\small Einfache Bedienung:} \hspace{0pt} \\
TODO von jemand der beschreiben kann inwieweit wir Materiel Design einhalten

\paragraph{\small Nutzeraufkommen:} \hspace{0pt} \\
TODO

\section{Testszenarien}

\subsection{Pflichtanforderungen}
	\subsubsection{funktionierende Testszenarien}
	\begin{itemize}
		\item T10 : Account erstellen
		\item T20 : Kontakt hinzufügen
		\item T40 : Karte anzeigen
		\item T60 : Übertragung der eigenen Position an die Gruppe
	\end{itemize}

	\subsubsection{Problematische Testszenarien}
	\begin{itemize}
		\item T80 : Kontoverwaltung
			\begin{itemize}
			\item Beim Löschen eines Accounts werden nicht die nötigen Updates
			produziert, um diesen aus all seinen Gruppen, Verabredungen, Abstimmungen
			etc zu entfernen
			\item Beim Ändern des Passworts wird keine Bestätigungsmail versand, da
			dies keinen Mehrwert mit sich bringen würde
			\end{itemize}
	\end{itemize}

\subsection{Küranforderung}
	\subsubsection{Funktionierende Testszenarien}
	\begin{itemize}
	\item placeholder
	\end{itemize}

	\subsubsection{Ungetestete Testszenarien}
	\begin{itemize}
		\item T100 : Abstimmung
		\item T120 : Innenraumkarten
		\item T130 : Stummschalten
		\item T140 : Gruppenrollen und -rechte
		\item T160 : Eigene Server
		\item T170 : Serverwechsel
		\item T180 : Passwort zurücksetzen
		\item T190 : Zeitplan für Positionsübertragung
	\end{itemize}

\end{document}
