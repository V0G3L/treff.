\documentclass[parskip=full,11pt]{scrartcl}
\usepackage[utf8]{inputenc}
\usepackage[T1]{fontenc}
\usepackage[german]{babel}
\usepackage[useregional]{datetime2}
\usepackage[pdfborderstyle={/S/U/W 0}]{hyperref}
\usepackage[nameinlink]{cleveref}
\usepackage[section]{placeins}
\usepackage{xcolor}
\usepackage{graphicx}
\usepackage{csquotes}
\usepackage{amsmath} % for $\text{}$
\usepackage{entwurf}
\usepackage{enumitem}
\setlist{nosep}

\newcommand\urlpart[2]{$\underbrace{\text{\texttt{#1}}}{\text{#2}}$}
\raggedbottom
\crefname{figure}{Abb}{Abb}

\newcommand\producttitle{treff.}
\hypersetup{
	pdftitle={Entwurf: \producttitle},
	bookmarks=true,
}

% section numbers in margins:
\renewcommand\sectionlinesformat[4]{\makebox[0pt][r]{#3}#4}

% header & footer
\usepackage{scrlayer-scrpage}
%\lofoot{\today}
%\refoot{\today}
\pagestyle{scrheadings}

\title{\includegraphics[width = 80mm]{images/logo_crop.png}}
\subtitle{\huge Entwurf}
\author{Lukas Dippon
        \and Jens Kienle
        \and Matthias Noll
        \and Fabian Röpke
        \and Tim Schmidt
        \and Simon Vögele}

\begin{document}

\maketitle
\thispagestyle{empty} % removed page number from title

\pagebreak
\tableofcontents

%%%%%%%%%%%%%%%%%%%
\pagebreak
\section{Einleitung}

\subsection{Client-Server}
% auf client-server architektur eingehen
% (Fließtext [kleiner Einleitungstext] und ggf kleines UML-Klassendiagram)

\subsection{Funktion und externes Verhalten - Client}
% Funktion und Verhalten des Clients (Fließtext und UML-Diagramme)

\subsection{Funktion und externes Verhalten - Server}
% Funktion und Verhalten des Servers (Fließtext und UML-Diagramme)

%%%%%%%%%%%%%%%%%%%%
\pagebreak
\section{Aufbau und internes Verhalten - Client}

\subsection{Aufbau}
% Aufteilung in GUI und Datenverarbeitung (Fließtext)
% Ihre Funktionen

\subsection{Bestandteile der GUI}
% Strukturierung der GUI durch Activities und deren Aufteilung in Fragments,
%	sowie das Zusammenspiel der Activities (Fließtext und UML-Klassendiagramme,
% ggf Sequenzdiagramme)

\subsection{Bestandteile Datenverarbeitung}
% Aufteilung der Funktionalitäten auf seperate Klassen,
% sowie deren Zusammenspiel (Fließtext und UML-Klassendiagramme, ggf Sequenzdiagramme)

\subsection{Kommunikation von GUI und Datenverarbeitung}
% Zusammenspiel von GUI und Datenverarbeitung
% (Fließtext und UML-Klassendiagramme, ggf Sequenzdiagramme)

\subsection{JavaDoc}

%%%%%%%%%%%%%%%%%%%%%%
\pagebreak
\section{Aufbau und internes Verhalten - Server}

\subsection{Aufbau}
% Aufteilung der Funktionalitäten auf seperate Klassen;
% Zusammenspiel dieser Klassen (Fließtext und UML-Klassendiagramme,
% ggf Sequenzdiagramme)

\subsection{Bestandteile des Servers}
% Aufbau und genaue Funktionalität dieser Klassen
% [z.B. Datenbank, Schnittstelle zum Client]
% (Fließtext und UML-Klassendiagramme, ggf Sequenzdiagramme)

\subsection{JavaDoc}

%%%%%%%%%%%%%%%%%%%%%%
\pagebreak
\section{Kommunikation zwischen Client und Server}
% Zusammenspiel von Client und Server
% (Fließtext und Sequenzdiagram, ggf UML-Klassendiagram)
% TODO: JSON verlinken / footnote
Client und Server kommunizieren mittels des JSON-Dateiformates.
Die oberste Ebene jeder Nachricht bildet ein einzelnes JSON-Objekt.
Parameter, Rückgabewerte und Fehlercodes werden als Eigenschaft dieses Objektes
kodiert.
Eine Anfrage eines Klienten an den Server beinhaltet immer mindestens eine
Eigenschaft mit dem Schlüssel \textbf{cmd}.
Die Namen der im folgenden definierten Befehle sind exakt die Werte, die bei
Kodierung als Nachricht den Wert der \textbf{cmd}-Eigenschaft bilden müssen.
Diese Namen sind als simple Zeichenketten zu kodieren.
\\Wird während der Ausführung des Befehls ein Fehler gefunden, möglicherweise durch
übermittelte Parameter bedingt, wird nur ein Fehlercode mit dem Schlüssel
\textbf{error} zurückgegeben. Die Bedeutung der Fehlercodes wird ebenfalls im
folgenden definiert.

\subsection{Befehle}
Bis auf dem \textbf{login}-Befehl erfordert jeder Befehl folgenden
Parameter:\\
\apiArgument{token/Authentifizierungtoken{,} welches beim Einloggen in das
Benutzerkonto vom Server zurückgegeben wurde}
\par\par Ist das übermittelte Token ungültig, wird der Befehl nicht
ausgeführt und folgender Fehlercode zurückgegeben:\\
\apiErrorCode{010/Authentifizierungtoken ungültig}

\apiCommand{login}
{Einloggen in ein Nutzerkonto}
{user/Benutzername des Kontos,
pass/Passwort des Kontos}
{token/Authentifizierungstoken{,} welches in weiteren Befehlen
als Eingabe verlangt wird}
{100/Nutzername{/}Passwort-Kombination ungültig}

\apiCommand{list-groups}
{Auflisten der Gruppen, in denen das Benutzerkonto Mitglied ist}
{---/Keine weiteren Parameter}
{groups/Ein Array an oberflächlich beschriebenen Gruppen (siehe
Kapitel \ref{sec:groupdesc})}
{---/Keine weiteren Fehlercodes}

\apiCommand{get-group-details}
{Abrufen von Details einer Gruppe, in denen das Benutzerkonto Mitglied ist}
{id/Identifikationsnummer der Gruppe}
{\dots/Eine ausführliche Beschreibung der Gruppe{,} die zu der angegebenen
Identifikationsnummer gehört (siehe \ref{sec:groupdesc})}
{300/Gruppen-Identifikationsnummer ungültig oder Benutzerkonto nicht Teil der
Gruppe}

\subsection{Gruppenbeschreibung}\label{sec:groupdesc}
Gruppen können in zwei verschiedenen Weisen beschrieben werden:
oberflächlich oder umfangreich.

\subsubsection{Oberflächlich}
\apiArgument{id/Eindeutige Identifikationsnummer der Gruppe{,} mit der diese
    in weiteren Befehlen referenziert werden kann,
    name/Name der Gruppe,
    icon-hash/Hashwert des Gruppenportraits}

\subsubsection{Umfangreich}
\apiArgument{members/Array von Benutzerkonten{,} die Teil der Gruppe sind
(siehe \ref{sec:accountdesc})}
\apiArgument{events/Array von Verabredungen{,} die in der Gruppe erstellt worden
sind (siehe \ref{sec:eventdesc})}
\apiArgument{polls/Array von Abstimmungen{,} die in der Gruppe erstellt worden
sind (siehe \ref{sec:polldesc})}

\subsection{Benutzerkontobeschreibung}\label{sec:accountdesc}
\apiArgument{user/Benutzername des Kontos}
\apiArgument{icon-hash/Hashwert des Benutzerkontoportraits}

\subsection{Verabredungsbeschreibung}\label{sec:eventdesc}
\apiArgument{placeholder/placeholder}

\subsection{Abstimmugsbeschreibung}\label{sec:polldesc}
\apiArgument{placeholder/placeholder}

\end{document}
