\documentclass[parskip=full,11pt]{scrartcl}
\usepackage[utf8]{inputenc}
\usepackage[T1]{fontenc}
\usepackage[german]{babel}
\usepackage[useregional]{datetime2}
\usepackage[pdfborderstyle={/S/U/W 0}]{hyperref}
\usepackage[nameinlink]{cleveref}
\usepackage[section]{placeins}
\usepackage{xcolor}
\usepackage{graphicx}
\usepackage{csquotes}
\usepackage{amsmath} % for $\text{}$
\usepackage{entwurf}
\usepackage{enumitem}
\usepackage{algorithm}
\usepackage{algorithmicx}
\usepackage{algpseudocode}
\usepackage{listings}
\usepackage{bera}

\setlist{nosep}

\newcommand\urlpart[2]{$\underbrace{\text{\texttt{#1}}}{\text{#2}}$}
\raggedbottom
\crefname{figure}{Abb}{Abb}

\newcommand\producttitle{treff.}
\hypersetup{
	pdftitle={Entwurf: \producttitle},
	bookmarks=true,
}

% section numbers in margins:
\renewcommand\sectionlinesformat[4]{\makebox[0pt][r]{#3}#4}

\colorlet{punct}{red!60!black}
\definecolor{background}{HTML}{EEEEEE}
\definecolor{delim}{RGB}{20,105,176}
\colorlet{numb}{magenta!60!black}
\lstdefinelanguage{json}{
    basicstyle=\normalfont\ttfamily,
    numbers=left,
    numberstyle=\scriptsize,
    stepnumber=1,
    numbersep=8pt,
    showstringspaces=false,
    breaklines=true,
    frame=lines,
    backgroundcolor=\color{background},
    literate=
     *{0}{{{\color{numb}0}}}{1}
      {1}{{{\color{numb}1}}}{1}
      {2}{{{\color{numb}2}}}{1}
      {3}{{{\color{numb}3}}}{1}
      {4}{{{\color{numb}4}}}{1}
      {5}{{{\color{numb}5}}}{1}
      {6}{{{\color{numb}6}}}{1}
      {7}{{{\color{numb}7}}}{1}
      {8}{{{\color{numb}8}}}{1}
      {9}{{{\color{numb}9}}}{1}
      {:}{{{\color{punct}{:}}}}{1}
      {,}{{{\color{punct}{,}}}}{1}
      {\{}{{{\color{delim}{\{}}}}{1}
      {\}}{{{\color{delim}{\}}}}}{1}
      {[}{{{\color{delim}{[}}}}{1}
      {]}{{{\color{delim}{]}}}}{1},
}

\lstdefinelanguage{norm}{
    basicstyle=\normalfont\ttfamily,
    escapeinside={(*}{*)},
    showstringspaces=false,
    breaklines=true,
    frame=lines,
    backgroundcolor=\color{background},
    literate=
     *{0}{{{\color{numb}0}}}{1}
      {1}{{{\color{numb}1}}}{1}
      {2}{{{\color{numb}2}}}{1}
      {3}{{{\color{numb}3}}}{1}
      {4}{{{\color{numb}4}}}{1}
      {5}{{{\color{numb}5}}}{1}
      {6}{{{\color{numb}6}}}{1}
      {7}{{{\color{numb}7}}}{1}
      {8}{{{\color{numb}8}}}{1}
      {9}{{{\color{numb}9}}}{1}
      {:}{{{\color{punct}{:}}}}{1}
      {,}{{{\color{punct}{,}}}}{1}
      {\{}{{{\color{delim}{\{}}}}{1}
      {\}}{{{\color{delim}{\}}}}}{1}
      {[}{{{\color{delim}{[}}}}{1}
      {]}{{{\color{delim}{]}}}}{1},
}

% Default fixed font does not support bold face
\DeclareFixedFont{\ttb}{T1}{txtt}{bx}{n}{12} % for bold
\DeclareFixedFont{\ttm}{T1}{txtt}{m}{n}{12}  % for normal
\definecolor{deepblue}{rgb}{0,0,0.5}
\definecolor{deepred}{rgb}{0.6,0,0}
\definecolor{deepgreen}{rgb}{0,0.5,0}

% Python style for highlighting
\newcommand\pythonstyle{\lstset{
    language=Python,
    basicstyle=\normalfont\ttfamily,
    otherkeywords={self},
    keywordstyle=\ttb\color{deepblue},
    emph={normalize},
    emphstyle=\ttb\color{deepred},
    stringstyle=\color{deepgreen},
    frame=tb,
    showstringspaces=false,
    backgroundcolor=\color{background},
    numbers=left,
    numberstyle=\scriptsize,
    stepnumber=1,
    numbersep=8pt,
    escapeinside={(*}{*)},
}}


% Python environment
\lstnewenvironment{python}[1][]
{
\pythonstyle
\lstset{#1}
}
{}

% Python for external files
\newcommand\pythonexternal[2][]{{
\pythonstyle
\lstinputlisting[#1]{#2}}}

% header & footer
\usepackage{scrlayer-scrpage}
%\lofoot{\today}
%\refoot{\today}
\pagestyle{scrheadings}

\title{\includegraphics[width = 80mm]{images/logo_crop.png}}
\subtitle{\huge Entwurf}
\author{Lukas Dippon
        \and Jens Kienle
        \and Matthias Noll
        \and Fabian Röpke
        \and Tim Schmidt
        \and Simon Vögele}

\begin{document}

\maketitle
\thispagestyle{empty} % removed page number from title

\pagebreak
\tableofcontents

%%%%%%%%%%%%%%%%%%%
\pagebreak
\section{Einleitung}

\subsection{Client-Server}
% auf client-server architektur eingehen
% (Fließtext [kleiner Einleitungstext] und ggf kleines UML-Klassendiagram)

\subsection{Funktion und externes Verhalten - Client}
% Funktion und Verhalten des Clients (Fließtext und UML-Diagramme)

\subsection{Funktion und externes Verhalten - Server}
% Funktion und Verhalten des Servers (Fließtext und UML-Diagramme)

%%%%%%%%%%%%%%%%%%%%
\pagebreak
\section{Aufbau und internes Verhalten - Client}

\subsection{Aufbau}
% Aufteilung in GUI und Datenverarbeitung (Fließtext)
% Ihre Funktionen

\subsection{Bestandteile der GUI}
% Strukturierung der GUI durch Activities und deren Aufteilung in Fragments,
%	sowie das Zusammenspiel der Activities (Fließtext und UML-Klassendiagramme,
% ggf Sequenzdiagramme)

\subsection{Bestandteile Datenverarbeitung}
% Aufteilung der Funktionalitäten auf seperate Klassen,
% sowie deren Zusammenspiel (Fließtext und UML-Klassendiagramme, ggf Sequenzdiagramme)

\subsection{Kommunikation von GUI und Datenverarbeitung}
% Zusammenspiel von GUI und Datenverarbeitung
% (Fließtext und UML-Klassendiagramme, ggf Sequenzdiagramme)

\subsection{JavaDoc}

%%%%%%%%%%%%%%%%%%%%%%
\pagebreak
\section{Aufbau und internes Verhalten - Server}

\subsection{Aufbau}
% Aufteilung der Funktionalitäten auf seperate Klassen;
% Zusammenspiel dieser Klassen (Fließtext und UML-Klassendiagramme,
% ggf Sequenzdiagramme)

\subsection{Bestandteile des Servers}
% Aufbau und genaue Funktionalität dieser Klassen
% [z.B. Datenbank, Schnittstelle zum Client]
% (Fließtext und UML-Klassendiagramme, ggf Sequenzdiagramme)

\subsection{JavaDoc}

%%%%%%%%%%%%%%%%%%%%%%
\pagebreak
\section{Kommunikation zwischen Client und Server}
% Zusammenspiel von Client und Server
% (Fließtext und Sequenzdiagram, ggf UML-Klassendiagram)
% TODO: JSON verlinken / footnote
% TODO: Alphabet und Escapes genau festlegen
Client und Server kommunizieren mittels Nachrichten im JSON-Format.
Die oberste Ebene jeder Nachricht bildet ein einzelnes JSON-Objekt.
Parameter, Rückgabewerte und Fehlercodes werden als Eigenschaften dieses
Objektes kodiert.
Eine Anfrage beinhaltet immer mindestens eine Eigenschaft mit dem Schlüssel
\textbf{cmd}.
Die Namen der im folgenden definierten Befehle sind exakt die Werte, die bei
Kodierung als Nachricht den Wert der \textbf{cmd}-Eigenschaft bilden müssen.
Diese Namen sind als Zeichenketten zu kodieren.
\\Wird während der Ausführung des Befehls ein Fehler gefunden, möglicherweise
durch übermittelte Parameter bedingt, wird nur ein Fehlercode mit dem Schlüssel
\textbf{error} zurückgegeben. Die Bedeutung der Fehlercodes wird ebenfalls im
folgenden definiert.

\subsection{Befehle}
Bis auf dem \textbf{login}-Befehl erfordert jeder Befehl folgenden
Parameter:\\
\apiArgument{token/Authentifizierungtoken{,} welches beim Einloggen in das
Benutzerkonto vom Server zurückgegeben wurde}
\par\par Ist das übermittelte Token ungültig, wird der Befehl nicht
ausgeführt und folgender Fehlercode zurückgegeben:\\
\apiErrorCode{010/Authentifizierungtoken ungültig}

\apiCommand{login}
{Einloggen in ein Nutzerkonto}
{user/Benutzername des Kontos,
pass/Passwort des Kontos}
{token/Authentifizierungstoken{,} welches in weiteren Befehlen
als Eingabe verlangt wird}
{100/Nutzername{/}Passwort-Kombination ungültig}

\apiCommand{list-groups}
{Auflisten der Gruppen, in denen das Benutzerkonto Mitglied ist}
{---/Keine weiteren Parameter}
{groups/Ein Array an oberflächlich beschriebenen Gruppen (siehe
Kapitel \ref{sec:groupdesc})}
{---/Keine weiteren Fehlercodes}

\apiCommand{get-group-details}
{Abrufen von Details einer Gruppe, in denen das Benutzerkonto Mitglied ist}
{id/Identifikationsnummer der Gruppe}
{\dots/Eine ausführliche Beschreibung der Gruppe{,} die zu der angegebenen
Identifikationsnummer gehört (siehe \ref{sec:groupdesc})}
{300/Gruppen-Identifikationsnummer ungültig oder Benutzerkonto nicht Teil der
Gruppe}

\subsection{Gruppenbeschreibung}\label{sec:groupdesc}

\apiArgument{id/Eindeutige Identifikationsnummer der Gruppe{,} mit der diese in
    weiteren Befehlen referenziert werden kann,
    name/Name der Gruppe,
    icon-checksum/Hashwert des Gruppenportraits,
    members/Array an Beschreibungen von Benutzerkonten{,} die Teil der Gruppe
    sind (siehe \ref{sec:accountdesc}),
    events/Array an oberflächlich beschriebenen Verabredungen{,} die in der
    Gruppe erstellt worden sind (siehe \ref{sec:eventdesc}),
    polls/Array an oberflächlich beschriebenen Abstimmungen{,} die in der
    Gruppe erstellt worden sind (siehe \ref{sec:polldesc})
}

\subsection{Benutzerkontobeschreibung}\label{sec:accountdesc}
\apiArgument{type/\enquote{account},
    id/Eindeutige Identifikationsnummer des Benutzerkontos{,} mit der
    dieses in weiteren Befehlen referenziert werden kann,
    user/Benutzername des Kontos,
    icon-checksum/Hashwert des Benutzerkontoportraits
}

\subsection{Verabredungsbeschreibung}\label{sec:eventdesc}

\subsubsection{Oberflächlich}
\apiArgument{type/\enquote{event},
    id/Eindeutige Identifikationsnummer der Verabredung{,} mit der
    diese in weiteren Befehlen referenziert werden kann,
    checksum/Prüfsumme der ausführlichen Beschreibung (siehe
    \ref{sec:checksum})
}

\subsubsection{Ausführlich}
\apiArgument{type/\enquote{event},
    id/Eindeutige Identifikationsnummer der Verabredung{,} mit der
    diese in weiteren Befehlen referenziert werden kann,
    title/Titel der Verabredung,
    creator/Nutzername des Benutzerkontos{,} mit dem die Verabredung erstellt
    wurde,
    time-start/Startzeitpunkt,
    time-end/Endzeitpunkt,
    latitude/Breitengrad{,} auf dem die Verabredung stattfindet
    (Gleitkommazahl),
    longitude/Längengrad{,} auf dem die Verabredung stattfindet
    (Gleitkommazahl),
    participants/Array an Benutzernamen der Benutzerkonten{,} die an der
    Verabredung teilnehmen
}

\subsection{Abstimmungsbeschreibung}\label{sec:polldesc}
\apiArgument{/placeholder}

\subsection{Prüfsummen für ausführliche Beschreibungen}
\newcommand{\hashAlg}{MD5}
Um unnötige ausführliche Übertragungen zu vermeiden beinhalten oberflächliche
Beschreibungen Prüfsummen der zum selben Objekt gehörenden ausführlichen
Beschreibung.
Der Client ist somit in der Lage, festzustellen, ob seine lokalen Datensätze
mit denen des Servers übereinstimmen, ohne die Datensätze selbst anfordern zu
müssen.

\par Zur Berechnung der Prüfsummen wird der \hashAlg-Algorithmus verwendet.

\par Da es mehrere unterschiedliche, aber äquivalente JSON-Darstellungen für
eine ausführliche Beschreibungen gibt, müssen diese vor Eingabe an den
\hashAlg-Algorithmus in eine normalisierte Form gebracht werden.

\par \textit{Hinweis: Im Folgenden bezeichnen Numerale nicht etwa die ASCII-Symbole
\enquote{0} bis \enquote{9} mit den Werten 48 bis 57, sondern tatsächlich die
nicht-druckbaren Zeichen mit den Werten 0 bis 9.}\\
JSON-Objekte werden mit einer 1 begonnen und mit einer 3 beendet.
Eigenschaften dieser JSON-Objekte werden mit einer 2 statt einem Komma
getrennt.
Die Eigenschaften werden gemäß der in den Beschreibungsspezifikationen
aufgeführten Reihenfolgen sortiert.
JSON-Arrays werden mit einer 4 begonnen und mit einer 6 beendet.
Elemente dieser JSON-Arrays werden mit einer 5 statt mit einem Komma getrennt.
In JSON-Arrays von Beschreibungen kommen lediglich Beschreibungen anderer
Objekte vor. Diese werden aufsteigend nach ihren IDs sortiert.

Alle JSON-Objekteigenschaften werden ohne Schlüssel, Doppelpunkte, geschwungene
Klammern und zusätzliche Leerzeichen oder Zeilenumbrüche konkateniert.
Alle JSON-Arrayelemente werden ohne eckige Klammern und zusätzliche
Leerzeichen oder Zeilenumbrüche konkateniert.
Die doppelten Anführungszeichen um Zeichenketten werden entfernt.
Da alle sechs Trennzeichen nur als Escape-Sequenzen, nicht aber direkt im
JSON-Format vorkommen können, ist diese Darstellung eindeutig.
% TODO: Das auch tatsächlich sicherstellen

\par Im folgenden beispielhaft eine Verabredung mit folgenden Details:
\begin{itemize}
    \item Besitzt die ID 12
    \item Trägt den Titel \enquote{Beispielverabredung 1010}
    \item Erstellt vom Benutzerkonto mit ID 200
    \item Beginnt am 01. Januar 2018 um 00:00:00.000
    \item Endet am 03. Januar 2018 um 18:30:00.000
    \item Findet auf der Position 49.011978 N, 8.416377 E statt
    \item Die Benutzerkonten mit den IDs 200, 3 und 43378 sind als Teilnehmer
        eingetragen
\end{itemize}

Eine mögliche Kodierung der ausführlichen Beschreibung im JSON-Format kann wie
folgt aussehen:
\begin{lstlisting}[language=json,firstnumber=1]
{   "type": "event",
    "id": 2,
    "title": "Beispielverabredung 1010",
    "creator": 200,
    "time-start": 1514761200000,
    "time-end": 1515000600000,
    "latitude": 49.011978,
    "longitude": 8.416377,
    "participants": [
        {   "type": "account",
            "id": 3,
            "user": "w4rum",
            "icon-checksum": "123abc"
        },
        {   "type": "account",
            "id": 200,
            "user": "notw4rum",
            "icon-checksum": "blaa"
        },
        {   "type": "account",
            "id": 43378,
            "user": "absolutetlynotw4rum",
            "icon-checksum": "foobar"
        }
    ]
}
\end{lstlisting}

\newcommand*\circled[1]{\tikz[baseline=(char.base)]{
            \node[shape=circle,draw,inner sep=1pt] (char) {\scriptsize #1};}}

Die normalisierte Darstellung dieser Verabredung sieht wie folgt aus.
Die nicht-druckbaren Zeichen werden durch \circled{1}, \circled{2}, \circled{3}
etc.  dargestellt.\\
\textit{Hinweis: Die Zeilenumbrüche dienen nur der besseren Lesbarkeit und sind
nicht in der normalisierten Darstellung enthalten.}

\begin{lstlisting}[language=norm,firstnumber=1]
(*\circled{1}*)"event"(*\circled{2}*)2(*\circled{2}%
*)"Beispielverabredung 1010"(*\circled{2}*)200(*\circled{2}*)1514761200000
(*\circled{2}*)1515000600000(*\circled{2}*)49.011978(*\circled{2}%
*)8.416377(*\circled{2}*)(*\circled{4}*)(*\circled{1}*)"account"(*\circled{2}%
*)3(*\circled{2}*)"w4rum"(*\circled{2}*)"123abc"
(*\circled{3}*)(*\circled{5}*)(*\circled{1}*)"account"(*\circled{2}*)200(*%
\circled{2}*)"notw4rum"(*\circled{2}*)"blaa"(*\circled{3}\circled{5}%
\circled{1}*)"account"(*\circled{2}*)43378
(*\circled{2}*)"absolutetlynotw4rum"(*\circled{2}*)"foobar"(*\circled{3}%
\circled{6}\circled{3}*)
\end{lstlisting}

\par Ein beispielhafter Algorithmus für die Normalisierung ist im folgenden
als Python-Implementierung angegeben:

\pythonexternal{scripts/normalize.py}

\end{document}
