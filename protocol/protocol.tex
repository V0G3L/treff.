\documentclass[parskip=full,11pt]{scrartcl}
\usepackage[utf8]{inputenc}
\usepackage[T1]{fontenc}
\usepackage[english]{babel}
\usepackage[useregional]{datetime2}
\usepackage[pdfborderstyle={/S/U/W 0}]{hyperref}
\usepackage[nameinlink]{cleveref}
\usepackage[section]{placeins}
\usepackage[top=2.5cm, bottom=2.5cm, left=4cm, right=3cm]{geometry}
\usepackage{xcolor}
\usepackage{graphicx}
\usepackage{csquotes}
\usepackage{amsmath} % for $\text{}$
\usepackage{protocol} % local .sty file
\usepackage{enumitem}
\usepackage{algorithm}
\usepackage{algorithmicx}
\usepackage{algpseudocode}

\setlist{nosep}

\newcommand\urlpart[2]{$\underbrace{\text{\texttt{#1}}}{\text{#2}}$}
\raggedbottom
\crefname{figure}{Abb}{Abb}

\newcommand\producttitle{treff.}
\newcommand\protocolversion{0.3}
\hypersetup{
	pdftitle={\producttitle~- Protocol Specification v\protocolversion},
	bookmarks=true,
}

% section numbers in margins:
\renewcommand\sectionlinesformat[4]{%
    \ifstr{#1}{subsubsection}{%
        \makebox[0pt][r]{#4}%
    }{%
        \makebox[0pt][r]{#3}#4%
    }%
}%

% header & footer
\usepackage{scrlayer-scrpage}
%\lofoot{\today}
%\refoot{\today}
\pagestyle{scrheadings}
%\let\raggedsection\centering

\title{\includegraphics[width = 80mm]{images/logo_crop.png}}
\subtitle{\huge Protocol Specification\\
          v\protocolversion}
\author{Lukas Dippon
        \and Jens Kienle
        \and Matthias Noll
        \and Fabian Röpke
        \and Tim Schmidt
        \and Simon Vögele}

\begin{document}

\maketitle
\thispagestyle{empty} % removed page number from title

\pagebreak
\tableofcontents

%%%%%%%%%%%%%%%%%%%
\pagebreak
\section{Network protocol and command format}
% Zusammenspiel von Klient und Server
% (Fließtext und Sequenzdiagram, ggf UML-Klassendiagram)
% TODO: JSON verlinken / footnote
% TODO: Alphabet und Escapes genau festlegen
Client and server communicate via a TLS-encrypted TCP connection.
This implies that a server must possess a valid SSL certificate.

The commands specified below are encoded as a single JSON object each and are
sent from client to server.
If not specified otherwise, each request produces a single response from the
server, also encoded as a JSON object.
A request is considered to be completely transferred to the server once the
server receives the closing curly bracket for the top-level JSON object.

The \textbf{cmd} property of the top-level JSON object of a request defines
which command the client is trying to execute.
As such, the \textbf{cmd} property must always be included in a request.
Parameters, return values and error codes are also encoded as properties of
the top-level JSON object of a request or response.

The names of the commands specified below are to be used as the value of the
\textbf{cmd} property.
These are to be encoded as strings.

If an error occurs during command execution, possibly caused by invalid
parameters, the response JSON object consists of only the single integer
property \textbf{error}.
The meaning of these error codes is specified in chapter \ref{sec:errorcodes}.

\section{Data types and restrictions}
While JSON itself defines several data types, the property
definitions in the following chapters use the conventional, more restrictive
variants as used, for example, in the Java programming language.
As such, mentions of \code{int} and \code{long} refer to the signed variants
within the upper and lower bounds defined in the Java programming language.

Strings may contain any Unicode symbol as long as they are properly escaped to
match the restriction set by JSON specifications.

The string properties defined in chapter \ref{sec:descriptions} might impose
additional restrictions, such as maximum amount of characters, i.e. Unicode
symbols, or a reduced set of symbols, e.g. only alphanumeric characters.
In this document, the set of alphanumeric characters is defined to consist
of arabic numbers and lower-case and upper-case letters of the ISO basic Latin
alphabet (a-z, A-Z, 0-9).

Properties of type \code{date} are to be represented as Unix time%
\footnote{\enquote{Unix time}, Wikipedia, Version 822615863,
\url{https://en.wikipedia.org/w/index.php?title=Unix_time&oldid=822615863}},
and encoded as \code{long}s.

Properties of type \code{decimal degree} must be valid decimal degrees
representing either latitude or longitude and are to be encoded as
\code{double}s.

While JSON supports arrays of different data types, this protocol does not.
All arrays defined here consist of elements of the same type.

\section{Object descriptions}\label{sec:descriptions}
In order to transmit information about domain model objects via JSON encoded
requests and reponses, the encoding of the objects into \enquote{descriptions}
is specified below.

Each object can be encoded as a shallow, complete, or recursively
complete description.
While complete descriptions contain all properties of the object itself and
may be requested from the server via the respective \textbf{get-*-details}
commands, recursively complete descriptions contain the complete descriptions
of certain types of referenced objects, e.g. a recursively complete description
of a usergroup contains recursively complete descriptions of all events and
polls.

Recursively complete descriptions are not transferred in any of the
commands specified in chapter \ref{sec:commands}.
Instead, they are used to generate the checksums contained in shallow
descriptions (see chapter \ref{sec:checksum}).

The commands defined in chapter \ref{sec:commands} might not use all of the
descriptions specified below.
For the sake of consistency, they are listed here anyways.

\apiDesc{Usergroup}{
    name/string/Name of the group,
    members/membership array/Complete descriptions of all memberships that are
    part of this group,
    events/int array/IDs of all events that are part of this group,
    polls/int array/IDs of all polls that are part of this group
}{
    events/object array/Complete descriptions of all events,
    polls/object array/Complete descriptions of all polls
}{
    name/Up to 64 characters
}

\apiDesc{Account}{
    username/string/Username of the account
}{}{
    username/Alphanumeric{,} up to 32 characters
}

\apiDesc{Event}{
    title/string/Title of the event,
    creator/int/ID of the account that created this event,
    time-start/date/Point in time at which the event starts,
    time-end/date/Point in time at which the event ends,
    latitude/decimal degree/Latitude of the position that the event is centered
    on,
    longitude/decimal degree/Longitude of the position that the event is
    centered on,
    participants/int array/IDs of all group members that are currently
    participating or are planning to participate in the event
}{}{
    title/Up to 64 characters
}

\apiDesc{Poll}{
    question/string/Question for which answered are supplied as poll options,
    creator/int/ID of the account that created this poll,
    time-close/date/Point in time at which voting closes,
    multi-choice/boolean/True iff an account can vote for multiple poll options
    simultaneously,
    options/int array/IDs of all poll options that are part of this poll
}{}{
    question/Up to 64 characters
}

\apiDesc{Poll option}{
    latitude/decimal degree/Latitude of the suggested position,
    longitude/decimal degree/Longitude of the suggested position,
    time-start/date/Suggested point in time at which the resulting event should
    start,
    time-end/date/Suggested point in time at which the resulting event should
    end,
    supporters/int array/IDs of all account that are currently voting for this
    poll option
}{}{}

\apiDesc{Membership}{
    account-id/int/ID of the account that this membership belongs to,
    permissions/object/Object holding a \code{permission-name -> boolean} pair
    for each permission defined in chapter \ref{sec:permissions}.
}{}{}

\apiDesc{Update}{
    TODO/TODO/TODO
}{}{}

\section{Commands}\label{sec:commands}

\apiErrorNew{syntax}{1000}{parameter wasn't given or has a wrong format}
\apiErrorNew{unkown-command}{1001}{unknown command}
\apiErrorNew{token-invalid}{1100}{authentication token invalid}
\apiErrorNew{cred-wrong}{1101}{username/password combination invalid}
\apiErrorNew{reset-code-invalid}{1102}{password reset code invalid}
\apiErrorNew{user-id-invalid}{1200}
{at least one account identification number is invalid}
\apiErrorNew{group-id-invalid}{1201}
{group identification number is invalid or account not part of the group}
\apiErrorNew{event-id-invalid}{1202}{event identification number invalid}
\apiErrorNew{poll-id-invalid}{1203}{poll identification number invalid}
\apiErrorNew{poll-option-id-invalid}{1204}
{poll option identification number invalid}
\apiErrorNew{username-already-in-use}{1300}{username already in use}
\apiErrorNew{email-invalid}{1301}{email is invalid}
\apiErrorNew{username-invalid}{1302}{username is invalid}
\apiErrorNew{time-end-in-past}{1400}{at least one end time is in the past}
\apiErrorNew{time-end-start-conflict}{1401}
{at least one end time is not after the corresponding start time}
\apiErrorNew{time-measured-future}{1402}
{the time of measurement is too far in the future}
\apiErrorNew{already-in-contacts}{1500}{account already in contact list}
\apiErrorNew{not-in-contacts}{1501}{account isn't part of the own contact list}
\apiErrorNew{reflexive-contact}{1502}{account can't be a contact of itself}
\apiErrorNew{contact-request-pending}{1503}{contact request already sent and
still pending}
\apiErrorNew{no-contact-request}{1504}{no contact request from specified user}
\apiErrorNew{being-blocked}{1505}{this account was blocked by the given account}
\apiErrorNew{blocking-already}{1506}
{the given account was blocked by this account}
\apiErrorNew{not-blocking}{1507}
{the given account wasn't blocked by this account}
\apiErrorNew{reflexive-block}{1508}{account can't block itself}
\apiErrorNew{no-user-invited}{1509}{no account invited except the own account}
\apiErrorNew{user-already-in-group}{1510}
{at least one account is already part of the group}
\apiErrorNew{user-not-in-group}{1511}
{at least one account is not part of the group}
\apiErrorNew{already-participating-event}{1512}
{account already part of this event}
\apiErrorNew{not-participating-event}{1513}{account no part of this event}
\apiErrorNew{already-voting-for-option}{1514}
{account already voted for this option}
\apiErrorNew{poll-not-multichoice}{1515}
{multi-choice is not available for this poll}
\apiErrorNew{not-voting-for-option}{1516}{account didn't vote for this option}
\apiErrorNew{no-permission-edit-permissions}{2000}
{account doesn't possess the permission to edit permissions}
\apiErrorNew{no-permission-edit-group}{2100}
{account doesn't possess the permission to edit the group}
\apiErrorNew{no-permission-manage-members}{2101}
{account doesn't possess the permission to edit the list of group members}
\apiErrorNew{no-permission-create-event}{2200}
{account doesn't possess the permission to create events in this group}
\apiErrorNew{no-permission-edit-any-event}{2201}
{account doesn't possess the permission to edit the event}
\apiErrorNew{no-permission-create-poll}{2300}
{account doesn't possess the permission to create polls in this group}
\apiErrorNew{no-permission-edit-any-poll}{2301}
{account doesn't possess the permission to edit the poll}

All commands except for \textbf{login}, \textbf{register},
\textbf{reset-password}, and \textbf{reset-password-confirm} require the
following parameter:\\
\apiArgument{token/String/Authentication token{,} produced by \textbf{login} or
\textbf{register}}
\par Transmission of an invalid token will produced the following error:\\
\apiErrorPrint{token-invalid}

\apiCommand{register}
{Registers a new user account.
Declaring an e-mail adress is optional.
It may be set at a later time via the \textbf{edit-email} command.}
{user/string/Username of the account,
pass/string/Password of the account}
{token/string/Authentication token}
{username-already-in-use}
{}

\apiCommand{login}
{Produces an authentication token for the specified user account if the
password is correct.}
{user/string/Username of the account,
pass/string/Password of the account}
{token/string/Authentication token}
{cred-wrong}
{}

\apiCommand{request-updates}
{Requests the list of undelivered updates from the server.}
{}
{updates/object array/All undelivered updates for the calling account{,} sorted
from oldest to youngest (see chapter \ref{subsec:Update-desc}).}
{user-id-invalid}
{}

\apiCommand{edit-username}
{Changes the executing user's username.}
{username/string/New username}
{}
{username-already-in-use}
{account-change}

\apiCommand{edit-email}
{Changes the executing user's email address.}
{email/string/New email address.
Must be a valid email address according to
RFC 822\footnote{See \url{https://tools.ietf.org/html/rfc822}}.}
{}
{cred-wrong, user-id-invalid, email-invalid}
{}

\apiCommand{edit-password}
{Changes the executing user's password.
Requires submission of the current password.}
{pass/string/Current password,
new-pass/string/New password}
{}
{cred-wrong, user-id-invalid}
{}

\apiCommand{reset-password}
{Requests a password reset code.
If the supplied email address is associated with a user account, then a
randomly generated sequence of 64 alphanumerical symbols, the password reset
code, is emailed to that address.
This sequence can be used as a paremeter in the \textbf{reset-password-confirm}
command to overwrite the current password.
The password reset code must be unique among all user accounts and should not
be reused.
If the supplied email address is not associated with any account, \textbf{no}
error is returned.
This prevents leaking information about users' email addresses.}
{email/string/The email address associated with the user account for which the
password reset request is to be made.
Must be a valid email address according to
RFC 822\footnote{See \url{https://tools.ietf.org/html/rfc822}}.}
{}
{}
{}

\apiCommand{reset-password-confirm}
{Resets the password of the account associated with the supplied passwort reset
code.
After invocation of this command, the password reset code is invalidated and
should be discarded.}
{code/string/Password reset code{,} as received via email after invocation of
    the \textbf{reset-password} command,
new-pass/string/New password}
{}
{reset-code-invalid}
{}

\apiCommand{delete-account}
{Deletes the executing user's account.
This will also invalidate a user's login token, hence preventing execution of
any further commands that require authentication.}
{pass/String/Passwort des eigenen Kontos}
{}
{cred-wrong}
{}

\apiCommand{get-user-id}
{Queries the the user account ID associated with the supplied username.}
{user/string/Username of the account}
{id/int/ID of the account associated with the supplied username}
{username-invalid}
{}

\apiCommand{send-contact-request}
{Sends a contact request to the specified user.
This will fail if either the specified user is already part of the executing
user's contact list or one of the users is blocking the other.}
{id/int/ID of the account}
{}
{user-id-invalid,being-blocked,blocking-already,contact-request-pending}
{}

\apiCommand{cancel-contact-request}
{Cancels a previously sent and still pending contact request that was issued
by the executing user.}
{id/int/ID of the account to which the request was sent}
{}
{user-id-invalid,no-contact-request}
{}

\apiCommand{accept-contact-request}
{Accepts the contact request of the specified user.
This adds the users to the contact list of each other.}
{id/int/ID of the account}
{}
{user-id-invalid,no-contact-request}
{}

\apiCommand{reject-contact-request}
{Rejects the contact request of the specified user.}
{id/int/ID of the account}
{}
{user-id-invalid,no-contact-request}
{}

\apiCommand{remove-contact}
{Removes a user from the executing user's contact list.
Will also remove the executing user from the specified user's contact list.}
{id/int/ID of the account}
{}
{user-id-invalid,not-in-contacts}
{}

\apiCommand{get-contact-list}
{Lists all contacts and all incoming and outgoing contact requests of the
executing user.}
{}
{contacts/int array/IDs of all accounts in contact list,
incoming-requests/int array/IDs all accounts that have sent a still pending
contact request to the executing user,
outgoing-requests/int array/IDs all accounts to which the executing user has
sent a still pending contact request}
{}
{user-id-invalid}

\apiCommand{block-account}
{Blocks a user.
Until the block is lifted, neither the executing user nor the blocked user can
send contact request to the other party.
If the users are in each other's contact list, they are removed from the
contact lists, just as if one of them had executed the \textbf{remove-contact}
command, before the block is applied.
Any pending contact requests are automatically rejected.}
{id/int/ID of the account}
{}
{user-id-invalid,blocking-already}
{}

\apiCommand{unblock-account}
{Lifts the block of the specified user that was previously issued by the
executing user.
This does not lift any block that was issued against the executing user.}
{id/int/ID of the blocked user's account}
{}
{user-id-invalid,not-blocking}
{}

\apiCommand{create-group}
{Creates a new user group.}
{group/object/Complete description of the new user group{,} omitting the ID.
The members array must contain both the executing user's account ID and at
least one other account ID.}
{id/int/ID of the newly created user group}
{user-id-invalid,no-user-invited}
{}

\apiCommand{edit-group}
{Edits a user group of which the executing user is a member.}
{group/object/New complete description of the group{,} omitting the members
array.
The ID specified in the description must match the ID of the group that
is to be edited.}
{}
{group-id-invalid,no-permission-edit-group}
{}

\apiCommand{add-group-members}
{Adds one or more members to a user group of which the executing user is a
member.}
{group-id/int/ID of the user group,
members/int array/ID of all user accounts that are to be added to the user
group.
No account may be specified that is already a member of the group.
In particular{,} the executing user's account ID may not be specified.}
{}
{group-id-invalid,user-id-invalid,no-permission-manage-members,%
user-already-in-group}
{}

\apiCommand{remove-group-members}
{Removes one or more accounts from a user group of which the executing user is
a member.}
{id/int/The ID of the group,
members/account array/Array of IDs of all members to remove from the user
group.
No account may be specified that is not a member of the group.
The executing user's account may be specified as well.
This will remove the executing user's account and therefore prevent any further
interaction of the executing user with the group unless a remaining member adds
him to the group again.
If all remaining members of the user group are removed by this command then the
user group will be deleted.}
{}
{group-id-invalid,user-id-invalid,no-permission-manage-members,
user-not-in-group}
{}

\apiCommand{get-membership-details}
{Gets the membership details of an account in a user group of which the
executing user is a member.
This command can be used to query a group member's permissions.}
{id/int/The ID of the user group,
user-id/int/The ID of the account}
{membership/object/The complete description of the membership belonging to
the specified user in the specified group (\ref{sec:Membership-desc})}
{group-id-invalid, user-not-in-group}
{}

\apiCommand{edit-membership}
{Edits the membership details of an account in a user group of which the
executing user is a member.
This command can be used to edit a group member's permissions.
To revoke a membership, i.e. remove a user from a group, use the
\textbf{remove-group-members} command.}
{id/int/The ID of the user group,
membership/object/The complete description of the new membership that is to
replace the current one (see \ref{sec:Membership-desc}).
The ID specified in the description must match the ID of the account whose
membership is to be edited.}
{}
{group-id-invalid, no-permission-edit-permissions}
{}

\apiCommand{create-event}
{Creates an event in a user group of which the executing user is a member.}
{group-id/int/The ID of the user group,
event/object/A new complete description of the event
without the ID and the participants array}
{id/int/The ID of the event}
{group-id-invalid,time-end-in-past,time-end-start-conflict,
no-permission-create-event}
{}

\apiCommand{edit-event}
{Edits an event of a user group of which the executing user is a member.}
{group-id/int/The ID of the user group,
event/object/New complete description of the event without the participants
array.
The ID specified in the description must match the ID of the event that
is to be edited.}
{}
{group-id-invalid,time-end-in-past,time-end-start-conflict,event-id-invalid,
no-permission-edit-any-event}
{}

\apiCommand{join-event}
{Joins an event of a user group of which the executing user is a member.}
{group-id/int/The ID of the user group,
id/int/The ID of the event}
{}
{group-id-invalid,event-id-invalid,already-participating-event}
{}

\apiCommand{leave-event}
{Leaves an event of a user group of which the executing user is a member.}
{group-id/int/The ID of the user group,
id/int/The ID of the event}
{}
{group-id-invalid,event-id-invalid,not-participating-event}
{}

\apiCommand{remove-event}
{Removes an event from a user group of which the executing user is a member.}
{group-id/int/The ID of the user group,
id/int/The ID of the event}
{}
{group-id-invalid,event-id-invalid,no-permission-edit-any-event}
{}

\apiCommand{create-poll}
{Creates a new poll in a user group of which the executing user is a member.}
{group-id/int/The ID of the user group,
poll/object/A new complete description of the poll without the ID, the
creator ID, and the options array}
{id/int/The ID of the poll}
{group-id-invalid,time-end-in-past,time-end-start-conflict,
no-permission-create-poll}
{}

\apiCommand{edit-poll}
{Edits a poll of a user group of which the executing user is a member.}
{group-id/int/The ID of the user group,
poll/object/A new complete description of the poll option without
the options array and the creator ID.
The ID specified in the description must match the ID of the poll that
is to be edited.}
{}
{group-id-invalid,poll-id-invalid,no-permission-edit-any-poll}
{}

\apiCommand{add-poll-option}
{Adds a poll option to a poll of a user group of which the executing user is a
member.}
{group-id/int/The ID of the user group,
poll-id/int/The ID of the poll,
poll-option/object/Complete description of the new poll option without the ID and the
supporters array.}
{id/int/The ID of the new poll option}
{group-id-invalid,poll-id-invalid,time-end-in-past,time-end-start-conflict,
no-permission-edit-any-poll}
{}

% TODO: „Wählerarray“ besser formulieren
\apiCommand{edit-poll-option}
{Edits a poll option of a poll of a user group of which the executing user is
a member.}
{group-id/int/The ID of the user group,
poll-id/int/The ID of the poll,
poll-option/object/A new complete description of the poll option without the
supporters array.
The ID specified in the description must match the ID of the poll option that
is to be edited.}
{}
{group-id-invalid,poll-id-invalid,poll-option-id-invalid,%
no-permission-edit-any-poll}
{}

\apiCommand{vote-for-option}
{Votes for a poll option of a poll of a user group of which the executing user
is a member.}
{group-id/int/The ID of the user group,
poll-id/int/The ID of the poll,
id/int/The ID of the poll option}
{}
{group-id-invalid,poll-id-invalid,poll-option-id-invalid,poll-not-multichoice,
already-voting-for-option}
{}

\apiCommand{withdraw-vote-for-option}
{Withdraws the vote of the executing user from a poll option.}
{group-id/int/The ID of the user group,
poll-id/int/The ID of the poll,
id/int/The ID of the poll option}
{}
{group-id-invalid,poll-id-invalid,poll-option-id-invalid,not-voting-for-option}
{}

\apiCommand{remove-poll-option}
{Removes a poll option from a poll of a user group of which the executing user
is a member.}
{group-id/int/The ID of the group,
poll-id/int/The ID of the poll,
id/int/The ID of the poll option}
{}
{group-id-invalid,poll-id-invalid,group-id-invalid,poll-id-invalid,
poll-option-id-invalid,no-permission-edit-any-poll}
{}

\apiCommand{remove-poll}
{Removes a poll from a user group of which the executing user is a member.}
{group-id/int/The ID of the group,
id/int/The ID of the poll}
{}
{group-id-invalid,poll-id-invalid,no-permission-edit-any-poll}
{}

\apiCommand{send-chat-message}
{Sends a chat message to a user group of which the executing user is a member.}
{group-id/int/the ID of the user group,
message/string/The chat message}
% TODO: Länge beschränken (vllt. auch Gruppengröße etc)
{}
{group-id-invalid}
{}

\apiCommand{list-groups}
{Lists all groups of which the executing user is a member.}
{}
{groups/user group array/An array of shallow descriptions of every user group
(see \ref{subsec:Usergroup-desc})}
{}
{}

\apiCommand{get-group-details}
{Gets the details of a user group of which the executing user is a member.}
{id/int/The ID of the user group}
{user-group/object/The complete description of the user group
(\ref{subsec:Usergroup-desc})}
{group-id-invalid}
{}

\apiCommand{get-user-details}
{Gets the details of an account.}
{id/int/the ID of the account}
{account/object/The complete description of the account
(\ref{subsec:Account-desc})}
{user-id-invalid}
{}

% TODO: Definiere „aktive Verabredung“
\apiCommand{get-event-details}
{Gets the details of an event in a user group of which the executing user is
a member.}
{id/int/The ID of the event,
group-id/int/The ID of the group}
{event/object/The complete description of the event (see \ref{subsec:Event-desc})}
{group-id-invalid,event-id-invalid}
{}

\apiCommand{get-poll-details}
{Gets the details of a poll in a user group of which the executing user is
a member.}
{id/int/The ID of the poll,
group-id/int/The ID of the group}
{poll/object/The complete description of the poll (see \ref{subsec:Poll-desc})}
{group-id-invalid,poll-id-invalid}
{}

% TODO: Zeit-Toleranz / Annahmen über Synchronität weiter oben spezifizieren
\apiCommand{request-position}
{Sends a position request to all members of a user group of which the executing
user is a member.}
{id/int/The ID of the group,
time/date/Point in time until which a location transmission is requested}
{}
{group-id-invalid}
{}

% TODO: Zeit-Toleranz / Annahmen über Synchronität weiter oben spezifizieren
\apiCommand{update-position}
{Updates the position of the executing user.
This will notify all members of all user groups for which the executing user
has enabled location sharing.
Executing this command while the executing user has not enabled location
sharing with any groups has no effect.}
{latitude/decimal degree/The latitude of the position,
longitude/decimal degree/The longitude of the position,
time-measured/date/Point in time at which the measurement was taken}
{}
{time-measured-future}
{}

% TODO: Zeit-Toleranz / Annahmen über Synchronität weiter oben spezifizieren
\apiCommand{publish-position}
{Enables sharing of the executing user's position with a user group that he is
a member of.
All members of the specified group will gain access to the executing user's
location while sharing is active.
Location sharing will be active until the specified point in time.
Specifying a point in time that lies in the past will immediately stop location
sharing with the specified group.
Note that \textbf{update-position} should be called immediately after this
command returns, as group member do not have access to locations updated before
sharing was active.
}
{group-id/int/ID of the user group with which to share the location,
time-end/date/Point in time at which the location sharing ends.
May lie in the past.}
{}
{group-id-invalid}
{}

\section{Error codes}\label{sec:errorcodes}
\apiErrorTable

\section{Updates}
To inform users of changes that affect them, update objects are created with
each successfully executed command that makes changes to domain model objects.

These update objects are divided in several types. All of them hold at least
the following properties:

\apiArgument{
    type/string/Type of the update object{,} see list below,
    time-created/date/Point in time at which this update was created,
    creator/int/ID of the account that caused the creation of this update
}

\updateType{account-change}{
    \apiArgument{
        account/object/Complete description of the account that changed{,}
        omitting all properties that didn't change.
    }
}

\updateType{chat}{
    \apiArgument{
        group-id/int/ID of the group to which the chat message was sent,
        message/string/Content of the message
    }
}

\section{Prüfsummen für ausführliche Beschreibungen}\label{sec:checksum}
\newcommand{\hashAlg}{MD5}
Um unnötige ausführliche Übertragungen zu vermeiden beinhalten oberflächliche
Beschreibungen Prüfsummen der zum selben Objekt gehörenden ausführlichen
Beschreibung.
Der Klient ist somit in der Lage, festzustellen, ob seine lokalen Datensätze
mit denen des Servers übereinstimmen, ohne die Datensätze erneut anfordern zu
müssen.

\par Zur Berechnung der Prüfsummen wird der \hashAlg-Algorithmus verwendet.

\par Als Grundlage der Eingabe an den \hashAlg-Algorithmus wird die
ausführliche Beschreibung des jeweiligen Objektes im gleichen Format wie in den
vorangegangenen Kapiteln verwendet.

\par Da es aufgrund von Umordnungen, Leerzeichen, Zeilenumbrüchen etc. mehrere
unterschiedliche, aber äquivalente JSON-Darstellungen für eine ausführliche
Beschreibung gibt, müssen diese vor Eingabe an den \hashAlg-Algorithmus in eine
normalisierte Form gebracht werden:

\par \textit{Hinweis: In dem folgenden Absatz bezeichnen Ziffern nicht etwa die
ASCII-Ziffern \enquote{0} bis \enquote{9} mit den Werten 48 bis 57, sondern die
nicht darstellbaren Steuerzeichen mit den Werten 0 bis 9.}\\
JSON-Objekte werden mit einer 1 begonnen und mit einer 3 beendet.
Eigenschaften dieser JSON-Objekte werden mit einer 2 statt einem Komma
getrennt.
Die Eigenschaften werden gemäß der in den Beschreibungsspezifikationen
aufgeführten Reihenfolgen sortiert.
JSON-Arrays werden mit einer 4 begonnen und mit einer 6 beendet.
Elemente dieser JSON-Arrays werden mit einer 5 statt mit einem Komma getrennt.
In JSON-Arrays von Beschreibungen kommen entweder nur Beschreibungen anderer
Objekte oder nur Integer vor.
Diese werden aufsteigend nach ihren IDs bzw. aufsteigend nach ihrem Wert
sortiert.

Alle JSON-Objekteigenschaften werden ohne Schlüssel, Doppelpunkte, geschwungene
Klammern und zusätzliche Leerzeichen oder Zeilenumbrüche konkateniert.
Alle JSON-Arrayelemente werden ohne eckige Klammern und zusätzliche
Leerzeichen oder Zeilenumbrüche konkateniert.
Die doppelten Anführungszeichen um Zeichenketten werden entfernt.
Da alle sechs Trennzeichen nur als Escape-Sequenzen, nicht aber direkt im
JSON-Format vorkommen können\footnote{Vgl. \enquote{RFC 7159}, Kapitel 7
\enquote{Strings}, \url{https://tools.ietf.org/html/rfc7159}} ist diese
Darstellung eindeutig.

% TODO: Update example
\par Im folgenden beispielhaft eine Verabredung mit folgenden Details:
\begin{itemize}
    \item Besitzt die ID 12
    \item Trägt den Titel \enquote{Beispielverabredung 1010}
    \item Erstellt vom Benutzerkonto mit ID 200
    \item Beginnt am 01. Januar 2018 um 00:00:00.000
    \item Endet am 03. Januar 2018 um 18:30:00.000
    \item Findet auf der Position 49.011978 N, 8.416377 E statt
    \item Die Benutzerkonten mit den IDs 200, 3 und 43378 sind als Teilnehmer
        eingetragen
\end{itemize}

Eine mögliche Kodierung der ausführlichen Beschreibung im JSON-Format kann wie
folgt aussehen:
\begin{lstlisting}[language=json,firstnumber=1]
{   "type" : "event",
    "id": 2,
    "title": "Beispielverabredung 1010",
    "creator": 200,
    "time-start": 1514761200000,
    "time-end": 1515000600000,
    "latitude": 49.011978,
    "longitude": 8.416377,
    "participants": [ 3, 200, 43378 ]
}
\end{lstlisting}

\newcommand*\circled[1]{\tikz[baseline=(char.base)]{
            \node[shape=circle,draw,inner sep=1pt] (char) {\scriptsize #1};}}

Die normalisierte Darstellung dieser Verabredung sieht wie folgt aus.
Die Steuerzeichen werden durch \circled{1}, \circled{2}, \circled{3} etc.
dargestellt.\\
\textit{Hinweis: Die Zeilenumbrüche dienen nur der besseren Lesbarkeit und sind
nicht in der normalisierten Darstellung enthalten.}

\begin{lstlisting}[language=norm,firstnumber=1]
(*\circled{1}*)"event"(*\circled{2}*)2(*\circled{2}%
*)"Beispielverabredung 1010"(*\circled{2}*)200(*\circled{2}*)1514761200000(*%
\circled{2}*)
1515000600000(*\circled{2}*)49.011978(*\circled{2}*)8.416377(*\circled{2}%
\circled{4}*)3(*\circled{5}*)200(*\circled{5}*)43378(*\circled{6}\circled{3}*)
\end{lstlisting}

\par Ein beispielhafter Algorithmus für die Normalisierung ist im folgenden
als Python-Implementierung angegeben:

\pythonexternal{scripts/normalize.py}

\section{Abrufen von Aktualisierungen}\label{sec:get-updates}
Zum Abrufen von Aktualisierungen gibt es mehrere Möglichkeiten.

\begin{itemize}
    \item Eine einzelne Anfrage an den Server, die alle aktuell für das eigene
        Benutzerkonto vorliegenden Änderungen abruft
    \item Eine Anfrage an den Server, die die aktuelle Verbindung als
        \enquote{Aktualisierungsverbindung} kennzeichnet.
        Auf einer Aktualisierungsverbindung werden alle Befehle des Klienten
        ignoriert.
        Erhält der Server für ein Benutzerkonto, welches eine offene
        Aktualisierungsverbindung zum Server hat, eine Aktualisierung, wird
        diese sofort über die Aktualisierungsverbindung an den Klienten
        gesendet.
    \item Push-Notifications.
        Diese Möglichkeit der Übertragung ist auf mehreren
        Smartphone-Betriebssystemen vertreten und erlaubt sehr
        ressourceneffiziente Mitteilungen an den Klienten.
        Für das verwendete Framework gibt es neben Google's Firebase noch
        wenige weitere quelloffene Kandidaten.
        Die Umsetzung dieser Übertragungsmöglichkeit wird sich erst während
        der Implementierung zeigen.
\end{itemize}


\end{document}
