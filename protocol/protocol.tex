\documentclass[parskip=full,11pt]{scrartcl}
\usepackage[utf8]{inputenc}
\usepackage[T1]{fontenc}
\usepackage[english]{babel}
\usepackage[useregional]{datetime2}
\usepackage[pdfborderstyle={/S/U/W 0}]{hyperref}
\usepackage[nameinlink]{cleveref}
\usepackage[section]{placeins}
\usepackage[top=2.5cm, bottom=2.5cm, left=4cm, right=3cm]{geometry}
\usepackage{xcolor}
\usepackage{graphicx}
\usepackage{csquotes}
\usepackage{amsmath} % for $\text{}$
\usepackage{protocol} % local .sty file
\usepackage{enumitem}
\usepackage{algorithm}
\usepackage{algorithmicx}
\usepackage{algpseudocode}

\setlist{nosep}

\newcommand\urlpart[2]{$\underbrace{\text{\texttt{#1}}}{\text{#2}}$}
\raggedbottom
\crefname{figure}{Abb}{Abb}

\newcommand\producttitle{treff.}
\newcommand\protocolversion{0.3}
\hypersetup{
	pdftitle={\producttitle~- Protocol Specification v\protocolversion},
	bookmarks=true,
}

% section numbers in margins:
\renewcommand\sectionlinesformat[4]{%
    \ifstr{#1}{subsubsection}{%
        \makebox[0pt][r]{#4}%
    }{%
        \makebox[0pt][r]{#3}#4%
    }%
}%

% header & footer
\usepackage{scrlayer-scrpage}
%\lofoot{\today}
%\refoot{\today}
\pagestyle{scrheadings}
%\let\raggedsection\centering

\title{\includegraphics[width = 80mm]{images/logo_crop.png}}
\subtitle{\huge Protocol Specification\\
          v\protocolversion}
\author{Lukas Dippon
        \and Jens Kienle
        \and Matthias Noll
        \and Fabian Röpke
        \and Tim Schmidt
        \and Simon Vögele}

\begin{document}

\maketitle
\thispagestyle{empty} % removed page number from title

\pagebreak
\tableofcontents

%%%%%%%%%%%%%%%%%%%
\pagebreak
\section{Network protocol and command format}
% Zusammenspiel von Klient und Server
% (Fließtext und Sequenzdiagram, ggf UML-Klassendiagram)
% TODO: JSON verlinken / footnote
% TODO: Alphabet und Escapes genau festlegen
Client and server communicate via a TLS-encrypted TCP connection.
This implies that a server must possess a valid SSL certificate.

The commands specified below are encoded as a single JSON object each and are
sent from client to server.
If not specified otherwise, each request produces a single response from the
server, also encoded as a JSON object.
A request is considered to be completely transferred to the server once the
server receives the closing curly bracket for the top-level JSON object.

The \textbf{cmd} property of the top-level JSON object of a request defines
which command the client is trying to execute.
As such, the \textbf{cmd} property must always be included in a request.
Parameters, return values and error codes are also encoded as properties of
the top-level JSON object of a request or response.

The names of the commands specified below are to be used as the value of the
\textbf{cmd} property.
These are to be encoded as strings.

If an error occurs during command execution, possibly caused by invalid
parameters, the response JSON object consists of only the single integer
property \textbf{error}.
The meaning of these error codes is specified in chapter \ref{sec:errorcodes}.

\section{Data types and restrictions}
While JSON itself defines several data types, the property
definitions in the following chapters use the conventional, more restrictive
variants as used, for example, in the Java programming language.
As such, mentions of \code{int} and \code{long} refer to the signed variants
within the upper and lower bounds defined in the Java programming language.

Strings may contain any Unicode symbol as long as they are properly escaped to
match the restriction set by JSON specifications.

The string properties defined in chapter \ref{sec:descriptions} might impose
additional restrictions, such as maximum amount of characters, i.e. Unicode
symbols, or a reduced set of symbols, e.g. only alphanumeric characters.
In this document, the set of alphanumeric characters is defined to consist
of arabic numbers and lower-case and upper-case letters of the ISO basic Latin
alphabet (a-z, A-Z, 0-9).

Properties of type \code{date} are to be represented as Unix time%
\footnote{\enquote{Unix time}, Wikipedia, Version 822615863,
\url{https://en.wikipedia.org/w/index.php?title=Unix_time&oldid=822615863}},
and encoded as \code{long}s.

Properties of type \code{decimal degree} must be valid decimal degrees
representing either latitude or longitude and are to be encoded as
\code{double}s.

While JSON supports arrays of different data types, this protocol does not.
All arrays defined here consist of elements of the same type.

\section{Object descriptions}\label{sec:descriptions}
In order to transmit information about domain model objects via JSON encoded
requests and reponses, the encoding of the objects into \enquote{descriptions}
is specified below.

Each object can be encoded as a shallow, complete, or recursively
complete description.
While complete descriptions contain all properties of the object itself and
may be requested from the server via the respective \textbf{get-*-details}
commands, recursively complete descriptions contain the complete descriptions
of certain types of referenced objects, e.g. a recursively complete description
of a usergroup contains recursively complete descriptions of all events and
polls.

Recursively complete descriptions are not transferred in any of the
commands specified in chapter \ref{sec:commands}.
Instead, they are used to generate the checksums contained in shallow
descriptions (see chapter \ref{sec:checksum}).

The commands defined in chapter \ref{sec:commands} might not use all of the
descriptions specified below.
For the sake of consistency, they are listed here anyways.

\apiDesc{Usergroup}{
    name/string/Name of the group,
    members/membership array/Complete descriptions of all memberships that are
    part of this group,
    events/int array/IDs of all events that are part of this group,
    polls/int array/IDs of all polls that are part of this group
}{
    events/object array/Complete descriptions of all events,
    polls/object array/Complete descriptions of all polls
}{
    name/Up to 64 characters
}

\apiDesc{Account}{
    username/string/Username of the account
}{}{
    username/Alphanumeric{,} up to 32 characters
}

\apiDesc{Event}{
    title/string/Title of the event,
    creator/int/ID of the account that created this event,
    time-start/date/Point in time at which the event starts,
    time-end/date/Point in time at which the event ends,
    latitude/decimal degree/Latitude of the position that the event is centered
    on,
    longitude/decimal degree/Longitude of the position that the event is
    centered on,
    participants/int array/IDs of all group members that are currently
    participating or are planning to participate in the event
}{}{
    title/Up to 64 characters
}

\apiDesc{Poll}{
    question/string/Question for which answered are supplied as poll options,
    creator/int/ID of the account that created this poll,
    multi-choice/boolean/True iff an account can vote for multiple poll options
    simultaneously,
    options/object array/Complete descriptions of the poll options that are
    part of this poll
}{}{
    question/Up to 64 characters
}

\apiDesc{Poll option}{
    latitude/decimal degree/Latitude of the suggested position,
    longitude/decimal degree/Longitude of the suggested position,
    time-start/date/Suggested point in time at which the resulting event should
    start,
    time-end/date/Suggested point in time at which the resulting event should
    end,
    supporters/int array/IDs of all account that are currently voting for this
    poll option
}{}{}

\apiDesc{Membership}{
    account-id/int/ID of the account that this membership belongs to,
    permissions/object/Object holding a \code{permission-name -> boolean} pair
    for each permission defined in chapter \ref{sec:permissions}.
}{}{}

\apiDesc{Update}{
    TODO/TODO/TODO
}{}{}

\section{Commands}\label{sec:commands}

\apiErrorNew{username-already-in-use}{000}{Username already in use}
\apiErrorNew{invalid-email}{010}{Email invalid}
\apiErrorNew{token-invalid}{020}{Authentication token invalid}
\apiErrorNew{cred-wrong}{030}{Unknown username/password combination}
\apiErrorNew{reset-code-invalid}{040}{Password reset code invalid}

\apiErrorNew{no-permission-group-edit}{100}{Das Benutzerkonto besitzt nicht
die nötigen Rechte, um die Gruppe zu editieren}
\apiErrorNew{no-permission-members-edit}{101}{Das Benutzerkonto besitzt nicht
die nötigen Rechte, um die Liste an Gruppenmitgliedern zu editieren}
\apiErrorNew{no-permission-event-create}{110}{Das Benutzerkonto besitzt nicht
die nötigen Rechte, um in der angegebenen Gruppe Verabredungen zu erstellen}
\apiErrorNew{no-permission-event-edit}{111}{Das Benutzerkonto besitzt nicht
die nötigen Rechte, um die Verabredung zu editieren}
\apiErrorNew{no-permission-poll-create}{120}{Das Benutzerkonto besitzt nicht
die nötigen Rechte, um in der angegebenen Gruppe Abstimmung zu erstellen}
\apiErrorNew{no-permission-poll-edit}{121}{Das Benutzerkonto besitzt nicht
die nötigen Rechte, um die Abstimmung zu editieren}
\apiErrorNew{no-permission-permissions-edit}{122}{Das Benutzerkonto besitzt
nicht die nötigen Rechte, um die Berechtigungen zu editieren}

\apiErrorNew{username-invalid}{200}{Nutzername ungültig}
\apiErrorNew{user-id-invalid}{201}{Mindestens eine
Benutzerkonto-Identifikationsnummer ungültig}
\apiErrorNew{group-id-invalid}{210}{Gruppen-Identifikationsnummer ungültig
oder Benutzerkonto nicht Teil der Gruppe}
\apiErrorNew{event-id-invalid}{220}{Verabredungs-Identifikationsnummer ungültig}
\apiErrorNew{poll-id-invalid}{230}{Abstimmungs-Identifikationsnummer ungültig}
\apiErrorNew{poll-option-id-invalid}{240}
{Abstimmungsoptions-Identifikationsnummer ungültig}

\apiErrorNew{time-end-in-past}{300}{Mindestens ein Endzeitpunkt liegt
in der Vergangenheit}
\apiErrorNew{time-end-start-conflict}{301}{Mindestens ein Endzeitpunkt liegt
nicht nach dem dazugehörigen Startzeitpunkt}
\apiErrorNew{time-measured-future}{310}{Messzeitpunkt liegt zu weit in der
Zukunft}

\apiErrorNew{not-in-contacts}{400}{Benutzerkonto nicht in
der eigenen Kontaktliste}
\apiErrorNew{being-blocked}{401}{Benutzerkonto wird vom angegeben
Benutzerkonto blockiert}
\apiErrorNew{blocking-already}{402}{Benutzerkonto wird durch
das eigene Benutzerkonto blockiert}
\apiErrorNew{not-blocking}{403}{Benutzerkonto wird nicht durch
das eigene Benutzerkonto blockiert}
\apiErrorNew{no-user-invited}{410}{Kein Benutzerkonto eingeladen,
das nicht das eigene ist}
\apiErrorNew{user-already-in-group}{411}{Mindestens eine Identifikationsnummer
gehört zu einem Benutzerkonto, welches bereits Teil der Gruppe ist}
\apiErrorNew{user-not-in-group}{412}{Mindestens eine Identifikationsnummer
gehört zu einem Benutzerkonto, welches nicht Teil der Gruppe ist}
\apiErrorNew{already-participating-event}{420}{Das Benutzerkonto ist bereits
Teil dieser Verabredung}
\apiErrorNew{not-participating-event}{421}{Das Benutzerkonto ist kein Teil
dieser Verabredung}
\apiErrorNew{already-voting-for-option}{430}{Das Benutzerkonto hat bereits
eine Stimme für diese Option abgegeben}
\apiErrorNew{poll-not-multichoice}{431}{Es ist keine Mehrfachauswahl für
diese Abstimmung möglich}
\apiErrorNew{not-voting-for-option}{432}{Das Benutzerkonto hat keine Stimme
für dies Option abgegeben}

All commands except for \textbf{login}, \textbf{register},
\textbf{reset-password}, and \textbf{reset-password-confirm} require the
following parameter:\\
\apiArgument{token/String/Authentication token{,} produced by \textbf{login} or
\textbf{register}}
\par Transmission of an invalid token will produced the following error:\\
\apiErrorPrint{token-invalid}

\apiCommand{register}
{Registers a new user account.
Declaring an e-mail adress is optional.
It may be set at a later time via the \textbf{edit-emai} command.}
{user/string/Username of the account,
pass/string/Password of the account}
{token/string/Authentication token}
{username-already-in-use}
{}

\apiCommand{login}
{Produces an authentication token for the specified user account if the
password is correct.}
{user/string/Username of the account,
pass/string/Password of the account}
{token/string/Authentication token}
{cred-wrong}
{}

\apiCommand{request-updates}
{Requests the list of undelivered updates from the server.}
{}
{updates/object array/All undelivered updates for the calling account{,} sorted
from oldest to youngest (see chapter \ref{updatedesc}).}
{user-id-invalid}
{}

\apiCommand{edit-username}
{Changes own username.}
{username/string/New username}
{}
{username-already-in-use}
{account-change}

\apiCommand{edit-email}
{Editieren der eigenen E-Mail-Adresse}
{pass/String/Passwort des eigenen Kontos,
email/String/vorgeschlagene{,} neue E-Mail}
{---//Leeres JSON-Objekt bei Erfolg}
{cred-wrong, user-id-invalid, invalid-email}
{}

\apiCommand{edit-password}
{Editieren des eigenen Passworts}
{pass/String/aktuelles Passwort des eigenen Kontos,
newpass/String/vorgeschlagenes{,} neues Passwort}
{---//Leeres JSON-Objekt bei Erfolg}
{cred-wrong, user-id-invalid}
{}

\apiCommand{reset-password}
{Sendet eine Zurücksetzungsanfrage für ein Passwort.
Falls die angegebene E-Mail-Adresse einem Benutzerkonto zugewiesen ist, wird
eine E-Mail mit einem Zeichencode an die E-Mail-Adresse gesandt, der mit dem
\textbf{reset-password-confirm}-Befehl genutzt werden kann, um das Passwort des
entsprechenden Benutzerkontos zurückzusetzen.
Ist die E-Mail-Adresse keinem Benutzerkonto zugewiesen, wird \textbf{kein}
Fehler ausgegeben.
Dies dient dem Schutz von E-Mail-Adressen vor Brute-Force-Angriffen.}
{email/E-Mail-Adresse des Benutzerkontos{,} dessen Passwort zurückgesetzt werden
soll}
{---/Leeres JSON-Objekt bei Erfolg}
{}
{}

\apiCommand{reset-password-confirm}
{Führt die Passwortzurücksetzung durch.
Der verwendete Bestätigungscode ist nach Ausführung ungültig und kann
verworfen werden.}
{code/Bestätigungscode{,} der nach Ausführung des \textbf{reset-password}-Befehls
an die dort angegebene E-Mail-Adresse gesandt wurde,
pass/Neues Passwort}
{---/Leeres JSON-Objekt bei Erfolg}
{reset-code-invalid}
{}

\apiCommand{delete-account}
{Löschen des eigenen Kontos. Dies führt wie erwartet dazu{,}
dass der Nutzer automatisch ausgeloggt wird und dadurch seinen
Authentifizierungstoken und somit den weiteren Zugriff auf die Funktionalitäten
der App verliert.}
{pass/String/Passwort des eigenen Kontos}
{---//Leeres JSON-Objekt bei Erfolg}
{cred-wrong}
{}

\apiCommand{get-user-id}
{Abrufen der Idenfikationsnummer eines anderen Benutzerkontos durch Übergeben
des Benutzernamens}
{user/Benutzername des Benutzerkontos}
{\dots/Eine Beschreibung eines Benutzerkontos{,} das zu dem angegebenen
Benutzernamen gehört (siehe \ref{sec:accountdesc})}
{username-invalid}
{}

\apiCommand{add-contact}
{Hinzufügen eines anderen Benutzerkontos zur eigenen Kontaktliste.
Fügt ebenfalls das eigene Benutzerkonto zur Kontaktliste des anderen
Benutzerkontos hinzu.
Dies schlägt fehl, wenn eines der zwei Benutzerkonten das andere blockiert.}
{id/Identifikationsnummer des anderen Benutzerkontos}
{---/Leeres JSON-Objekt bei Erfolg}
{user-id-invalid,being-blocked,blocking-already}
{}

\apiCommand{remove-contact}
{Entfernen eines anderen Benutzerkontos von der eigenen Kontaktliste.
Entfernt ebenfalls das eigene Benutzerkonto von der Kontaktliste des anderen.}
{id/Identifikationsnummer des anderen Benutzerkontos}
{---/Leeres JSON-Objekt bei Erfolg}
{user-id-invalid,not-in-contacts}
{}

\apiCommand{list-contacts}
{Auflisten aller Kontakte des Benutzerkonto}
{---/Keine weiteren Parameter}
{contacts/Ein Array von Indetifikationsnummern anderer Benutzerkonten}
{}
{user-id-invalid}

\apiCommand{block-account}
{Blockiert das angegebene Benutzerkonto.
Bis dies aufgehoben wird, können das eigene und das angegebene
Benutzerkonto sich nicht gegenseitig zur ihrer Kontaktliste hinzufügen.
Sind die beiden Benutzerkonten bereits Kontakte, werden diese vorher aus der
Kontaktliste des jeweils anderen Benutzerkontos entfernt.}
{id/Identifikationsnummer des anderen Benutzerkontos}
{---/Leeres JSON-Objekt bei Erfolg}
{user-id-invalid,blocking-already}
{}

\apiCommand{unblock-account}
{Hebt die Blockierung des angegebenen Benutzerkontos durch das eigene
Benutzerkonto auf.}
{id/Identifikationsnummer des anderen Benutzerkontos}
{---/Leeres JSON-Objekt bei Erfolg}
{user-id-invalid,not-blocking}
{}

\apiCommand{create-group}
{Erstellen einer Gruppe}
{name/Name der Gruppe,
members/Array an Identifikationsnummern aller Benutzerkonten{,} die nach
Erstellung der Gruppe dieser hinzugefügt werden sollen.
Die Identifikationsnummer des eigenen Benutzerkontos muss nicht angegeben
werden.
Falls dennoch vorhanden{,} wird diese ignoriert.
Das Array muss mindestens eine gültige Identifikationsnummer beinhalten{,} die
nicht die eigene ist.}
{id/Eindeutige Identifikationsnummer der Gruppe{,} mit der diese in
weiteren Befehlen referenziert werden kann}
{user-id-invalid,no-user-invited}
{}

\apiCommand{edit-group-name}
{Editieren des Namens einer Gruppe, in der das Benutzerkonto Mitglied ist}
{id/Eindeutige Identifikationsnummer der Gruppe,
name/Neuer name der Gruppe}
{---/Leeres JSON-Objekt bei Erfolg}
{group-id-invalid,no-permission-group-edit}
{}

\apiCommand{add-group-members}
{Hinzufügen eines oder mehrerer Benutzerkonten zu einer Gruppe, in der das
Benutzerkonto Mitglied ist}
{id/Eindeutige Identifikationsnummer der Gruppe,
members/Array an Identifikationsnummern aller Benutzerkonten{,} die der Gruppe
hinzugefügt werden sollen.
Es dürfen keine Identifikationsnummern von Benutzerkonten angegeben werden{,}
die bereits Teil der Gruppe ist.
Insbesondere darf nicht die Identifikationsnummer des eigenen Benutzerkontos
angegeben werden.}
{---/Leeres JSON-Objekt bei Erfolg}
{group-id-invalid,user-id-invalid,no-permission-members-edit,%
user-already-in-group}
{}

\apiCommand{remove-group-members}
{Entfernen eines oder mehrerer Benutzerkonten von einer Gruppe, in der das
Benutzerkonto Mitglied ist}
{id/Eindeutige Identifikationsnummer der Gruppe,
members/Array an Identifikationsnummern aller Benutzerkonten{,} die aus der
Gruppe entfernt werden sollen.
Es dürfen keine Identifikationsnummern von Benutzerkonten angegeben werden{,}
die nicht Teil der Gruppe sind.
Es darf die Identifikationsnummer des eigenen Benutzerkontos angegeben werden.
Dies führt wie erwartet dazu{,} dass das eigene Benutzerkonto den weiteren
Zugriff auf die Gruppe verliert.
Werden durch diesen Befehl alle Mitglieder einer Gruppe entfernt{,} wird die
Gruppe gelöscht.}
{---/Leeres JSON-Objekt bei Erfolg}
{group-id-invalid,user-id-invalid,no-permission-members-edit,user-not-in-group}
{}

\apiCommand{get-permissions}
{Abfragen der Rechte eines Benutzerkontos in einer Gruppe}
{id/Eindeutige Identifikationsnummer der Gruppe,
userid/Eindeutige Identifikationsnummer des Benutzerkontos}
{permissions/Eine Rechtebeschreibung (Siehe Kapitel \ref{sec:permdesc})}
{group-id-invalid}
{}

\apiCommand{edit-permissions}
{Ändern der Rechte eines Benutzerkontos in einer Gruppe}
{id/Eindeutige Identifikationsnummer der Gruppe,
userid/Eindeutige Identifikationsnummer des Benutzerkontos,
permissions/Eine partielle Rechtebeschreibung{,} die alle zu überschreibenden
Wertepaare enthält (Siehe Kapitel \ref{sec:permdesc})}
{---/Leeres JSON-Objekt bei Erfolg}
{group-id-invalid, no-permission-permissions-edit}
{}

\apiCommand{create-event}
{Erstellen einer Verabredung in einer Gruppe, in der das Benutzerkonto Mitglied
ist}
{group-id/Identifikationsnummer der Gruppe{,} in der die Verabredung erstellt
werden soll,
event/Complete description of the event but without ID and participants array}
{id/Eindeutige Identifikationsnummer der Verabredung{,} mit der diese in
weiteren Befehlen referenziert werden kann}
{group-id-invalid,time-end-in-past,time-end-start-conflict,
no-permission-event-create}
{}

\apiCommand{edit-event}
{Editieren einer Verabredung in einer Gruppe, in der das Benutzerkonto Mitglied
ist}
{group-id/Identifikationsnummer der Gruppe{,} in der die Verabredung erstellt
wurde,
event/New complete description of the event without the participants array}
{---/Leeres JSON-Objekt bei Erfolg}
{group-id-invalid,time-end-in-past,time-end-start-conflict,event-id-invalid,
no-permission-event-edit}
{}

\apiCommand{join-event}
{Beitreten einer Verabredung einer Gruppe, in der das Benutzerkonto Mitglied
ist}
{group-id/Identifikationsnummer der Gruppe{,} in der die Verabredung erstellt
wurde,
id/Identifikationsnummer der Verabredung}
{---/Leeres JSON-Objekt bei Erfolg}
{group-id-invalid,event-id-invalid,already-participating-event}
{}

\apiCommand{leave-event}
{Verlassen einer Verabredung, an welcher das Benutzerkonto teilnimmt}
{group-id/Identifikationsnummer der Gruppe{,} in der die Verabredung erstellt
wurde,
id/Identifikationsnummer der Verabredung}
{---/Leeres JSON-Objekt bei Erfolg}
{group-id-invalid,event-id-invalid,not-participating-event}
{}

\apiCommand{remove-event}
{Entfernen einer Verabredung aus einer Gruppe, in der das Benutzerkonto
Mitglied ist}
{group-id/Identifikationsnummer der Gruppe{,} in der die Verabredung erstellt
wurde,
id/Identifikationsnummer der Verabredung}
{---/Leeres JSON-Objekt bei Erfolg}
{group-id-invalid,event-id-invalid,no-permission-event-edit}
{}

\apiCommand{create-poll}
{Erstellen einer Abstimmung in einer Gruppe, in der das Benutzerkonto Mitglied
ist}
{group-id/Identifikationsnummer der Gruppe{,} in der die Abstimmung erstellt
werden soll,
question/Frage{,} über dessen Antwort abgestimmt wird,
multi-choice/Boolscher Wert{,} der angibt{,} ob ein Nutzer für mehrere
Optionen stimmen kann,
options/Array an Beschreibungen von Abstimmungsoptionen{,} die nach der
Erstellung der Abstimmung hinzugefügt werden sollen}
{id/Eindeutige Identifikationsnummer der Abstimmung{,} mit der diese in
weiteren Befehlen referenziert werden kann,
option-ids/Array an Identifikationsnummern der Abstimmungsoptionen. Die
Reihenfolge dieser IDs entspricht genau der Reihenfolge der angegebenen
Abstimmungsoptionen.}
{group-id-invalid,time-end-in-past,time-end-start-conflict,
no-permission-poll-create}
{}

\apiCommand{edit-poll}
{Editieren einer Abstimmung in einer Gruppe, in der das Benutzerkonto Mitglied
ist}
{group-id/Identifikationsnummer der Gruppe{,} in der die Verabredung erstellt
wurde,
id/Identifikationsnummer der Abstimmung,
\dots/Alle Eigenschaften{,} mit Ausnahme des Abstimmungsoptionsarrays und der
Identifikationsnummer{,} die überschrieben werden sollen.
Die Schlüssel entsprechen denen einer ausführlichen Abstimmungsbeschreibung.}
{---/Leeres JSON-Objekt bei Erfolg}
{group-id-invalid,poll-id-invalid,no-permission-poll-edit}
{}

\apiCommand{add-poll-option}
{Hinzufügen einer Abstimmungsoption in einer Gruppe, in der das Benutzerkonto
Mitglied ist}
{group-id/Identifikationsnummer der Gruppe{,} in der die Abstimmung erstellt
wurde,
poll-id/Identifikationsnummer der Abstimmung{,} zu der die Abstimmungsoption
hinzugefügt werden soll,
latitude/Breitengrad der vorgeschlagenen Position (Angabe als Dezimalgrad),
longitude/Längengrad der vorgeschlagenen Position (Angabe als Dezimalgrad),
time-start/Vorgeschlagener Startzeitpunkt (Angabe in Unixzeit),
time-end/Vorgeschlagener Endzeitpunkt (Angabe in Unixzeit)}
{id/Eindeutige Identifikationsnummer der Abstimmungsoption{,} mit der diese in
weiteren Befehlen referenziert werden kann}
{group-id-invalid,poll-id-invalid,time-end-in-past,time-end-start-conflict,
no-permission-poll-edit}
{}

% TODO: „Wählerarray“ besser formulieren
\apiCommand{edit-poll-option}
{Editieren einer Abstimmungsoption in einer Gruppe, in der das Benutzerkonto
Mitglied ist}
{group-id/Identifikationsnummer der Gruppe{,} in der die Abstimmung erstellt
wurde,
poll-id/Identifikationsnummer der Abstimmung,
id/Identifikationsnummer der Abstimmungsoption,
\dots/Alle Eigenschaften{,} mit Ausnahme des Wählerarrays und der
Identifikationsnummer{,} die überschrieben werden sollen.
Die Schlüssel entsprechen denen einer ausführlichen Abstimmungsbeschreibung.}
{---/Leeres JSON-Objekt bei Erfolg}
{group-id-invalid,poll-id-invalid,poll-option-id-invalid,no-permission-poll-edit}
{}

\apiCommand{vote-for-option}
{Stimmen für eine Abstimmungsoption in einer Abstimmung einer Gruppe,
in der das Benutzerkonto Mitglied ist}
{group-id/Identifikationsnummer der Gruppe{,} in der die Abstimmung erstellt
wurde,
poll-id/Identifikationsnummer der Abstimmung,
id/Identifikationsnummer der Abstimmungsoption}
{---/Leeres JSON-Objekt bei Erfolg}
{group-id-invalid,poll-id-invalid,poll-option-id-invalid,poll-not-multichoice,
already-voting-for-option}
{}

\apiCommand{withdraw-vote-for-option}
{Stimme zurückziehen für eine Abstimmungsoption in einer Abstimmung
einer Gruppe, in der das Benutzerkonto Mitglied ist}
{group-id/Identifikationsnummer der Gruppe{,} in der die Abstimmung erstellt
wurde,
poll-id/Identifikationsnummer der Abstimmung,
id/Identifikationsnummer der Abstimmungsoption}
{---/Leeres JSON-Objekt bei Erfolg}
{group-id-invalid,poll-id-invalid,poll-option-id-invalid,not-voting-for-option}
{}

\apiCommand{remove-poll-option}
{Entfernen einer Abstimmungsoption in einer Gruppe, in der das Benutzerkonto
Mitglied ist}
{group-id/Identifikationsnummer der Gruppe{,} in der die Abstimmung erstellt
wurde,
poll-id/Identifikationsnummer der Abstimmung,
id/Identifikationsnummer der Abstimmungsoption}
{---/Leeres JSON-Objekt bei Erfolg}
{group-id-invalid,poll-id-invalid,group-id-invalid,poll-id-invalid,
poll-option-id-invalid,no-permission-poll-edit}
{}

\apiCommand{remove-poll}
{Entfernen einer Abstimmung aus einer Gruppe, in der das Benutzerkonto Mitglied
ist}
{group-id/Identifikationsnummer der Gruppe{,} in der die Verabredung erstellt
wurde,
id/Identifikationsnummer der Abstimmung}
{---/Leeres JSON-Objekt bei Erfolg}
{group-id-invalid,poll-id-invalid,no-permission-poll-edit}
{}

\apiCommand{send-chat-message}
{Senden einer Textnachricht an eine Gruppe, in der das Benutzerkonto Mitglied
ist}
{group-id/Identifikationsnummer der Gruppe,
message/Textnachricht} % TODO: Länge beschränken (vllt. auch Gruppengröße etc)
{---/Leeres JSON-Objekt bei Erfolg}
{group-id-invalid}
{}

\apiCommand{list-groups}
{Auflisten der Gruppen, in denen das Benutzerkonto Mitglied ist}
{---/Keine weiteren Parameter}
{groups/Ein Array an oberflächlich beschriebenen Gruppen (siehe
Kapitel \ref{sec:groupdesc})}
{}
{}

\apiCommand{get-group-details}
{Abrufen von Details einer Gruppe, in denen das Benutzerkonto Mitglied ist}
{id/Identifikationsnummer der Gruppe}
{\dots/Eine ausführliche Beschreibung der Gruppe{,} die zu der angegebenen
Identifikationsnummer gehört (siehe \ref{sec:groupdesc})}
{group-id-invalid}
{}

\apiCommand{get-user-details}
{Abrufen von Details eines Benutzerkontos}
{id/Identifikationsnummer des Benutzerkontos}
{\dots/Eine Beschreibung eines Benutzerkontos{,} das zu der angegebenen
ID gehört (siehe \ref{sec:accountdesc})}
{user-id-invalid}
{}

% TODO: Definiere „aktive Verabredung“
\apiCommand{get-event-details}
{Abrufen von Details einer aktiven Verabredung, die in einer Gruppe erstellt
wurde, in der das Benutzerkonto Mitglied ist}
{id/Identifikationsnummer der Verabredung,
group-id/Identifikationsnummer der Gruppe{,} in der die Verabredung erstellt
wurde}
{\dots/Eine ausführliche Beschreibung der Verabredung{,} die zu der angegebenen
Identifikationsnummer gehört (siehe \ref{sec:eventdesc})}
{group-id-invalid,event-id-invalid}
{}

\apiCommand{get-poll-details}
{Abrufen von Details einer aktiven Abstimmung, die in einer Gruppe erstellt
wurde, in der das Benutzerkonto Mitglied ist}
{id/Identifikationsnummer der Abstimmung,
group-id/Identifikationsnummer der Gruppe{,} in der die Abstimmung erstellt
wurde}
{\dots/Eine ausführliche Beschreibung der Abstimmung{,} die zu der angegebenen
Identifikationsnummer gehört (siehe \ref{sec:polldesc})}
{group-id-invalid,poll-id-invalid}
{}

% TODO: Zeit-Toleranz / Annahmen über Synchronität weiter oben spezifizieren
\apiCommand{request-position}
{Sendet eine Standortanfrage an eine Gruppe}
{id/Eindeutige Identifikationsnummer der Gruppe,
time/ Endzeit der Standortübertragung}
{---/Leeres JSON-Objekt bei Erfolg}
{group-id-invalid}
{}

% TODO: Zeit-Toleranz / Annahmen über Synchronität weiter oben spezifizieren
\apiCommand{update-position}
{Aktualisiert die Position des Benutzerkontos.
Der Messzeitpunkt darf aus Sicht der Servers maximal 30 Sekunden in der Zukunft
liegen.}
{latitude/Breitengrad der vorgeschlagenen Position (Angabe als Dezimalgrad),
longitude/Längengrad der vorgeschlagenen Position (Angabe als Dezimalgrad),
time-measured/Zeitpunkt{,} zu dem die Position gemessen wurde (Angabe in
Unixzeit)}
{---/Leeres JSON-Objekt bei Erfolg}
{time-measured-future}
{}

\section{Error codes}\label{sec:errorcodes}
\apiErrorTable

\section{Updates}
To inform users of changes that affect them, update objects are created with
each successfully executed command that makes changes to domain model objects.

These update objects are divided in several types. All of them hold at least
the following properties:

\apiArgument{
    type/string/Type of the update object{,} see list below,
    time-created/date/Point in time at which this update was created,
    creator/int/ID of the account that caused the creation of this update
}

\updateType{account-change}{
    \apiArgument{
        account/object/Complete description of the account that changed{,}
        omitting all properties that didn't change.
    }
}

\updateType{chat}{
    \apiArgument{
        group-id/int/ID of the group to which the chat message was sent,
        message/string/Content of the message
    }
}

\section{Prüfsummen für ausführliche Beschreibungen}\label{sec:checksum}
\newcommand{\hashAlg}{MD5}
Um unnötige ausführliche Übertragungen zu vermeiden beinhalten oberflächliche
Beschreibungen Prüfsummen der zum selben Objekt gehörenden ausführlichen
Beschreibung.
Der Klient ist somit in der Lage, festzustellen, ob seine lokalen Datensätze
mit denen des Servers übereinstimmen, ohne die Datensätze erneut anfordern zu
müssen.

\par Zur Berechnung der Prüfsummen wird der \hashAlg-Algorithmus verwendet.

\par Als Grundlage der Eingabe an den \hashAlg-Algorithmus wird die
ausführliche Beschreibung des jeweiligen Objektes im gleichen Format wie in den
vorangegangenen Kapiteln verwendet.

\par Da es aufgrund von Umordnungen, Leerzeichen, Zeilenumbrüchen etc. mehrere
unterschiedliche, aber äquivalente JSON-Darstellungen für eine ausführliche
Beschreibung gibt, müssen diese vor Eingabe an den \hashAlg-Algorithmus in eine
normalisierte Form gebracht werden:

\par \textit{Hinweis: In dem folgenden Absatz bezeichnen Ziffern nicht etwa die
ASCII-Ziffern \enquote{0} bis \enquote{9} mit den Werten 48 bis 57, sondern die
nicht darstellbaren Steuerzeichen mit den Werten 0 bis 9.}\\
JSON-Objekte werden mit einer 1 begonnen und mit einer 3 beendet.
Eigenschaften dieser JSON-Objekte werden mit einer 2 statt einem Komma
getrennt.
Die Eigenschaften werden gemäß der in den Beschreibungsspezifikationen
aufgeführten Reihenfolgen sortiert.
JSON-Arrays werden mit einer 4 begonnen und mit einer 6 beendet.
Elemente dieser JSON-Arrays werden mit einer 5 statt mit einem Komma getrennt.
In JSON-Arrays von Beschreibungen kommen entweder nur Beschreibungen anderer
Objekte oder nur Integer vor.
Diese werden aufsteigend nach ihren IDs bzw. aufsteigend nach ihrem Wert
sortiert.

Alle JSON-Objekteigenschaften werden ohne Schlüssel, Doppelpunkte, geschwungene
Klammern und zusätzliche Leerzeichen oder Zeilenumbrüche konkateniert.
Alle JSON-Arrayelemente werden ohne eckige Klammern und zusätzliche
Leerzeichen oder Zeilenumbrüche konkateniert.
Die doppelten Anführungszeichen um Zeichenketten werden entfernt.
Da alle sechs Trennzeichen nur als Escape-Sequenzen, nicht aber direkt im
JSON-Format vorkommen können\footnote{Vgl. \enquote{RFC 7159}, Kapitel 7
\enquote{Strings}, \url{https://tools.ietf.org/html/rfc7159}} ist diese
Darstellung eindeutig.

% TODO: Update example
\par Im folgenden beispielhaft eine Verabredung mit folgenden Details:
\begin{itemize}
    \item Besitzt die ID 12
    \item Trägt den Titel \enquote{Beispielverabredung 1010}
    \item Erstellt vom Benutzerkonto mit ID 200
    \item Beginnt am 01. Januar 2018 um 00:00:00.000
    \item Endet am 03. Januar 2018 um 18:30:00.000
    \item Findet auf der Position 49.011978 N, 8.416377 E statt
    \item Die Benutzerkonten mit den IDs 200, 3 und 43378 sind als Teilnehmer
        eingetragen
\end{itemize}

Eine mögliche Kodierung der ausführlichen Beschreibung im JSON-Format kann wie
folgt aussehen:
\begin{lstlisting}[language=json,firstnumber=1]
{   "type" : "event",
    "id": 2,
    "title": "Beispielverabredung 1010",
    "creator": 200,
    "time-start": 1514761200000,
    "time-end": 1515000600000,
    "latitude": 49.011978,
    "longitude": 8.416377,
    "participants": [ 3, 200, 43378 ]
}
\end{lstlisting}

\newcommand*\circled[1]{\tikz[baseline=(char.base)]{
            \node[shape=circle,draw,inner sep=1pt] (char) {\scriptsize #1};}}

Die normalisierte Darstellung dieser Verabredung sieht wie folgt aus.
Die Steuerzeichen werden durch \circled{1}, \circled{2}, \circled{3} etc.
dargestellt.\\
\textit{Hinweis: Die Zeilenumbrüche dienen nur der besseren Lesbarkeit und sind
nicht in der normalisierten Darstellung enthalten.}

\begin{lstlisting}[language=norm,firstnumber=1]
(*\circled{1}*)"event"(*\circled{2}*)2(*\circled{2}%
*)"Beispielverabredung 1010"(*\circled{2}*)200(*\circled{2}*)1514761200000(*%
\circled{2}*)
1515000600000(*\circled{2}*)49.011978(*\circled{2}*)8.416377(*\circled{2}%
\circled{4}*)3(*\circled{5}*)200(*\circled{5}*)43378(*\circled{6}\circled{3}*)
\end{lstlisting}

\par Ein beispielhafter Algorithmus für die Normalisierung ist im folgenden
als Python-Implementierung angegeben:

\pythonexternal{scripts/normalize.py}

\section{Abrufen von Aktualisierungen}\label{sec:get-updates}
Zum Abrufen von Aktualisierungen gibt es mehrere Möglichkeiten.

\begin{itemize}
    \item Eine einzelne Anfrage an den Server, die alle aktuell für das eigene
        Benutzerkonto vorliegenden Änderungen abruft
    \item Eine Anfrage an den Server, die die aktuelle Verbindung als
        \enquote{Aktualisierungsverbindung} kennzeichnet.
        Auf einer Aktualisierungsverbindung werden alle Befehle des Klienten
        ignoriert.
        Erhält der Server für ein Benutzerkonto, welches eine offene
        Aktualisierungsverbindung zum Server hat, eine Aktualisierung, wird
        diese sofort über die Aktualisierungsverbindung an den Klienten
        gesendet.
    \item Push-Notifications.
        Diese Möglichkeit der Übertragung ist auf mehreren
        Smartphone-Betriebssystemen vertreten und erlaubt sehr
        ressourceneffiziente Mitteilungen an den Klienten.
        Für das verwendete Framework gibt es neben Google's Firebase noch
        wenige weitere quelloffene Kandidaten.
        Die Umsetzung dieser Übertragungsmöglichkeit wird sich erst während
        der Implementierung zeigen.
\end{itemize}


\end{document}
