\documentclass[parskip=full,11pt]{scrartcl}
\usepackage[utf8]{inputenc}
\usepackage[T1]{fontenc}
\usepackage[german]{babel}
\usepackage[useregional]{datetime2}
\usepackage[pdfborderstyle={/S/U/W 0}]{hyperref}
\usepackage[nameinlink]{cleveref}
\usepackage[section]{placeins}
\usepackage[top=2.5cm, bottom=2.5cm, left=4cm, right=3cm]{geometry}
\usepackage{xcolor}
\usepackage{graphicx}
\usepackage{csquotes}
\usepackage{amsmath} % for $\text{}$
\usepackage{changelog} % local .sty file
\usepackage{enumitem}
\usepackage{algorithm}
\usepackage{algorithmicx}
\usepackage{algpseudocode}

\setlist{nosep}

\newcommand\urlpart[2]{$\underbrace{\text{\texttt{#1}}}{\text{#2}}$}
\raggedbottom
\crefname{figure}{Abb}{Abb}

\newcommand\producttitle{treff.}
\newcommand\protocolversion{0.3}
\hypersetup{
	pdftitle={\producttitle~- Änderungsdokument~-~Implementierung},
	bookmarks=true,
}

% section numbers in margins:
\renewcommand\sectionlinesformat[4]{%
    \ifstr{#1}{subsubsection}{%
        \makebox[0pt][r]{#4}%
    }{%
        \makebox[0pt][r]{#3}#4%
    }%
}%

% header & footer
\usepackage{scrlayer-scrpage}
%\lofoot{\today}
%\refoot{\today}
\pagestyle{scrheadings}
%\let\raggedsection\centering

\title{\includegraphics[width = 80mm]{images/logo_crop.png}}
\subtitle{\huge Änderungsdokument\\Implementierung}
\author{Lukas Dippon
        \and Jens Kienle
        \and Matthias Noll
        \and Fabian Röpke
        \and Tim Schmidt
        \and Simon Vögele}

\begin{document}

\maketitle
\thispagestyle{empty} % removed page number from title

\pagebreak
\tableofcontents

%%%%%%%%%%%%%%%%%%%
\pagebreak
\section{Entwurf}

\subsection{Kontakte}
Ursprünglich wurde mit der Pflichtanforderung F20 im Pflichtenheft implizit 
festgelegt, dass das Hinzufügen von Kontakten keiner Zustimmung bedarf. Da so 
auch eine ungewollte Kontaktaufnahme durch Fremde leicht möglich ist, wird das 
einfache Hinzufügen durch das Senden einer Freundschaftsanfrage ersetzt.

%%%%%%%%%%%%%%%%%%%
\pagebreak
\section{Server}

\subsection{Datenmodell}
	\begin{itemize}
	\item Da die Gruppenzugehörigkeit eines Benutzers zu einer Gruppe eindeutig
		durch die Verbindung der Benutzer-ID und Gruppen-ID identifiziert wird
		und die ID der Gruppenzugehörigkeit nicht gebraucht wird, wurde sie
		entfernt und als Primärschlüssel durch die Verbindung von Benutzer-ID
		und Gruppen-ID ersetzt.
	\item Die Gruppenzugehörigkeit speichert nun zusätzlich den Zeitpunkt bis
		zu dem ein Benutzer seinen Standort mit der Gruppe teilt.
	\end{itemize}

\subsection{Permissions-Enum}
Das Permission Enum wurde aus dem SQL-Paket in das Haupt-Paket bewegt, da es
nicht spezifisch für die SQL-Implementation ist und auch allen alternativen
Datenbank-Implementationen zur Verfügung stehen muss.

\subsection{Verwendung eines Command-Entwurfsmusters}
Da offensichtlicherweise Gottklassen schlecht zu testen und zu warten sind und
deshalb vermieden werden sollten, wurde die Funktionalität der Befehle in Form
einer Befehlsstruktur aus der Klasse "Requesthandler" ausgelagert.

\subsection{Rückgabewert des RequestHandlers}
Der String-Rückgabewert des RequestHandlers an den WebSocketEndPoint, früher
Connectionhandler, wurde zu einer eigenen Klasse. Diese unterstützt neben der
Weiterleitung der Antwort an den Klienten weitere Informationen für den
WebSocketEndPoint. Diese wird genutzt, um beim Anfragen einer
PersistentConnection durch den Klienten das Verhalten des WebSocketEndpoint zu
verändern.

\subsection{Änderungen im SQL-Paket}
	\begin{itemize}
		\item Die Erstellung der Datenbank-Tabellen wurde in die seperate
			Klasse \enquote{SQLDatabase} ausgelagert.
	\end{itemize}

\subsection{Interessante Abhängigkeiten}
	\begin{itemize}
		\item \enquote{Jackson}:
			(De-)Serialisierung von JSON-kodierter Ein- und Ausgabe
		\item \enquote{Apache Commons DBUtils}:
			Vereinfachter Zugang zu JDBC-Methoden
		\item \enquote{MariaDB4j}:
			MySQL-in-a-Jar für einfaches Testen von Methoden, die eine
			Verbindung zur MySQL-Datenbank benötigen
	\end{itemize}

\subsection{Verschiedenes}
	\begin{itemize}
	\item Für einige früh abzufangende Fehlerzustände wurden neue Exceptions
		eingeführt.
	\end{itemize}


%%%%%%%%%%%%%%%%%%%
\pagebreak
\section{Klient}

\subsection{TreffPunkt}
Für einzelne Daten von primitiven Datentypen bietet es sich an, diese in
Shared-Preferences zu speichern. Allerdings wird ein Context benötigt, um auf
diese zuzugreifen. Um Klassen, welche keinen eigenen Context besitzen, trotzdem
den Zugriff auf Shared-Preferences zu ermöglichen, wurde die TreffPunkt Klasse
als Superklasse für Application eingeführt. Sie bietet statisch einen Context,
welcher nicht von irgendwelchen Lebenszyklen abhängig ist.\\
Dieser ermöglicht es auch, auf String Ressourcen zuzugreifen, in unserem Fall
zum Beispiel Errorcodes in mehreren Sprachen zur Verfügung zu stellen und diese
Contextunabhängig zu verwenden.

\subsection{Befehle}
Die interne Umsetzung der Netzwerkbefehle wurde um das Commands Package
erweitert. Nach dem Command Design Pattern werden nun die einzelnen Server
Befehle durch Objekte repräsentiert. Die Hauptfunktion der neuen Klassen
besteht zum einen darin, Objekte für die JSON Serialisierung zu verwalten. Zum
anderen werden von dort aus aber auch die lokalen Datenbankzugriffe (und
andere Reaktionen auf Antworten des Servers) gesteuert. So wird vermieden, dass
die ViewModels auf Reaktionen des Servers warten müssen und die UI einfriert.
Um eine zu enge Kopplung zu verhindern, wird die Ausführung eines Befehls
weiterhin die jeweilige Methode des RequestEncoders veranlasst\\
Zudem gibt es ein weiteres Package, welches die im Protokoll beschriebenen
Entitäten enthält. Dieses ist nötig, um eine gewisse Konsistenz zwischen Server
und Klient zu schaffen.\\
Für die bereits angesprochene JSON-Serialisierung wird sowohl server-, als
auch klientseitig die Jackson-Library verwendet.

\subsection{Karte}
Da bereits die Karte vom Open Street Map Projekt stammt, wurde noch das 
osmdroid Bonuspaket hinzugefügt. Dieses Paket ermöglicht zum Beispiel 
das Clustering von Positionen.\\
Desweiteren wurde noch ein marker Paket erstellt, in dem unter anderem die 
Marker für Benutzer oder Events enthalten sind.

\subsection{Datenbank}
Events enthalten nun die eindeutige Identifikationsnummer der Gruppe,
zu der sie gehören. So müssen Events nicht innerhalb der Gruppe
gespeichert oder referenziert werden und können leichter in der
Datenbank gefunden werden.\\
User Entitäten haben nun weitere Felder, die angeben, ob der aktuelle Benutzer
mit ihnen befreundet ist, sie blockiert hat, eine Freundschaftsanfrage
gesendet oder bekommen hat.\\
Außerdem wurde die eigene Position Entität durch android.location.Location
ersetzt und der TypeConverter angepasst. Diese Klasse speichert bereits
einen Zeitpunkt und macht so das Speichern des Zeitpunktes des letzten
Positionsupdates überflüssig.

\subsection{Ortung}
Der GPSProvider erbt nun von Service, nicht von IntentService, da dieser
nur eine Anfrage gleichzeitig bearbeiten könnte. Dies benötigen wir
zum Beispiel, um mehreren Gruppen gleichzeitig eine Position zur
Verfügung zu stellen. Dieser Service und das MapViewModel selbst sind
jeweils beide LocationListener.\\
Zusätzlich wurde der GPSProviderManager hinzugefügt, um den
GPSProvider zu starten und die Anfragen zu verwalten.

\subsection{Gruppeneinstellungen}
Der Einstellungsbildschirm der Gruppe ist nun eine Activity statt einem
Fragment, sodass die Historie der vergangenen Bildschirme und die
Navigation einfacher und sinnvoller umgesetzt werden können.\\
Das zugehörige ViewModel wurde entfernt, stattdessen wird das
GroupViewModel verwendet, um Redundanz zu vermeiden.\\
Um die Gruppenmitglieder anzuzeigen, wurde ein MemberListAdapter
hinzugefügt.

\subsection{Über uns}
Eine Activity zum Anzeigen von Informationen über die App, Entwickler und
Lizenz wurde in einem neuen dazugehörigen Paket hinzugefügt.

\subsection{Login}
Die LoginActivity hält nun entweder ein RegisterFragment oder das neu
hinzugefügte LoginFragment.

\subsection{Dependency Injection}
Die Klasse AppModule mit Konfigurationen für Dagger und Dagger selbst
wurden entfernt. Die als Singleton implementierte ViewModelFactory stellt
nun alle repositories für die ViewModels bereit.

\subsection{ViewModels}
Dem Klient wurde ein util Paket hinzugefügt, das einige Hilfsklassen beinhaltet.\\
Dazu gehören die bereits erwähnte TreffPunkt Klasse und die
ViewModelFactory, sowie:
\begin{itemize}
\item
SingleLiveEvent

\item
State

\item
ViewCall

\end{itemize} 
Einige Methodenaufrufe in ViewModels benötigen einen Context. In einem
SingleLiveEvent Objekt wird dieser Aufruf festgehalten. Es hält einen State,
also einen Wert aus dem ViewCall enum und einen optionalen Parameter.\\
Der Context, also das Fragment oder die Activity, observiert das SingleLiveEvent und
kann entsprechend auf Änderungen reagieren.


\subsection{Paging}
Es werden jetzt PagedListAdapter statt \enquote{normalen} Adaptern verwendet,
sodass nur beispielweise 30 ListItems geladen werden. Dies spart Netzwerkbandbreite
und macht die App responsiver.

\subsection{NavigationDrawer}
Die Abstraktionsebene der TabbedNavigationActivity wurde entfernt, da
sie keinen zusätzlichen Nutzen bietet. Die HomeActivity ist die einzige
Activity, die den NavigationDrawer darstellt und Tabs hält.\\
Das NavigationDrawerFragment wird nun nur im xml verwendet, nicht als Java-Klasse.

\pagebreak
\section{Protokoll}

Zur Spezifikation der Netzwerkschnittstellen von Server und Klient wurde das
Netzwerkprotokoll in ein eigenes, englischsprachiges Dokument ausgelagert.
Hierbei haben sich einige Veränderungen zum Entwurf ergeben,
die im Folgenden erläutert werden.
Um hierbei Redundanz durch Dopplung mit dem Protokoll zu vermeiden,
werden diese Änderungen nur oberflächlich beleuchtet.

\subsection{Veränderung der Befehle}
	\begin{itemize}
		\item Statt TCP wird nun WebSocket verwendet, da WebSocket
			nachrichtenbasiert und nicht nur bytebasiert arbeiten kann und sich
			leichter in Verbindung mit einem Tomcat Server verwenden lässt.
		\item Da viele der Befehle kleinere Fehler, vor allem im Bereich der
			Parameter und Rückgabewerte enthielten, welche erst im Laufe der
			Implementierung entdeckt worden sind, wurden diese angepasst.
		\item Da der Vorgang des Zurücksetzens des Passworts aus zwei und nicht
			wie fälschlicherweise im Entwurfsheft dokumentiert aus einem
			Schritt besteht, wurde der zugehörige Befehl zweigeteilt.
		\item Die Verwendung von Freundschaftsanfragen anstelle des direkten
			Hinzufügens verursachte diverse Änderungen der Befehle.
			So wurde der Befehl Befehl \enquote{add-contact} entfernt und
			stattdessen die Befehle \enquote{send-contact-request},
			\enquote{cancel-contact-request}, \enquote{accept-contact-request}
			und \enquote{reject-contact-request} hinzugefügt.
		\item Zur Vereinfachungen des Registrierungsvorgangs auf Seiten der
			Benutzer wurde die Angabe einer E-Mail-Adresse optional gemacht.
		\item Um den Vorgang, Klienten über neue Updates zu informieren, zu
			erleichtern, wurde mit dem Befehl
			\enquote{request-persistent-connection} die Möglichkeit eingebaut,
			eine bestehende WebSocket-Verbindung offenzuhalten und komplett dem
			Benachrichtigungsvorgang zu dedizieren.
	\end{itemize}

\subsection{Umgestaltung einiger Fehlercodes}
	Aufgrund des Änderns und Hinzufügens von API-Befehlen
	sind weitere potenzielle Fehlerszenarien entstanden oder entdeckt worden.
	Dies hatte zur Folge, dass Fehlercodes angepasst oder hinzugefügt werden
	mussten.
	Desweiteren wurden diese in diesem Zusammenhang neu kategorisiert,
	sortiert und passend nummeriert, wodurch sich ein vierstelliger Code
	anstelle eines dreistelligen ergab.

\end{document}
