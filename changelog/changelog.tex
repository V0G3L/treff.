\documentclass[parskip=full,11pt]{scrartcl}
\usepackage[utf8]{inputenc}
\usepackage[T1]{fontenc}
\usepackage[german]{babel}
\usepackage[useregional]{datetime2}
\usepackage[pdfborderstyle={/S/U/W 0}]{hyperref}
\usepackage[nameinlink]{cleveref}
\usepackage[section]{placeins}
\usepackage[top=2.5cm, bottom=2.5cm, left=4cm, right=3cm]{geometry}
\usepackage{xcolor}
\usepackage{graphicx}
\usepackage{csquotes}
\usepackage{amsmath} % for $\text{}$
\usepackage{changelog} % local .sty file
\usepackage{enumitem}
\usepackage{algorithm}
\usepackage{algorithmicx}
\usepackage{algpseudocode}

\setlist{nosep}

\newcommand\urlpart[2]{$\underbrace{\text{\texttt{#1}}}{\text{#2}}$}
\raggedbottom
\crefname{figure}{Abb}{Abb}

\newcommand\producttitle{treff.}
\newcommand\protocolversion{0.3}
\hypersetup{
	pdftitle={\producttitle~- Änderungsdokument},
	bookmarks=true,
}

% section numbers in margins:
\renewcommand\sectionlinesformat[4]{%
    \ifstr{#1}{subsubsection}{%
        \makebox[0pt][r]{#4}%
    }{%
        \makebox[0pt][r]{#3}#4%
    }%
}%

% header & footer
\usepackage{scrlayer-scrpage}
%\lofoot{\today}
%\refoot{\today}
\pagestyle{scrheadings}
%\let\raggedsection\centering

\title{\includegraphics[width = 80mm]{images/logo_crop.png}}
\subtitle{\huge Änderungsdokument}
\author{Lukas Dippon
        \and Jens Kienle
        \and Matthias Noll
        \and Fabian Röpke
        \and Tim Schmidt
        \and Simon Vögele}

\begin{document}

\maketitle
\thispagestyle{empty} % removed page number from title

\pagebreak
\tableofcontents

%%%%%%%%%%%%%%%%%%%
\pagebreak
\section{Entwurf}



\pagebreak
\section{Server}

\subsection{Datenmodell}

\subsection{Verwendung eines Command-Entwurfsmusters}
	\begin{itemize}
	\item Da offensichtlicherweise Gottklassen schlecht zu testen und zu warten sind und
	aus vielen weiteren Gründen vermeidet werden sollten, wurde die Funktionalität
	der Befehle in Form einer Befehlsstruktur aus der Klasse "Requesthandler"
	ausgelagert.
	\item Um unübersichtliche Fallunterscheidungen über Präfixe von Strings
	zu vermeiden, wurde der Rückgabetyp des Requesthandlers in den speziell
	hierfür angeferigten Typ "RequestHandlerResponse" geändert


\pagebreak
\section{Klient}



\pagebreak
\section{Protokoll}

Zur Spezifikation der API wurde ein Protokoll angelegt.
Hierbei haben sich einige Veränderungen zum Entwurf ergeben,
die im Folgenden aufgezählt und erläutert werden.
Um hierbei Redundanz durch Dopplung mit dem Protokoll zu vermeiden,
werden diese Änderungen nur oberflächlich beleuchtet.

\subsection{Veränderung der Befehle}
	\begin{itemize}
	\item Da viele der Befehle kleine Logikfehler vor allem
	im Bereich der Parameter und Rückgabewerte enthielten,
	welche leider erst im Laufe der Implementierung entdeckt worden waren,
	mussten diese geringfügig angepasst und verändert werden.
	\item Da der Vorgang des Zurücksetzen des Passwortes aus zwei
	und nicht wie anfangs angenommen aus einem,
	wurde der zugehörige Befehl in 2 seperate aufgeteilt.
	\item Die Verwendung von Freundschaftsanfragen anstelle des direkten
	Hinzufügens konsequenterweise auch diverse Änderungen der Befehle.
	So wurde der Befehl Befehl "add-contact" entfernt und stattdessen die Befehle
	"send-contact-request", "cancel-cntact-request", "accept-contact-request" und
	"reject-contact-request" hinzugefügt.
	\item zur Vereinfachungen des Regestrierungsvorgangs auf Seiten der Benutzer
	sowie auf Seiten der Entwickler wurde die Angabe einer E-Mail optional gemacht
	und die Aktivierug sowie Bestätigung durch diese entfernt.
	\item um das Übermitteln von Updates an die Klienten zu handhaben wird eine
	sog. "persistant connection" verwendet, welche von Klienten über den neuen
	Befehl "request-persistant-connection" angefordert werden kann.
	\end{itemize}

\subsection{neues Design der Fehlercodes}
	Aufgrund dem Ändern und Hinzufügen von API-Befehlen
	sind diverse potentielle Fehler entstanden oder entdeckt worden.
	Dies hatte zur Folge, dass Fehlercods angepasst
	oder hinzugefügt werden mussten.
	Desweiteren wurden diese in diesem Zusammenhang neu kategoresiert,
	sortiert und passend nummeriert, wodurch sich ein vierstelliger Code
	anstelle eines dreistelligen ergab.



\end{document}
