\documentclass[parskip=full,11pt]{scrartcl}
\usepackage[utf8]{inputenc}
\usepackage[T1]{fontenc}
\usepackage[german]{babel}
\usepackage[useregional]{datetime2}
\usepackage[pdfborderstyle={/S/U/W 0}]{hyperref}
\usepackage[nameinlink]{cleveref}
\usepackage[section]{placeins}
\usepackage[top=2.5cm, bottom=2.5cm, left=4cm, right=3cm]{geometry}
\usepackage{xcolor}
\usepackage{graphicx}
\usepackage{csquotes}
\usepackage{amsmath} % for $\text{}$
\usepackage{changelog} % local .sty file
\usepackage{enumitem}
\usepackage{algorithm}
\usepackage{algorithmicx}
\usepackage{algpseudocode}

\setlist{nosep}

\newcommand\urlpart[2]{$\underbrace{\text{\texttt{#1}}}{\text{#2}}$}
\raggedbottom
\crefname{figure}{Abb}{Abb}

\newcommand\producttitle{treff.}
\newcommand\protocolversion{0.3}
\hypersetup{
	pdftitle={\producttitle~- Änderungsdokument~-~Implementierung},
	bookmarks=true,
}

% section numbers in margins:
\renewcommand\sectionlinesformat[4]{%
    \ifstr{#1}{subsubsection}{%
        \makebox[0pt][r]{#4}%
    }{%
        \makebox[0pt][r]{#3}#4%
    }%
}%

% header & footer
\usepackage{scrlayer-scrpage}
%\lofoot{\today}
%\refoot{\today}
\pagestyle{scrheadings}
%\let\raggedsection\centering

\title{\includegraphics[width = 80mm]{images/logo_crop.png}}
\subtitle{\huge Änderungsdokument\\Implementierung}
\author{Lukas Dippon
        \and Jens Kienle
        \and Matthias Noll
        \and Fabian Röpke
        \and Tim Schmidt
        \and Simon Vögele}

\begin{document}

\maketitle
\thispagestyle{empty} % removed page number from title

\pagebreak
\tableofcontents

%%%%%%%%%%%%%%%%%%%
\pagebreak
\section{Entwurf}



\pagebreak
\section{Server}

\subsection{Datenmodell}
	\begin{itemize}
	\item Da die Gruppenzugehörigkeit eines Benutzers zu einer Gruppe eindeutig
		durch die Verbindung der Benutzer-ID und Gruppen-ID identifiziert wird
		und die ID der Gruppenzugehörigkeit nicht gebraucht wird, wurde sie
		entfernt und als Primärschlüssel durch die Verbindung von Benutzer-ID
		und Gruppen-ID ersetzt.
	\item Die Gruppenzugehörigkeit speichert nun zusätzlich den Zeitpunkt bis
		zu dem ein Benutzer seinen Standort mit der Gruppe teilt.
	\end{itemize}

\subsection{Permissions-Enum}
Das Permission Enum wurde aus dem SQL-Paket in das Haupt-Paket bewegt, da es
nicht spezifisch für die SQL-Implementation ist und auch allen alternativen
Datenbank-Implementationen zur Verfügung stehen muss.

\subsection{Verwendung eines Command-Entwurfsmusters}
	\begin{itemize}
	\item Da offensichtlicherweise Gottklassen schlecht zu testen und zu warten
		sind und deshalb vermieden werden sollten, wurde
		die Funktionalität der Befehle in Form einer Befehlsstruktur aus der
		Klasse "Requesthandler" ausgelagert.
	\item Der String-Rückgabewert des RequestHandlers an den WebSocketEndPoint,
		früher Connectionhandler, wurde zu einer eigenen Klasse.
		Diese unterstützt neben der Weiterleitung der Antwort an den Klienten
		weitere Informationen für den WebSocketEndPoint.
		Diese wird genutzt, um beim Anfragen einer PersistentConnection durch
		den Klienten das Verhalten des WebSocketEndpoint zu verändern.
	\end{itemize}

\subsection{Änderungen im SQL-Paket}
	\begin{itemize}
		\item Die Erstellung der Datenbank-Tabellen wurde in die seperate
			Klasse \enquote{SQLDatabase} ausgelagert.
	\end{itemize}

\subsection{Interessante Abhängigkeiten}
	\begin{itemize}
		\item \enquote{Jackson}:
			(De-)Serialisierung von JSON-kodierter Ein- und Ausgabe
		\item \enquote{Apache Commons DBUtils}:
			Vereinfachter Zugang zu JDBC-Methoden
		\item \enquote{MariaDB4j}:
			MySQL-in-a-Jar für einfaches Testen von Methoden, die eine
			Verbindung zur MySQL-Datenbank benötigen
	\end{itemize}

\subsection{Verschiedenes}
	\begin{itemize}
	\item Für einige früh abzufangende Fehlerzustände wurden neue Exceptions
		eingeführt.
	\end{itemize}

\pagebreak
\section{Klient}



\pagebreak
\section{Protokoll}

Zur Spezifikation der Netzwerkschnittstellen von Server und Klient wurde das
Netzwerkprotokoll in ein eigenes, englischsprachiges Dokument ausgelagert.
Hierbei haben sich einige Veränderungen zum Entwurf ergeben,
die im Folgenden erläutert werden.
Um hierbei Redundanz durch Dopplung mit dem Protokoll zu vermeiden,
werden diese Änderungen nur oberflächlich beleuchtet.

\subsection{Veränderung der Befehle}
	\begin{itemize}
		\item Statt TCP wird nun WebSocket verwendet, da WebSocket
			nachrichtenbasiert und nicht nur bytebasiert arbeiten kann und sich
			leichter in Verbindung mit einem Tomcat Server verwenden lässt.
		\item Da viele der Befehle kleinere Fehler, vor allem im Bereich der
			Parameter und Rückgabewerte enthielten, welche erst im Laufe der
			Implementierung entdeckt worden sind, wurden diese angepasst.
		\item Da der Vorgang des Zurücksetzens des Passworts aus zwei und nicht
			wie fälschlicherweise im Entwurfsheft dokumentiert aus einem
			Schritt besteht, wurde der zugehörige Befehl zweigeteilt.
		\item Die Verwendung von Freundschaftsanfragen anstelle des direkten
			Hinzufügens verursachte diverse Änderungen der Befehle.
			So wurde der Befehl Befehl \enquote{add-contact} entfernt und
			stattdessen die Befehle \enquote{send-contact-request},
			\enquote{cancel-contact-request}, \enquote{accept-contact-request}
			und \enquote{reject-contact-request} hinzugefügt.
		\item Zur Vereinfachungen des Registrierungsvorgangs auf Seiten der
			Benutzer wurde die Angabe einer E-Mail-Adresse optional gemacht.
		\item Um den Vorgang, Klienten über neue Updates zu informieren, zu
			erleichtern, wurde mit dem Befehl
			\enquote{request-persistent-connection} die Möglichkeit eingebaut,
			eine bestehende WebSocket-Verbindung offenzuhalten und komplett dem
			Benachrichtigungsvorgang zu dedizieren.
	\end{itemize}

\subsection{Umgestaltung einiger Fehlercodes}
	Aufgrund des Änderns und Hinzufügens von API-Befehlen
	sind weitere potenzielle Fehlerszenarien entstanden oder entdeckt worden.
	Dies hatte zur Folge, dass Fehlercodes angepasst oder hinzugefügt werden
	mussten.
	Desweiteren wurden diese in diesem Zusammenhang neu kategorisiert,
	sortiert und passend nummeriert, wodurch sich ein vierstelliger Code
	anstelle eines dreistelligen ergab.

\end{document}
