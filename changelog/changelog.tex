\documentclass[parskip=full,11pt]{scrartcl}
\usepackage[utf8]{inputenc}
\usepackage[T1]{fontenc}
\usepackage[german]{babel}
\usepackage[useregional]{datetime2}
\usepackage[pdfborderstyle={/S/U/W 0}]{hyperref}
\usepackage[nameinlink]{cleveref}
\usepackage[section]{placeins}
\usepackage[top=2.5cm, bottom=2.5cm, left=4cm, right=3cm]{geometry}
\usepackage{xcolor}
\usepackage{graphicx}
\usepackage{csquotes}
\usepackage{amsmath} % for $\text{}$
\usepackage{changelog} % local .sty file
\usepackage{enumitem}
\usepackage{algorithm}
\usepackage{algorithmicx}
\usepackage{algpseudocode}

\setlist{nosep}

\newcommand\urlpart[2]{$\underbrace{\text{\texttt{#1}}}{\text{#2}}$}
\raggedbottom
\crefname{figure}{Abb}{Abb}

\newcommand\producttitle{treff.}
\newcommand\protocolversion{0.3}
\hypersetup{
	pdftitle={\producttitle~- Änderungsdokument},
	bookmarks=true,
}

% section numbers in margins:
\renewcommand\sectionlinesformat[4]{%
    \ifstr{#1}{subsubsection}{%
        \makebox[0pt][r]{#4}%
    }{%
        \makebox[0pt][r]{#3}#4%
    }%
}%

% header & footer
\usepackage{scrlayer-scrpage}
%\lofoot{\today}
%\refoot{\today}
\pagestyle{scrheadings}
%\let\raggedsection\centering

\title{\includegraphics[width = 80mm]{images/logo_crop.png}}
\subtitle{\huge Änderungsdokument}
\author{Lukas Dippon
        \and Jens Kienle
        \and Matthias Noll
        \and Fabian Röpke
        \and Tim Schmidt
        \and Simon Vögele}

\begin{document}

\maketitle
\thispagestyle{empty} % removed page number from title

\pagebreak
\tableofcontents

%%%%%%%%%%%%%%%%%%%
\pagebreak
\section{Entwurf}



\pagebreak
\section{Server}

\subsection{Datenmodell}
\begin{itemize}
\item Da die Gruppenzugehörigkeit eines Benutzers zu einer Gruppe eindeutig
durch die Verbindung der Benutzer-Id und Gruppen-Id identifiziert wird und die
Id der Gruppenzugehörigkeit nirgends gebraucht wird, wurde sie entfernt und als
Primärschlüssel durch die Verbindung von Benutzer-Id und Gruppen-Id ersetzt.
\item Die Gruppenzugehörigkeit speichert nun zusätzlich den Zeitpunkt bis zu
dem ein Benutzer seinen Standort mit der Gruppe teilt, da diese Zeit
einzigartig für eine Gruppenzugehörigkeit ist.
\end{itemize}

\subsection{Verwendung eines Command-Entwurfsmusters}
Da offensichtlicherweise Gottklassen schlecht zu testen und zu warten sind und
aus vielen weiteren Gründen vermieden werden sollten, wurde die Funktionalität
der Befehle in Form einer Befehlsstruktur aus der Klasse Requesthandler
ausgelagert.

\subsection{Rückgabetyp des Requesthandlers}
Um unübersichtliche Fallunterscheidungen über Präfixe von Strings
zu vermeiden, wurde der Rückgabetyp des Requesthandlers in den speziell
hierfür angeferigten Typ RequestHandlerResponse geändert.

\subsection{Permissions-Enum}
Das Permission Enum wurde aus dem SQL-Paket in das Haupt-Paket bewegt, da es
nicht spezifisch für die SQL-Implementation ist und auch allen alternativen
Datenbank-Implementationen zur Verfügung stehen muss.

\pagebreak
\section{Klient}



\pagebreak
\section{Protokoll}

Zur Spezifikation der API wurde ein Protokoll angelegt.
Hierbei haben sich einige Veränderungen zum Entwurf ergeben,
die im Folgenden aufgezählt und erläutert werden.
Um hierbei Redundanz durch Dopplung mit dem Protokoll zu vermeiden,
werden diese Änderungen nur oberflächlich beleuchtet.

\subsection{Veränderung der Befehle}
	\begin{itemize}
	\item Da viele der Befehle kleine Logikfehler vor allem
	im Bereich der Parameter und Rückgabewerte enthielten,
	welche erst im Laufe der Implementierung entdeckt wurden,
	mussten diese geringfügig angepasst und verändert werden.
	\item Da der Vorgang des Zurücksetzens des Passwortes aus zwei
	und nicht wie anfangs angenommen aus einem besteht,
	wurde der zugehörige Befehl in 2 seperate aufgeteilt.
	\item Die Verwendung von Freundschaftsanfragen anstelle des direkten
	Hinzufügens hat auch diverse Änderungen der Befehle mit sich gezogen.
	So wurde der Befehl Befehl add-contact entfernt und stattdessen die
    Befehle send-contact-request, cancel-contact-request,
    accept-contact-request und reject-contact-request hinzugefügt.
	\item Zur Vereinfachungen des Registrierungsvorgangs auf Seiten der
    Benutzer sowie auf Seiten der Entwickler, wurde die Angabe einer E-Mail
    optional gemacht und die Aktivierug sowie Bestätigung durch diese entfernt.
	\item Um das Übermitteln von Updates an die Klienten zu handhaben wird eine
	sogenannten persistent connection verwendet, welche von Klienten über den neuen
	Befehl request-persistent-connection angefordert werden kann.
	\end{itemize}

\subsection{neues Design der Fehlercodes}
	Aufgrund dem Ändern und Hinzufügen von API-Befehlen
	sind diverse potentielle Fehler entstanden oder entdeckt worden.
	Dies hatte zur Folge, dass Fehlercodes angepasst
	oder hinzugefügt werden mussten.
	Desweiteren wurden diese in diesem Zusammenhang neu kategoresiert,
	sortiert und passend nummeriert, wodurch sich ein vierstelliger Code
	anstelle eines dreistelligen ergab.



\end{document}
